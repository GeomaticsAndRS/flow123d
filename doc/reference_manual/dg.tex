%%%%%%%%%%%%%%%%%%%%%%%%%%%%%%%%%%%%%%%%%%%%%%%%%%%%%%%%%%%%%%%%%%%%%
\section{Discontinuous Galerkin method}

\def\Eh{\mathcal E_h}       % edges of \Th
\def\Ehb{\mathcal E_{h,B}}  % edges of \Th on boundary
\def\Ehcom{\mathcal E_{h,C}}         % edges of \Th on interface with lower dimension
\def\Ehdir{\mathcal E_{h,D}}         % Dirichlet edges of \Th
\def\Ehint{\mathcal E_{h,I}}       % interior edges of \Th
\def\Ehneu{\mathcal E_{h,N}}         % Neumann or Robin edges of \Th
\def\avg#1{\left\{#1\right\}}
\def\wavg#1{\avg{#1}_{\omega}}


Models mentioned in sections \ref{sc:transport_model} and \ref{sc:heat} are collectively described
 as a system of abstract advection-diffusion equations on domains $\Omega_d$, $d=1,2,3$,
 connected by communication terms.
Consider for $d=1,2,3$ the equation 
\begin{subequations}
 \label{eq:abstr_system}
 \begin{equation}
  \partial_t u_d + \div(\vc b u_d) - \div(\tn A\nabla u_d) = f + q(u_{d+1},u_d) \mbox{ in }\Omega_d
 \end{equation}
 with initial and boundary conditions
 \begin{align}
  u_d(0,\cdot) &= u^0 &&\mbox{ in }\Omega_d,\\
  \label{eq:bc_abstr_dir} u_d &= u^D &&\mbox{ on }\Gamma^D_d,\\
  \label{eq:bc_abstr_neu} (\vc b u_d-\tn A\nabla u_d)\cdot\vc n &= f^N + \sigma^R(u_d - u^D) &&\mbox{ on }\Gamma^N_d,\\
  (\vc b u_d-\tn A\nabla u_d)\cdot\vc n &= q(u_d,u_{d-1}) &&\mbox{ on } \Gamma^C_d:=\overline\Omega_d\cap\overline\Omega_{d-1}.
 \end{align}
 The communication term $q(u_{d+1},u_d)$ has the form
 \begin{equation}
  q(u_{d+1},u_d) =
  \begin{cases}
      \alpha u_{d+1} + \beta u_d
    & \mbox{ in }\Gamma^C_{d+1},~d=1,2,\\ 0
    & \mbox{ on }\Omega_d\setminus\Gamma^C_{d+1},~d=1,2,\mbox{ and for }d=0,3.
  \end{cases}
 \end{equation}
\end{subequations}
System \eqref{eq:abstr_system} is spatially discretized by the discontinuous Galerkin method
 with weighted averages,
 which was derived for the case of one domain in \cite{ern_stephansen_zunino}
 (for a posteriori estimate see \cite{ern2010guaranteed}).
For time discretization we use the implicit Euler method.

Let $\tau$ and $h$ denote the time step and the spatial discretization parameter, respectively.
For a regular splitting $\Th^d$ of $\Omega^d$, $d=1,2,3$, into simplices,
 whose norm (the longest edge) is $h$, we define the following sets of element sides:
\begin{align*}
 &\Eh^d &&\mbox{sides of all elements in $\Th^d$ (i.e. triangles for $d=3$, lines for $d=2$ and nodes for $d=1$)},\\
 &\Ehint^d &&\mbox{interior sides},\\
 &\Ehb^d &&\mbox{outer sides},\\
 &\Ehdir^d(t) &&\mbox{sides, where the Dirichlet condition \eqref{eq:bc_abstr_dir} is given},\\
 &\Ehneu^d(t) &&\mbox{sides, where the Neumann or Robin condition \eqref{eq:bc_abstr_neu} is given},\\
 &\Ehcom^d &&\mbox{sides coinciding with $\Gamma^C_d$}.
\end{align*}
For an interior side $E$ we denote by $T^-(E)$ and $T^+(E)$ the elements which share $E$.
The symbol $\n$ stands for the unit normal vector to $E$ pointing from $T^-(E)$ to $T^+(E)$.
The jump in values of a function $f$ between the adjacent elements is defined by $\jmp{f}=f_{|T^-(E)}-f_{|T^+(E)}$,
 similarly we introduce the average $\avg{f}=\frac12(f_{|T^-(E)} + f_{|T^+(E)})$
 and a weighted average $\wavg{f}=\omega f_{|T^-(E)} + (1-\omega) f_{|T^+(E)}$.
The weight $\omega$ is selected in a specific way (see \cite{ern_stephansen_zunino})
 taking into account the possible inhomogeneity of the tensor $\tn A$.

% Let us fix one substance and the space dimension $d$.
For every time step $t_k=k\tau$ we look for the discrete solution $u^{h,k}=(u_1^{h,k},u_2^{h,k},u_3^{h,k})\in V^h=\prod_{d=1}^3 V_d^h$,
 where
$$ V_d^h = \{v:\overline{\Omega^d}\to\R\where v_{|T}\in P_p(T)~\forall T\in\Th^d\} $$
is the space of piecewise polynomial functions of degree at most $p$ on elements $\Th^d$,
 generally discontinuous on interfaces of elements.
The initial condition for $u_d^{h,0}$ is defined as the $L^2$-projection of $u^0$ to $V_d^h$.
For $k=1,2,\ldots$, $u^{h,k}$ is given as the solution of the problem
\begin{equation*}
 \frac1\tau\sc{u^{h,k}-u^{h,k-1}}{v}_{V^h} + a^{h,k}(u^{h,k},v) = b^{h,k}(v) \quad \forall v\in V^h.
\end{equation*}
Here $\sc{f}{g}_{V^h}=\sum_{d=1}^d\sc{f}{g}_{\Omega^d}$, $\sc{f}{g}_{\Omega^d}=\int_{\Omega^d} f g$,
 and forms $a^{h,k}$, $b^{h,k}$
 are defined as follows:
% \begin{multline*}
\[
  a^{h,k}((u_1,u_2,u_3),(v_1,v_2,v_3))
   = \sum_{d=1}^3\bigg( a^{h,k}_d(u_d,v_d)
    - \sc{q(u_{d+1},u_d)}{v_d}_{\Omega^d}
    - \sum_{E\in\Ehcom^d(t_k)}\sc{q(u_d,u_{d-1})}{v_d}_E \bigg),
\]
% \end{multline*}
\[ b^{h,k}((v_1,v_2,v_3)) = \sum_{d=1}^3 b^{h,k}_d(v_d), \mbox{\hspace{11.7cm}} \]
\begin{align*}
 a^{h,k}_d(u,v) = &\sc{\tn A\nabla u}{\nabla v}_{\Omega^d}
 - \sc{\vc b u}{\nabla v}_{\Omega^d}\\
 &- \sum_{E\in\Ehint^d}\bigg(\sc{\wavg{\tn A\nabla u}\cdot\n}{\jmp{v}}_E + \Theta\sc{\wavg{\tn A\nabla v}\cdot\n}{\jmp{u}}_E
 + \sc{\vc b\cdot\n\avg{u}}{\jmp{v}}_E
 + \gamma_E\sc{\jmp{u}}{\jmp{v}}_E\bigg)\\
%  & + \sum_{E\in\Ehb^d}\sc{\vc b\cdot\n u}{v}_E
 &+ \sum_{E\in\Ehdir^d(t_k)}\bigg(\gamma_E\sc{u}{v}_E + \sc{\vc b\cdot\n u}{v}_E - \sc{\tn A\nabla u\cdot\vc n}{v}_E - \Theta\sc{\tn A\nabla v\cdot\vc n}{u}_E\bigg)\\
 &+ \sum_{E\in\Ehneu^d(t_k)}\sc{\sigma^R u}{v}_E,\\
% \end{multline*}
% 
% \begin{equation*}
 b^{h,k}_d(v) = &\sc{f}{v}_{\Omega^d} + \sum_{E\in\Ehdir^d(t_k)}\bigg(\gamma_E\sc{u^D}{v}_E - \Theta\sc{u^D}{\tn A\nabla v\cdot\vc n}_E\bigg)
 + \sum_{E\in\Ehneu^d(t_k)}\sc{\sigma^R u^D-f^N}{v}_E.
\end{align*}
The Dirichlet condition is here enforced by a penalty with an arbitrary parameter $\gamma_E>0$;
 its value influences the level of solution's discontinuity.
For $\gamma_E\to+\infty$ we obtain asymptotically (at least formally) the finite element method.
The constant $\Theta$ can take the values $-1$, $0$ or $1$,
 which correspond to the nonsymetric, incomplete and symetric variant, respectively, of the discontinuous Galerkin method.

% ??? NENÍ PŘEDCHOZÍ VĚTA BLBOST ???

% \paragraph{Communication between regions of the same dimension.}
In the case of lower dimension ($\Omega^1$, $\Omega^2$) we can consider more complex topology,
 where one side is shared by more than 2 elements.
For this situation, the method is generalized under the assumption of the so-called ideal mixing.
Suppose that a side $E$ is shared by elements $T_1,\ldots,T_n$.
Denote $q_i:=(\vc b\cdot\vc n)_{|T_i}$ the outflow from $T_i$, $i=1,\ldots,n$,
 and define $I^-:=\{i\where q_i\le 0\}$, $I^+:=\{i\where q_i>0\}$
 the index sets of all outflow and inflow elements, respectively.

For every pair $(i,j)\in I^+\times I^-$ we define the flux from $T_i$ to $T_j$ as
$$ q_{i\to j} := \frac{q_i q_j}{\sum_{k\in I^-}{q_k}}.$$
In the bilinear form $a_d^{h,k}$ we then replace the expression $\sc{\vc b\cdot\vc n\avg{u}}{\jmp{v}}_E$ by
$$ \sum_{(i,j)\in I^+\times I^-}\sc{q_{i\to j}\avg{u}}{\jmp{v}}_E, $$
where the operators $\avg{\cdot}$ and $\jmp{\cdot}$ are considered relative to the pair of elements $(T_i,T_j)$.
