
\section{Darcy Flow Model} \label{sec:darcy_flow}
We consider the simplest model for the velocity of the steady or unsteady flow in porous and fractured medium given by 
the Darcy flow:
\begin{equation}
    \label{eq:darcy}
    \vc w = -\tn K \grad H \quad\text{in }\Omega_d,\ \text{for $d=1,2,3$}.
\end{equation}
Here and later on, we drop the dimension index $d$ of the quantities if it can be deduced from the context.
In \eqref{eq:darcy}, $\vc w$ \units{}{1}{-1} is \href{http://en.wikipedia.org/wiki/Superficial_velocity}{the superficial velocity},
$\tn K_d$ is the conductivity tensor, and $H$ \units{}{1}{} is the piezometric head. The velocity $\vc w_d$ is related to the flux $\vc q_d$ 
\units{}{4-d}{-1} through
\[
    \vc q_d = \delta_d \vc w_d,
\]
where $\delta_d$ \units{}{3-d}{} is the \hyperA{DarcyFlowMH-Data::cross-section}{cross section} coefficient,
in particular $\delta_3=1$, $\delta_2$ \units{}{1}{} is the thickness of a fracture, and $\delta_1$ \units{}{2}{} is the cross-section of a channel.
The flux $\vc q_d\cdot\vc n$ is the volume of the liquid (water) that passes through a unit square ($d=3$),
unit line ($d=2$), or through a point ($d=1$) per one second. 
The conductivity tensor is given by the product \penalty-500
$\tn K_d = k_d \tn A_d$, where $k_d>0$ \units{}{1}{-1}
 is the \hyperA{DarcyFlowMH-Data::conductivity}{hydraulic conductivity}  and 
$\tn A_d$  is the 
$3\times 3$ dimensionless \hyperA{DarcyFlowMH-Data::anisotropy}{anisotropy tensor} which has to be symmetric and positive definite.
The piezometric-head $H_d$ is related to the pressure head
$h_d$ through
\begin{equation}
    \label{eq:piezo_head}
    H_d = h_d + z
\end{equation}
assuming that the gravity force acts in the negative direction of the $z$-axis. 
Combining these relations, we get the Darcy law in the form:
\begin{equation}
    \label{eq:darcy_flux}
    \vc q = -\delta k\tn A \grad (h+z)  \qquad\text{in }\Omega_d,\ \text{for $d=1,2,3$}.
\end{equation}

Next, we employ the continuity equation for saturated porous medium and the dimensional reduction from the preceding section
(with $w=u:=H$, $\vc j:=\vc w$, $\tn A:=\tn K$ and $\vc b:=\vc 0$), which yields:
\begin{equation}
    \label{eq:continuity}
    \prtl_t (\delta S\, h) + \div \vc q = F \qquad \text{in }\Omega_d,\ \text{for $d=1,2,3$},
\end{equation}
where  $S_d$ \units{}{-1}{} is the \hyperA{DarcyFlowMH-Data::storativity}{storativity} and $F_d$ \units{}{3-d}{-1} is 
the source term. In our setting the principal unknowns of the system 
(\ref{eq:darcy_flux}, \ref{eq:continuity}) are the pressure head $h_d$ and the flux $\vc q_d$.


The storativity (or the volumetric specific storage) $S_d>0$ can be expressed as
\begin{equation}
  S_d = \gamma_w(\beta_r + \vartheta \beta_w),
\end{equation}
where $\gamma_w$ \units{1}{-2}{-2} is the specific weight of water, $\vartheta$ \units{}{}{} is the porosity,
$\beta_r$ is compressibility of the bulk material of the pores (rock)
and $\beta_w$ is compressibility of the water, both with units \units{-1}{1}{-2}. For steady problems, we set $S_d=0$ for all dimensions $d=1,2,3$.
The source term $F_d$ on the right hand side of \eqref{eq:continuity} consists of the volume density of the 
\hyperA{DarcyFlowMH-Data::water-source-density}{water source} 
 $f_d$\units{}{}{-1} and flux from the from the higher dimension. 
Precise form of $F_d$ slightly differs for every dimension and will be discussed presently.

In $\Omega_3$ we simply have $F_3  = f_3$ \units{}{}{-1}.

In the set $\Omega_2 \cap \Omega_3$ the fracture is surrounded by at most one 3D surface from every side.
On $\prtl\Omega_3 \cap \Omega_2$ we prescribe a boundary condition of the Robin type:
\begin{align*}
        \vc{q}_3\cdot \vc n^{+} &= q_{32}^{+} =\sigma_{3} (h_3^{+}-h_2),\\
        \vc{q}_3\cdot \vc n^{-} &= q_{32}^{-} =\sigma_{3} (h_3^{-}-h_2),
\end{align*}
where $\vc{q}_3\cdot\vc n^{+/-}$ \units{}{1}{-1} is the outflow from $\Omega_3$, $h_3^{+/-}$ is
a trace of the pressure head in $\Omega_3$, $h_2$ is the pressure head in $\Omega_2$, and 
$\sigma_{3}$ \units{}{}{-1} is the transition coefficient given by (see section \ref{sc:ad_on_fractures} and \cite{martin_modeling_2005})
\[
\label{e:sigma3_law}
  \sigma_3 = \sigma_{32} \frac{2\tn K_2 :\vc n_2\otimes\vc n_2 }{\delta_2}.
\]
Here $\vc n_2$ is the unit normal to the fracture (sign does not matter).
On the other hand, the sum of the interchange fluxes $q_{32}^{+/-}$ forms
a volume source in $\Omega_2$.  Therefore $F_2$ \units{}{1}{-1} on the right hand side of \eqref{eq:continuity} is
given by
\begin{equation}
   \label{source_2D}
   F_2 = \delta_2 f_2 + (q_{32}^{+} + q_{32}^{-}).
\end{equation}

The communication between $\Omega_2$  and  $\Omega_1$ is similar.  However, in the 3D ambient space,
a 1D channel can join multiple 2D fractures $1,\dots, n$. Therefore, we have $n$
independent outflows from $\Omega_2$:
\begin{equation*}
        \vc{q}_2\cdot \vc n^{i} = q_{21}^{i} =\sigma_{2} (h_2^{i}-h_1),
\end{equation*}
where $\sigma_2$ \units{}{1}{-1} is the transition coefficient integrated over the width of the fracture $i$:
\[
\label{e:sigma2_law}
  \sigma_2 = \sigma_{21} \frac{2\delta_2^2\tn K_1:{\vc n_1^i}\otimes{\vc n_1^i}}{\delta_1}.
\]
Here $\vc n_1^i$ is the unit normal to the channel that is tangential to the fracture $i$.
Sum of the fluxes forms a part of $F_1$ \units{}{2}{-1}:
\begin{equation}
   \label{source_1D}
   F_1 = \delta_1 f_1 + \sum_{i=1}^n q_{21}^{i}. 
\end{equation}
We remark that the direct communication between 3D and 1D (e.g. model of a well) is not supported yet.
The \hyperA{DarcyFlowMH-Data::sigma}{transition coefficients} 
{$\sigma_{32}$} \units{}{}{} and
{$\sigma_{21}$} \units{}{}{} are independent scaling parameters which represent 
the ratio of the crosswind and the tangential conductivity in the fracture. For example, in the case of impermeable film
on the fracture walls one may choice $\sigma_{32} < 1$.

In order to obtain unique solution we have to prescribe boundary conditions. Currently we support three basic 
\hyperA{DarcyFlowMH-Data::bc-type}{types of boundary conditions}:

{\bf Dirichlet} boundary condition
\[
    h_d = h_d^D        \text{ on }\Gamma_d^D,
\]
where $h_d^D$ \units{}{1}{} is the \hyperA{DarcyFlowMH-Data::bc-pressure}{boundary pressure head} .
Alternatively one can prescribe the \hyperA{DarcyFlowMH-Data::bc-piezo-head}{boundary piezometric head}
$H_d^D$ \units{}{1}{} related to the pressure head through \eqref{eq:piezo_head}.

{\bf Neuman} boundary condition
\[
    \vc q_d \cdot \vc n = q_d^N         \text{ on }\Gamma_d^N,
\]
where $q_d^N$ \units{}{4-d}{-1} is the \hyperA{DarcyFlowMH-Data::bc-flux}{surface density of the water outflow}.

{\bf Robin} boundary condition 
\[
    \vc q_d \cdot \vc n = \sigma_d^R ( h_d - h_d^R)     \text{ on }\Gamma_d^R.
\]
where $h_d^R$ is the \Alink{DarcyFlowMH-Data::bc-pressure}{boundary pressure head} and
$\sigma_d^R$ \units{}{3-d}{-1}  
is the \hyperA{DarcyFlowMH-Data::bc-robin-sigma}{transition coefficient}.
As before one can also prescribe the \Alink{DarcyFlowMH-Data::bc-piezo-head}{boundary piezo head}
$H_d^R$ to specify $h_d^R$.

{\bf Seepage face} boundary condition 
\[
    h_d \le h_d^D\quad \text{and} \quad \vc q_d \cdot \vc n \ge q_d^N
\]    
while the equality holds in at least one inequality. This condition is used to model spring area. 
The former inequality
(usually with default value $h_d^D=0$) disallow non-zero water height on the surface, the later
inequality allow only out flow from the domain (i.e. spring).

{\bf River} boundary condition

This boundary condition models kind of free water surface, namely a watercourse.
In order to derive the inequalities similar to that in the Seepage face condition, we
start with a model that use the combined flux condition
\begin{equation}
    \label{bc:robin_flux}
    \vc q \cdot \vc n = \sigma (h - h_0) + q_i,
\end{equation}
where $\sigma$ is a kind of transmissivity of the river bedrock (see detailed discussion later on) and $q_i$ can represent e.g. precipitation.
From practical point of view, the river can be understood as an information that the underground water
is close to the surface. One possibility how to express this information in the model is to constrain the piezometric head:
\[
    H \ge H_c, \text{ or equivalently } h \ge H_c - z
\]
where $H$ is the piezometric head of the solution pressure $h$ and $H_c$ is the piezometric head constraint usually something little bit below
the river's altitude. Then we use \eqref{bc:robin_flux} to derive complementary constraint for the flux:
\[
    \vc q \cdot \vc n \ge \sigma(H_c - z - h_0) + q_i.
\]


Other possibility is to constrain the infiltration from the river into the underground domain:
\[
    \vc q \cdot \vc n \ge q_c
\]
and then use \eqref{bc:robin_flux} to obtain complementary condition for the pressure:
\[
    h \ge \frac{q_c - q_i}{\sigma} + h_0.
\]



%\begin{equation}
%    \vc q^d \cdot \vc n \ge q_d^N
%\end{equation}
%
%where $\sigma$ \units{}{3-d}{-1} may be precalculated by the user as 
%\[
%  \sigma = \frac{\delta k^R}{l^R}
%\]
%where $k^R$ is the hydraulic conductivity of the river badrock and $l^R$ is its thicknes.
%Percipitation can be prescribed through $q_d^N$.
%From parctical point of view, the river can be understud as an information that the underground water
%is close to the surface. So we constrain the infiltration from the river into the domain by the percipitation
%\begin{equation}
%    \vc q^d \cdot \vc n \ge q_d^N
%\end{equation}
%which, using \eqref{bc:river_flux}, is equivalent to constrain for the pressure head
%\begin{equation}
%    h_d \ge h_d^D
%\end{equation}
%



We consider a disjoint decomposition of the boundary
\[
    \prtl\Omega_d =  \cap \Gamma_d^N \cap \Gamma_d^R
\]
into Dirichlet, Neumann, and Robin part. In order to obtain well posed steady state problem one have to 
prescribe Dirichlet or Robin boundary condition on part of boundary that is connected (geometricaly of through
the inter dimensional coupling) to hte rest of the domain.

For unsteady problems one has to specify an initial condition in terms of the 
\hyperA{DarcyFlowMH-Data::init-pressure}{initial pressure head}
$h_d^0$ \units{}{1}{}
or the \hyperA{DarcyFlowMH-Data::init-piezo-head}{initial piezometric head}
$H_d^0$ \units{}{1}{}.

\paragraph{Volume balance.}
The equation \eqref{eq:continuity} satisfies the volume balance of the liquid in the following form:
\[ V(0) + \int_0^t s(\tau) \,d\tau - \int_0^t f(\tau) \,d\tau = V(t) \]
for any instant $t$ in the computational time interval.
Here
$$ V(t) := \sum_{d=1}^3\int_{\Omega^d}(\delta S h)(t,\vc x)\,d\vc x, $$
$$ s(t) := \sum_{d=1}^3\int_{\Omega^d}F(t,\vc x)\,d\vc x, $$
$$ f(t) := \sum_{d=1}^3\int_{\partial\Omega^d}\vc q(t,\vc x)\cdot\vc n(\vc x) \,d\vc x $$
is the volume \units{}{3}{}, the volume source \units{}{3}{-1} and the volume flux \units{}{3}{-1} of the liquid at time $t$, respectively.
The volume, flux and source on every geometrical region is calculated at each computational time step and the values together with the control sums are written to the file \texttt{water\_balance.txt}.




