% Copyright (C) 2007 Technical University of Liberec.  All rights reserved.
%
% Please make a following refer to Flow123d on your project site if you use the program for any purpose,
% especially for academic research:
% Flow123d, Research Centre: Advanced Remedial Technologies, Technical University of Liberec, Czech Republic
%
% This program is free software; you can redistribute it and/or modify it under the terms
% of the GNU General Public License version 3 as published by the Free Software Foundation.
%
% This program is distributed in the hope that it will be useful, but WITHOUT ANY WARRANTY;
% without even the implied warranty of MERCHANTABILITY or FITNESS FOR A PARTICULAR PURPOSE.
% See the GNU General Public License for more details.
%
% You should have received a copy of the GNU General Public License along with this program; if not,
% write to the Free Software Foundation, Inc., 59 Temple Place - Suite 330, Boston, MA 021110-1307, USA.

\normalsize
 \section*{Flow123D ini file format}
 \begin{flushleft}
 Flow123D version: 03.10.08
 \vspace{20pt}
 
 Note: All string values have maximal length MAXBUFF - 1 (=1023).
 \vspace{20pt}
 \end{flushleft}
 
\begin{initable}{Global}
 \key{Problem\_type} & \type{int} & NULL &
 Type of solved problem. Currently supported:\break %\br 
 1 = steady saturated flow\break %\br
 3 = variable-density saturated flow
 \\
 \hline
 \key{Description} & \type{string} & {\it undefined} &
 Short description of solved problem - any text.
 \\
 \hline
 \key{Stop\_time} & \type{double} & 1.0 &
 Time interval of the whole problem.%\br
 [time units]
 \\
 \hline
 \key{Save\_step} & \type{double} & 1.0 &
 The output with transport is written every
 {\tt Save\_step}. [time units]
 \\
 \hline
 \key{Density\_step} & \type{double} & 1.0 &
 Time interval of one density iteration
 in the varible-density calculation (type=3)
 [time units]
 \\
 \hline
\end{initable}

 
%\begin{initable}{Parallel}
% \key{part\_type} & \type{int} & 0 &
%Partitioning based on (EXPERIMENTAL):
% 1 - GMSH element partitioning\br
% 2 - ParMETIS edges graph of highest dimension\br
% 3 - ParMETIS full edges graph
%\\
%\hline
%\end{initable}

\begin{initable}{Input}
 \key{File\_type} & \type{int} & -1 &
 Type of the input files. Now only the value 1 
 (GMSH-like files) is accepted.
\\
\hline
\key{Mesh} & \type{string} & NULL & 
Name of file containig definition of the mesh
for the problem.
\\
\hline
\key{Material} & \type{string} & NULL &
Name of file with hydraulical properties of
the elements.
\\
\hline
\key{Boundary} & \type{string} & NULL &
Name of file with boundary condition data.
\\
\hline
\key{Neighbouring} & \type{string} & NULL &
Name of file describing topology of the mesh.
\\
\hline
\key{Sources} & \type{string} & NULL &
Name of file with definition of fluid sources. 
This is optional file, if this key is not
defined, calculation goes on without sources.
\\
\hline
\end{initable}


%%%%%%%%%%%%%%%%%%%%%%%%%%%%%%%%%%%%%%%%%%%%%%%%%%%%%%%%%%%%%%%%%%%%%%%%5
\pagebreak

\begin{initable}{Transport}
 \key{Transport\_on} & \type{YES/NO} & NO & 
If set "YES" program compute transport too.
\\ 
\hline
\key{Sorption} & \type{YES/NO} & NO & 
If set "YES" program include sorption too.
\\
\hline
\key{Dual\_porosity} & \type{YES/NO} & NO & 
If set "YES" program include dual porosity too.
\\
\hline
\key{Reactions} & \type{YES/NO} & NO & 
If set "YES" program include reactions too.
\\
\hline
\key{Concentration} & \type{string} & NULL &
Name of file with initial concentration.
\\ 
\hline
\key{Transport\_BCD} & \type{string} & NULL &
Name of file with boundary condition for transport.
\\
\hline
\key{Transport\_out} & \type{string} & NULL &
Name of transport output file.
\\
\hline
\key{Transport\_out\_im} & \type{string} & NULL &
Name of transport immobile output file.
\\ 
\hline
\key{Transport\_out\_sorp} & \type{string} & NULL &
Name of transport sorbed output file.
\\ 
\hline
\key{Transport\_out\_im\_sorp} & \type{string} & NULL &
Name of transport sorbed immobile output file.
\\ 
\hline
\key{N\_substances} & \type{int} & -1 &
Number of substances.
\\
\hline
\key{Subst\_names} & \type{string} & {\it undefined} &
Names of the substances separated by commas.
\\ 
\hline
\key{Substances\_density\_scales} & \type{list of doubles} & 1.0 &
Scales of substances for the density flow calculation.\\
  \hline
\end{initable}
 
\begin{initable}{Constants}
\key{g} & \type{double} & 1.0 &
Gravity acceleration.
\\ 
\hline
\key{rho} & \type{double} & 1.0 &
Density of fluid.
\\
\hline
\end{initable}
 
 \normalsize
 
\begin{initable}{Run}
\key{Log\_file} & \type{string} & mixhyb.log &
Name of log file.
\\ 
\hline
\key{Screen\_verbosity} & \type{int} & 8 &
Amount of messages printed on the screen. (0 = no messages, ..., 7 = all messages)
\\
\hline
\key{Log\_verbosity} & int & 8 &
Amount of messages printed to the log file. (0 = no messages, ..., 7 = all messages)
\\
\hline
\key{Pause\_after\_run} & \type{YES/NO} & NO &
If set to "YES", the program waits for a key press before it finishes.
\\ 
\hline
\end{initable}
 
\begin{initable}{Solver}
\key{Use\_last\_solution} & \type{YES/NO} & NO &
If set to "YES", uses last known solution for chosen solver.
\\
\hline
\\
\key{Solver\_name} & \type{string} & matlab &
Command for calling external solver.\br
Supported solvers are: {\tt petsc}, {\tt isol}, and {\tt matlab}.
\\
\hline
\key{Solver\_params} & \type{string} & NULL & 
Optional parameters for the external solver passed on the command line or
PETSc options if the PETSC solver is chosen (see doc/petsc\_help). 
\\
\hline
\key{Keep\_solver\_files} & \type{YES/NO} & NO &
If set to "YES", files for solver are not deleted after the run of the solver.
\\
\hline
\key{Manual\_solver\_run} & \type{YES/NO} & NO &
If set to "YES", programm stops after writing input files for solver and lets user to run it.
\\ 
\hline
\key{Use\_control\_file} & \type{YES/NO} & NO &
If set to "YES", programm do not create control file for solver, it uses given file.
\\
\hline
\key{Control\_file} & \type{string} & NULL &
Name of control file for situation, when {\tt Use\_control\_file} \= YES.
\\
\hline
\key{NSchurs} & \type{int} & 2 &
Number of Schur complements to use. Valid values are 0,1,2. The last one should be the fastest.
\\
\hline
\end{initable}
 
\begin{initable}{Solver parameters}
\key{Solver\_accuracy} & \type{double} & 1e-6 &
When to stop solver run - value of residum of matrix. 
Useful values from 1e-4 to 1e-10.\br
Bigger number = faster run, less accuracy.
\\
\hline\\
%%
%%
%%  particular parameters for ISOL - reduce them
%%
%% method & string & fgmres & (i)\\
%%  \hline\\
%% stop\_crit & string & backerr &(i)\\
%%  \hline\\
%% be\_tol & double & 1e-10 &(i)\\
%%  \hline\\
%% stop\_check & int & 1 &(i)\\
%%  \hline\\
%% scaling & string & mc29\_30 &(i)\\
%%  \hline\\
%% precond & string & ilu &(i)\\
%%  \hline\\
%% sor\_omega & double & 1.0 &(i)\\
%%  \hline\\
%% ilu\_cpiv & int & 0 &(i)\\
%%  \hline\\
%% ilu\_droptol & double & 1e-3 &(i)\\
%%  \hline\\
%% ilu\_dskip & int & -1 &(i)\\
%%  \hline\\
%% ilu\_lfil & int & -1 &(i)\\
%%  \hline\\
%% ilu\_milu & int & 0 &(i)\\
%%  \hline\\
\end{initable}
 Note: For aditional documentation see manual of the solver, (i) - isol manual
\pagebreak
%%%%%%%%%%%%%%%%%%%%%%%%%%%%%%%%%%%%%%%%%%%%%%%%%%%%%%%%%%%%%%%%%%%%%%%%%%%%%%%%%%%%%%%%%%5 

\begin{initable}{Output}
\key{Write\_output\_file} & \type{YES/NO} & NO &
If set to "YES", writes output file.
\\
\hline
\key{Output\_file} & \type{string} & NULL &
Name of the output file (type 1).
\\
\hline
\key{Output\_file\_2} & \type{string} & NULL &
Name of the output file (type 2).
\\
\hline
\key{Output\_digits} & \type{int} & 6 &
Number of digits used for floating point numbers in output file.
\\
\hline
\key{Output\_file\_type} & \type{int} & 1 &
Type of output file\br
 1 - GMSH like format\br
 2 - Flow data file\br
 3 - both files (two separate names)
\\
\hline
\key{POS\_set\_view} & \type{YES/NO} & NO &
Write a header setting the view in GMSH to POS.
\\
\hline
\key{POS\_view\_params} & \type{double[8]}& 
        0 0 0\br
        1 1 1\br
        0 0 &
 [x y z] angle of rotation "RotationX"\br
 [x y z] scaling "ScaleX"\br
 [x y] screen position shift "TranslationX"
\\
\hline
\key{Write\_ftrans\_out} & \type{YES/NO} & NO &
If set to "YES", writes output file for ftrans.
\\
\hline
\key{Cross\_section} & \type{YES/NO} & NO &
If set to "YES", uses cross section output.
\\
\hline
\key{Cs\_params} & \type{double[7]} & {\it zero} &
Params for cross section,\br
[x0 y0 z0] initial point\br
[xe ye ze] end point\br
[delta] cylinder radius.
\\
\hline\\
\key{Specify\_elm\_type} & \type{YES/NO} & NO &
If set to "YES", next param. specify type of prefered elements. If set to
"NO", each element is included.
\\ 
\hline
\key{Output\_elm\_type} & \type{int} & -1 &
Spefify type of element dimension\br
1 - 1D (line), 2 - 2D (triangle), \br
3 - 3D (tetrahedron).
\\
\hline
\key{BTC\_elms} & \type{list of ints} & {\it undefined} &
List of the breakthrough curve elements, ints this concentrations are written to
seperate file with extension *.btc.
\\
\hline\\
\key{FCs\_params}& double[4] & {\it zero} &
Params of flow cross section\br
[x y z 1] plane of cut (general equation),\br
 output values are written by coordinate\br
 of axis: x - [0], y - [1], z - [2]
\\
\hline
\key{Pos\_format} & \type{string} & ASCII &
Format of the POS output file [ASCII / BIN] (opening a binary file in the GMSH is much faster).
\\
\hline\\
\end{initable}
Description: Options controling output file of the programm

 \begin{initable}{Density}
 \key{Density\_implicit} & \type{YES/NO} & NO &
 NO = explicit iteration (simple flow update)\br
 YES = implicit iteration (more accurate flow update)
\\
\hline
\key{Density\_max\_iter} & int & 20 &
Maximum number of iterations for implicit density calcultation.
\\
\hline\\
\key{Eps\_iter} & \type{double} & 1e-5&
Stopping criterium for iterations (maximum norm of pressure difference).
\\
\hline
\key{Write\_iterations} & \type{YES/NO} & NO &
Write conc values during iterations to POS file.
\\
\hline
\end{initable}

 
 