\begin{RecordType}{\HTRaised{IT::Root}{Root}}{}{}{\AddDoc{Root}}{Root record of JSON input for Flow123d.}
\KeyItem{\hyperB{Root::problem}{problem}}{abstract type: \hyperlink{IT::Problem}{Problem}}{\textless\it obligatory\textgreater}{\AddDoc{Root::problem}}{Simulation problem to be solved.}
\KeyItem{\hyperB{Root::pause-after-run}{pause\_after\_run}}{Bool}{false}{\AddDoc{Root::pause\_after\_run}}{If true, the program will wait for key press before it terminates.}
\KeyItem{\hyperB{Root::output-streams}{output\_streams}}{Array  of record: \hyperlink{IT::OutputStream}{OutputStream}}{\textless\it optional\textgreater}{\AddDoc{Root::output\_streams}}{Array of formated output streams to open.}
\end{RecordType}

\begin{AbstractType}{\HTRaised{IT::Problem}{Problem}}{}{\AddDoc{Problem}}{The root record of description of particular the problem to solve.}
\Descendant{\hyperlink{IT::SequentialCoupling}{SequentialCoupling}}
\end{AbstractType}

\begin{RecordType}{\HTRaised{IT::SequentialCoupling}{SequentialCoupling}}{\hyperlink{IT::Problem}{Problem}}{}{\AddDoc{SequentialCoupling}}{Record with data for a general sequential coupling.
}
\KeyItem{\hyperB{SequentialCoupling::TYPE}{TYPE}}{selection: Problem\_TYPE\_selection}{SequentialCoupling}{\AddDoc{SequentialCoupling::TYPE}}{Sub-record selection.}
\KeyItem{\hyperB{SequentialCoupling::description}{description}}{String (generic)}{\textless\it optional\textgreater}{\AddDoc{SequentialCoupling::description}}{Short description of the solved problem.
Is displayed in the main log, and possibly in other text output files.}
\KeyItem{\hyperB{SequentialCoupling::mesh}{mesh}}{record: \hyperlink{IT::Mesh}{Mesh}}{\textless\it obligatory\textgreater}{\AddDoc{SequentialCoupling::mesh}}{Computational mesh common to all equations.}
\KeyItem{\hyperB{SequentialCoupling::time}{time}}{record: \hyperlink{IT::TimeGovernor}{TimeGovernor}}{\textless\it optional\textgreater}{\AddDoc{SequentialCoupling::time}}{Simulation time frame and time step.}
\KeyItem{\hyperB{SequentialCoupling::primary-equation}{primary\_equation}}{abstract type: \hyperlink{IT::DarcyFlowMH}{DarcyFlowMH}}{\textless\it obligatory\textgreater}{\AddDoc{SequentialCoupling::primary\_equation}}{Primary equation, have all data given.}
\KeyItem{\hyperB{SequentialCoupling::secondary-equation}{secondary\_equation}}{abstract type: \hyperlink{IT::Transport}{Transport}}{\textless\it optional\textgreater}{\AddDoc{SequentialCoupling::secondary\_equation}}{The equation that depends (the velocity field) on the result of the primary equation.}
\end{RecordType}

\begin{RecordType}{\HTRaised{IT::Mesh}{Mesh}}{}{}{\AddDoc{Mesh}}{Record with mesh related data.}
\KeyItem{\hyperB{Mesh::mesh-file}{mesh\_file}}{input file name}{\textless\it obligatory\textgreater}{\AddDoc{Mesh::mesh\_file}}{Input file with mesh description.}
\KeyItem{\hyperB{Mesh::regions}{regions}}{Array  of record: \hyperlink{IT::Region}{Region}}{\textless\it optional\textgreater}{\AddDoc{Mesh::regions}}{List of additional region definitions not contained in the mesh.}
\KeyItem{\hyperB{Mesh::sets}{sets}}{Array  of record: \hyperlink{IT::RegionSet}{RegionSet}}{\textless\it optional\textgreater}{\AddDoc{Mesh::sets}}{List of region set definitions. There are three region sets implicitly defined:
ALL (all regions of the mesh), BOUNDARY (all boundary regions), and BULK (all bulk regions)}
\KeyItem{\hyperB{Mesh::partitioning}{partitioning}}{record: \hyperlink{IT::Partition}{Partition}}{any\_neighboring}{\AddDoc{Mesh::partitioning}}{Parameters of mesh partitioning algorithms.
}
\end{RecordType}

\begin{RecordType}{\HTRaised{IT::Region}{Region}}{}{}{\AddDoc{Region}}{Definition of region of elements.}
\KeyItem{\hyperB{Region::name}{name}}{String (generic)}{\textless\it obligatory\textgreater}{\AddDoc{Region::name}}{Label (name) of the region. Has to be unique in one mesh.
}
\KeyItem{\hyperB{Region::id}{id}}{Integer [0, ]}{\textless\it obligatory\textgreater}{\AddDoc{Region::id}}{The ID of the region to which you assign label.}
\KeyItem{\hyperB{Region::element-list}{element\_list}}{Array  of Integer [0, ]}{\textless\it optional\textgreater}{\AddDoc{Region::element\_list}}{Specification of the region by the list of elements. This is not recomended}
\end{RecordType}

\begin{RecordType}{\HTRaised{IT::RegionSet}{RegionSet}}{}{}{\AddDoc{RegionSet}}{Definition of one region set.}
\KeyItem{\hyperB{RegionSet::name}{name}}{String (generic)}{\textless\it obligatory\textgreater}{\AddDoc{RegionSet::name}}{Unique name of the region set.}
\KeyItem{\hyperB{RegionSet::region-ids}{region\_ids}}{Array  of Integer [0, ]}{\textless\it optional\textgreater}{\AddDoc{RegionSet::region\_ids}}{List of region ID numbers that has to be added to the region set.}
\KeyItem{\hyperB{RegionSet::region-labels}{region\_labels}}{Array  of String (generic)}{\textless\it optional\textgreater}{\AddDoc{RegionSet::region\_labels}}{List of labels of the regions that has to be added to the region set.}
\KeyItem{\hyperB{RegionSet::union}{union}}{Array [2, 2] of String (generic)}{\textless\it optional\textgreater}{\AddDoc{RegionSet::union}}{Defines region set as a union of given pair of sets. Overrides previous keys.}
\KeyItem{\hyperB{RegionSet::intersection}{intersection}}{Array [2, 2] of String (generic)}{\textless\it optional\textgreater}{\AddDoc{RegionSet::intersection}}{Defines region set as an intersection of given pair of sets. Overrides previous keys.}
\KeyItem{\hyperB{RegionSet::difference}{difference}}{Array [2, 2] of String (generic)}{\textless\it optional\textgreater}{\AddDoc{RegionSet::difference}}{Defines region set as a difference of given pair of sets. Overrides previous keys.}
\end{RecordType}

\begin{RecordType}{\HTRaised{IT::Partition}{Partition}}{}{\hyperlink{Partition::graph-type}{graph\_type}}{\AddDoc{Partition}}{Setting for various types of mesh partitioning.}
\KeyItem{\hyperB{Partition::tool}{tool}}{selection: \hyperlink{IT::PartTool}{PartTool}}{METIS}{\AddDoc{Partition::tool}}{Software package used for partitioning. See corresponding selection.}
\KeyItem{\hyperB{Partition::graph-type}{graph\_type}}{selection: \hyperlink{IT::GraphType}{GraphType}}{any\_neighboring}{\AddDoc{Partition::graph\_type}}{Algorithm for generating graph and its weights from a multidimensional mesh.}
\end{RecordType}

\begin{SelectionType}{\HTRaised{IT::PartTool}{PartTool}}{Select the partitioning tool to use.}
\KeyItem{PETSc}{Use PETSc interface to various partitioning tools.}
\KeyItem{METIS}{Use direct interface to Metis.}
\end{SelectionType}

\begin{SelectionType}{\HTRaised{IT::GraphType}{GraphType}}{Different algorithms to make the sparse graph with weighted edges
from the multidimensional mesh. Main difference is dealing with 
neighborings of elements of different dimension.}
\KeyItem{any\_neighboring}{Add edge for any pair of neighboring elements.}
\KeyItem{any\_wight\_lower\_dim\_cuts}{Same as before and assign higher weight to cuts of lower dimension in order to make them stick to one face.}
\KeyItem{same\_dimension\_neghboring}{Add edge for any pair of neighboring elements of same dimension (bad for matrix multiply).}
\end{SelectionType}

\begin{RecordType}{\HTRaised{IT::TimeGovernor}{TimeGovernor}}{}{}{\AddDoc{TimeGovernor}}{Setting of the simulation time. (can be specific to one eqaution)}
\KeyItem{\hyperB{TimeGovernor::start-time}{start\_time}}{Double }{0.0}{\AddDoc{TimeGovernor::start\_time}}{Start time of the simulation.}
\KeyItem{\hyperB{TimeGovernor::end-time}{end\_time}}{Double }{\textless\it obligatory\textgreater}{\AddDoc{TimeGovernor::end\_time}}{End time of the simulation.}
\KeyItem{\hyperB{TimeGovernor::init-dt}{init\_dt}}{Double [0, ]}{\textless\it optional\textgreater}{\AddDoc{TimeGovernor::init\_dt}}{Initial guess for the time step. The time step is fixed if hard time step limits are not set.}
\KeyItem{\hyperB{TimeGovernor::min-dt}{min\_dt}}{Double [0, ]}{"Machine precision or 'init\_dt' if specified"}{\AddDoc{TimeGovernor::min\_dt}}{Hard lower limit for the time step.}
\KeyItem{\hyperB{TimeGovernor::max-dt}{max\_dt}}{Double [0, ]}{"Whole time of the simulation or 'init\_dt' if specified"}{\AddDoc{TimeGovernor::max\_dt}}{Hard upper limit for the time step.}
\end{RecordType}

\begin{AbstractType}{\HTRaised{IT::DarcyFlowMH}{DarcyFlowMH}}{}{\AddDoc{DarcyFlowMH}}{Mixed-Hybrid  solver for saturated Darcy flow.}
\Descendant{\hyperlink{IT::Steady-MH}{Steady\_MH}}
\Descendant{\hyperlink{IT::Unsteady-MH}{Unsteady\_MH}}
\Descendant{\hyperlink{IT::Unsteady-LMH}{Unsteady\_LMH}}
\end{AbstractType}

\begin{RecordType}{\HTRaised{IT::Steady-MH}{Steady\_MH}}{\hyperlink{IT::DarcyFlowMH}{DarcyFlowMH}}{}{\AddDoc{Steady\_MH}}{Mixed-Hybrid  solver for STEADY saturated Darcy flow.}
\KeyItem{\hyperB{Steady-MH::TYPE}{TYPE}}{selection: DarcyFlowMH\_TYPE\_selection}{Steady\_MH}{\AddDoc{Steady\_MH::TYPE}}{Sub-record selection.}
\KeyItem{\hyperB{Steady-MH::n-schurs}{n\_schurs}}{Integer [0, 2]}{2}{\AddDoc{Steady\_MH::n\_schurs}}{Number of Schur complements to perform when solving MH sytem.}
\KeyItem{\hyperB{Steady-MH::solver}{solver}}{abstract type: \hyperlink{IT::Solver}{Solver}}{\textless\it obligatory\textgreater}{\AddDoc{Steady\_MH::solver}}{Linear solver for MH problem.}
\KeyItem{\hyperB{Steady-MH::output}{output}}{record: \hyperlink{IT::DarcyMHOutput}{DarcyMHOutput}}{\textless\it obligatory\textgreater}{\AddDoc{Steady\_MH::output}}{Parameters of output form MH module.}
\KeyItem{\hyperB{Steady-MH::mortar-method}{mortar\_method}}{selection: \hyperlink{IT::MH-MortarMethod}{MH\_MortarMethod}}{None}{\AddDoc{Steady\_MH::mortar\_method}}{Method for coupling Darcy flow between dimensions.}
\KeyItem{\hyperB{Steady-MH::bc-data}{bc\_data}}{Array  of record: \hyperlink{IT::DarcyFlowMH-Steady-BoundaryData}{DarcyFlowMH\_Steady\_BoundaryData}}{\textless\it obligatory\textgreater}{\AddDoc{Steady\_MH::bc\_data}}{}
\KeyItem{\hyperB{Steady-MH::bulk-data}{bulk\_data}}{Array  of record: \hyperlink{IT::DarcyFlowMH-Steady-BulkData}{DarcyFlowMH\_Steady\_BulkData}}{\textless\it obligatory\textgreater}{\AddDoc{Steady\_MH::bulk\_data}}{}
\end{RecordType}

\begin{AbstractType}{\HTRaised{IT::Solver}{Solver}}{}{\AddDoc{Solver}}{Solver setting.}
\Descendant{\hyperlink{IT::Petsc}{Petsc}}
\Descendant{\hyperlink{IT::Bddc}{Bddc}}
\end{AbstractType}

\begin{RecordType}{\HTRaised{IT::Petsc}{Petsc}}{\hyperlink{IT::Solver}{Solver}}{}{\AddDoc{Petsc}}{Solver setting.}
\KeyItem{\hyperB{Petsc::TYPE}{TYPE}}{selection: Solver\_TYPE\_selection}{Petsc}{\AddDoc{Petsc::TYPE}}{Sub-record selection.}
\KeyItem{\hyperB{Petsc::a-tol}{a\_tol}}{Double [0, ]}{1.0e-9}{\AddDoc{Petsc::a\_tol}}{Absolute residual tolerance.}
\KeyItem{\hyperB{Petsc::r-tol}{r\_tol}}{Double [0, 1]}{1.0e-7}{\AddDoc{Petsc::r\_tol}}{Relative residual tolerance (to initial error).}
\KeyItem{\hyperB{Petsc::max-it}{max\_it}}{Integer [0, ]}{10000}{\AddDoc{Petsc::max\_it}}{Maximum number of outer iterations of the linear solver.}
\KeyItem{\hyperB{Petsc::options}{options}}{String (generic)}{}{\AddDoc{Petsc::options}}{Options passed to the petsc instead of default setting.}
\end{RecordType}

\begin{RecordType}{\HTRaised{IT::Bddc}{Bddc}}{\hyperlink{IT::Solver}{Solver}}{}{\AddDoc{Bddc}}{Solver setting.}
\KeyItem{\hyperB{Bddc::TYPE}{TYPE}}{selection: Solver\_TYPE\_selection}{Bddc}{\AddDoc{Bddc::TYPE}}{Sub-record selection.}
\KeyItem{\hyperB{Bddc::a-tol}{a\_tol}}{Double [0, ]}{1.0e-9}{\AddDoc{Bddc::a\_tol}}{Absolute residual tolerance.}
\KeyItem{\hyperB{Bddc::r-tol}{r\_tol}}{Double [0, 1]}{1.0e-7}{\AddDoc{Bddc::r\_tol}}{Relative residual tolerance (to initial error).}
\KeyItem{\hyperB{Bddc::max-it}{max\_it}}{Integer [0, ]}{10000}{\AddDoc{Bddc::max\_it}}{Maximum number of outer iterations of the linear solver.}
\end{RecordType}

\begin{RecordType}{\HTRaised{IT::DarcyMHOutput}{DarcyMHOutput}}{}{}{\AddDoc{DarcyMHOutput}}{Parameters of MH output.}
\KeyItem{\hyperB{DarcyMHOutput::save-step}{save\_step}}{Double [0, ]}{1.0}{\AddDoc{DarcyMHOutput::save\_step}}{Regular step between MH outputs.}
\KeyItem{\hyperB{DarcyMHOutput::output-stream}{output\_stream}}{record: \hyperlink{IT::OutputStream}{OutputStream}}{\textless\it obligatory\textgreater}{\AddDoc{DarcyMHOutput::output\_stream}}{Parameters of output stream.}
\KeyItem{\hyperB{DarcyMHOutput::velocity-p0}{velocity\_p0}}{String (generic)}{\textless\it optional\textgreater}{\AddDoc{DarcyMHOutput::velocity\_p0}}{Output stream for P0 approximation of the velocity field.}
\KeyItem{\hyperB{DarcyMHOutput::pressure-p0}{pressure\_p0}}{String (generic)}{\textless\it optional\textgreater}{\AddDoc{DarcyMHOutput::pressure\_p0}}{Output stream for P0 approximation of the pressure field.}
\KeyItem{\hyperB{DarcyMHOutput::pressure-p1}{pressure\_p1}}{String (generic)}{\textless\it optional\textgreater}{\AddDoc{DarcyMHOutput::pressure\_p1}}{Output stream for P1 approximation of the pressure field.}
\KeyItem{\hyperB{DarcyMHOutput::piezo-head-p0}{piezo\_head\_p0}}{String (generic)}{\textless\it optional\textgreater}{\AddDoc{DarcyMHOutput::piezo\_head\_p0}}{Output stream for P0 approximation of the piezometric head field.}
\KeyItem{\hyperB{DarcyMHOutput::balance-output}{balance\_output}}{output file name}{water\_balance.txt}{\AddDoc{DarcyMHOutput::balance\_output}}{Output file for water balance table.}
\KeyItem{\hyperB{DarcyMHOutput::subdomains}{subdomains}}{String (generic)}{\textless\it optional\textgreater}{\AddDoc{DarcyMHOutput::subdomains}}{Output stream for subdomain indices (partitioning of mesh elements) used by DarcyFlow module.}
\KeyItem{\hyperB{DarcyMHOutput::raw-flow-output}{raw\_flow\_output}}{output file name}{\textless\it optional\textgreater}{\AddDoc{DarcyMHOutput::raw\_flow\_output}}{Output file with raw data form MH module.}
\end{RecordType}

\begin{RecordType}{\HTRaised{IT::OutputStream}{OutputStream}}{}{}{\AddDoc{OutputStream}}{Parameters of output.}
\KeyItem{\hyperB{OutputStream::name}{name}}{String (generic)}{\textless\it obligatory\textgreater}{\AddDoc{OutputStream::name}}{The name of this stream. Used to reference the output stream.}
\KeyItem{\hyperB{OutputStream::file}{file}}{output file name}{\textless\it obligatory\textgreater}{\AddDoc{OutputStream::file}}{File path to the connected output file.}
\KeyItem{\hyperB{OutputStream::format}{format}}{abstract type: \hyperlink{IT::OutputFormat}{OutputFormat}}{\textless\it optional\textgreater}{\AddDoc{OutputStream::format}}{Format of output stream and possible parameters.}
\end{RecordType}

\begin{AbstractType}{\HTRaised{IT::OutputFormat}{OutputFormat}}{}{\AddDoc{OutputFormat}}{Format of output stream and possible parameters.}
\Descendant{\hyperlink{IT::vtk}{vtk}}
\Descendant{\hyperlink{IT::gmsh}{gmsh}}
\end{AbstractType}

\begin{RecordType}{\HTRaised{IT::vtk}{vtk}}{\hyperlink{IT::OutputFormat}{OutputFormat}}{}{\AddDoc{vtk}}{Parameters of vtk output format.}
\KeyItem{\hyperB{vtk::TYPE}{TYPE}}{selection: OutputFormat\_TYPE\_selection}{vtk}{\AddDoc{vtk::TYPE}}{Sub-record selection.}
\KeyItem{\hyperB{vtk::variant}{variant}}{selection: \hyperlink{IT::VTK variant (ascii or binary)}{VTK variant (ascii or binary)}}{ascii}{\AddDoc{vtk::variant}}{Variant of output stream file format.}
\KeyItem{\hyperB{vtk::parallel}{parallel}}{Bool}{false}{\AddDoc{vtk::parallel}}{Parallel or serial version of file format.}
\KeyItem{\hyperB{vtk::compression}{compression}}{selection: \hyperlink{IT::Type of compression of VTK file format}{Type of compression of VTK file format}}{none}{\AddDoc{vtk::compression}}{Compression used in output stream file format.}
\end{RecordType}

\begin{SelectionType}{\HTRaised{IT::VTK variant (ascii or binary)}{VTK variant (ascii or binary)}}{}
\KeyItem{ascii}{ASCII variant of VTK file format}
\KeyItem{binary}{Binary variant of VTK file format (not supported yet)}
\end{SelectionType}

\begin{SelectionType}{\HTRaised{IT::Type of compression of VTK file format}{Type of compression of VTK file format}}{}
\KeyItem{none}{Data in VTK file format are not compressed}
\KeyItem{zlib}{Data in VTK file format are compressed using zlib (not supported yet)}
\end{SelectionType}

\begin{RecordType}{\HTRaised{IT::gmsh}{gmsh}}{\hyperlink{IT::OutputFormat}{OutputFormat}}{}{\AddDoc{gmsh}}{Parameters of gmsh output format.}
\KeyItem{\hyperB{gmsh::TYPE}{TYPE}}{selection: OutputFormat\_TYPE\_selection}{gmsh}{\AddDoc{gmsh::TYPE}}{Sub-record selection.}
\end{RecordType}

\begin{SelectionType}{\HTRaised{IT::MH-MortarMethod}{MH\_MortarMethod}}{}
\KeyItem{None}{Mortar space: P0 on elements of lower dimension.}
\KeyItem{P0}{Mortar space: P0 on elements of lower dimension.}
\KeyItem{P1}{Mortar space: P1 on intersections, using non-conforming pressures.}
\end{SelectionType}

\begin{RecordType}{\HTRaised{IT::DarcyFlowMH-Steady-BoundaryData}{DarcyFlowMH\_Steady\_BoundaryData}}{}{}{\AddDoc{DarcyFlowMH\_Steady\_BoundaryData}}{Record to set BOUNDARY fields of the equation 'DarcyFlowMH\_Steady'.
The fields are set only on the domain specified by one of the keys: 'region', 'rid', 'r\_set'
and after the time given by the key 'time'. The field setting can be overridden by
 any DarcyFlowMH\_Steady\_BoundaryData record that comes later in the boundary data array.}
\KeyItem{\hyperB{DarcyFlowMH-Steady-BoundaryData::r-set}{r\_set}}{String (generic)}{\textless\it optional\textgreater}{\AddDoc{DarcyFlowMH\_Steady\_BoundaryData::r\_set}}{Name of region set where to set fields.}
\KeyItem{\hyperB{DarcyFlowMH-Steady-BoundaryData::region}{region}}{String (generic)}{\textless\it optional\textgreater}{\AddDoc{DarcyFlowMH\_Steady\_BoundaryData::region}}{Label of the region where to set fields. }
\KeyItem{\hyperB{DarcyFlowMH-Steady-BoundaryData::rid}{rid}}{Integer [0, ]}{\textless\it optional\textgreater}{\AddDoc{DarcyFlowMH\_Steady\_BoundaryData::rid}}{ID of the region where to set fields.}
\KeyItem{\hyperB{DarcyFlowMH-Steady-BoundaryData::time}{time}}{Double [0, ]}{0.0}{\AddDoc{DarcyFlowMH\_Steady\_BoundaryData::time}}{Apply field setting in this record after this time.
These times have to form an increasing sequence.}
\KeyItem{\hyperB{DarcyFlowMH-Steady-BoundaryData::bc-type}{bc\_type}}{abstract type: \hyperlink{IT::Field:R3 - Enum}{Field:R3 -> Enum}}{\textless\it optional\textgreater}{\AddDoc{DarcyFlowMH\_Steady\_BoundaryData::bc\_type}}{Boundary condition type, possible values:}
\KeyItem{\hyperB{DarcyFlowMH-Steady-BoundaryData::bc-pressure}{bc\_pressure}}{abstract type: \hyperlink{IT::Field:R3 - Real}{Field:R3 -> Real}}{\textless\it optional\textgreater}{\AddDoc{DarcyFlowMH\_Steady\_BoundaryData::bc\_pressure}}{Dirichlet BC condition value for pressure.}
\KeyItem{\hyperB{DarcyFlowMH-Steady-BoundaryData::bc-flux}{bc\_flux}}{abstract type: \hyperlink{IT::Field:R3 - Real}{Field:R3 -> Real}}{\textless\it optional\textgreater}{\AddDoc{DarcyFlowMH\_Steady\_BoundaryData::bc\_flux}}{Flux in Neumman or Robin boundary condition.}
\KeyItem{\hyperB{DarcyFlowMH-Steady-BoundaryData::bc-robin-sigma}{bc\_robin\_sigma}}{abstract type: \hyperlink{IT::Field:R3 - Real}{Field:R3 -> Real}}{\textless\it optional\textgreater}{\AddDoc{DarcyFlowMH\_Steady\_BoundaryData::bc\_robin\_sigma}}{Conductivity coefficient in Robin boundary condition.}
\KeyItem{\hyperB{DarcyFlowMH-Steady-BoundaryData::bc-piezo-head}{bc\_piezo\_head}}{abstract type: \hyperlink{IT::Field:R3 - Real}{Field:R3 -> Real}}{\textless\it optional\textgreater}{\AddDoc{DarcyFlowMH\_Steady\_BoundaryData::bc\_piezo\_head}}{Boundary condition for pressure as piezometric head.}
\KeyItem{\hyperB{DarcyFlowMH-Steady-BoundaryData::flow-old-bcd-file}{flow\_old\_bcd\_file}}{input file name}{\textless\it optional\textgreater}{\AddDoc{DarcyFlowMH\_Steady\_BoundaryData::flow\_old\_bcd\_file}}{}
\end{RecordType}

\begin{AbstractType}{\HTRaised{IT::Field:R3 - Enum}{Field:R3 -> Enum}}{\hyperlink{IT::FieldConstant}{FieldConstant}}{\AddDoc{Field:R3 -> Enum}}{Abstract record for all time-space functions.}
\Descendant{\hyperlink{IT::FieldConstant}{FieldConstant}}
\Descendant{\hyperlink{IT::FieldFormula}{FieldFormula}}
\Descendant{\hyperlink{IT::FieldPython}{FieldPython}}
\Descendant{\hyperlink{IT::FieldElementwise}{FieldElementwise}}
\end{AbstractType}

\begin{RecordType}{\HTRaised{IT::FieldConstant}{FieldConstant}}{\hyperlink{IT::Field:R3 - Enum}{Field:R3 -> Enum}}{\hyperlink{FieldConstant::value}{value}}{\AddDoc{FieldConstant}}{R3 -> Enum Field constant in space.}
\KeyItem{\hyperB{FieldConstant::TYPE}{TYPE}}{selection: Field:R3 -> Enum\_TYPE\_selection}{FieldConstant}{\AddDoc{FieldConstant::TYPE}}{Sub-record selection.}
\KeyItem{\hyperB{FieldConstant::value}{value}}{selection: \hyperlink{IT::EqData-bc-Type}{EqData\_bc\_Type}}{\textless\it obligatory\textgreater}{\AddDoc{FieldConstant::value}}{Value of the constant field.
For vector values, you can use scalar value to enter constant vector.
For square NxN-matrix values, you can use:
* vector of size N to enter diagonal matrix
* vector of size (N+1)*N/2 to enter symmetric matrix (upper triangle, row by row)
* scalar to enter multiple of the unit matrix.}
\end{RecordType}

\begin{SelectionType}{\HTRaised{IT::EqData-bc-Type}{EqData\_bc\_Type}}{}
\KeyItem{none}{Homogeneous Neoumann BC.}
\KeyItem{dirichlet}{}
\KeyItem{neumann}{}
\KeyItem{robin}{}
\KeyItem{total\_flux}{}
\end{SelectionType}

\begin{RecordType}{\HTRaised{IT::FieldFormula}{FieldFormula}}{\hyperlink{IT::Field:R3 - Enum}{Field:R3 -> Enum}}{}{\AddDoc{FieldFormula}}{R3 -> Enum Field given by runtime interpreted formula.}
\KeyItem{\hyperB{FieldFormula::TYPE}{TYPE}}{selection: Field:R3 -> Enum\_TYPE\_selection}{FieldFormula}{\AddDoc{FieldFormula::TYPE}}{Sub-record selection.}
\KeyItem{\hyperB{FieldFormula::value}{value}}{String (generic)}{\textless\it obligatory\textgreater}{\AddDoc{FieldFormula::value}}{String, array of strings, or matrix of strings with formulas for individual entries of scalar, vector, or tensor value respectively.
For vector values, you can use just one string to enter homogeneous vector.
For square NxN-matrix values, you can use:
* array of strings of size N to enter diagonal matrix
* array of strings of size (N+1)*N/2 to enter symmetric matrix (upper triangle, row by row)
* just one string to enter (spatially variable) multiple of the unit matrix.
Formula can contain variables x,y,z,t and usual operators and functions.}
\end{RecordType}

\begin{RecordType}{\HTRaised{IT::FieldPython}{FieldPython}}{\hyperlink{IT::Field:R3 - Enum}{Field:R3 -> Enum}}{}{\AddDoc{FieldPython}}{R3 -> Enum Field given by a Python script.}
\KeyItem{\hyperB{FieldPython::TYPE}{TYPE}}{selection: Field:R3 -> Enum\_TYPE\_selection}{FieldPython}{\AddDoc{FieldPython::TYPE}}{Sub-record selection.}
\KeyItem{\hyperB{FieldPython::script-string}{script\_string}}{String (generic)}{"Obligatory if 'script\_file' is not given."}{\AddDoc{FieldPython::script\_string}}{Python script given as in place string}
\KeyItem{\hyperB{FieldPython::script-file}{script\_file}}{input file name}{"Obligatory if 'script\_striong' is not given."}{\AddDoc{FieldPython::script\_file}}{Python script given as external file}
\KeyItem{\hyperB{FieldPython::function}{function}}{String (generic)}{\textless\it obligatory\textgreater}{\AddDoc{FieldPython::function}}{Function in the given script that returns tuple containing components of the return type.
For NxM tensor values: tensor(row,col) = tuple( M*row + col ).}
\end{RecordType}

\begin{RecordType}{\HTRaised{IT::FieldElementwise}{FieldElementwise}}{\hyperlink{IT::Field:R3 - Enum}{Field:R3 -> Enum}}{}{\AddDoc{FieldElementwise}}{R3 -> Enum Field constant in space.}
\KeyItem{\hyperB{FieldElementwise::TYPE}{TYPE}}{selection: Field:R3 -> Enum\_TYPE\_selection}{FieldElementwise}{\AddDoc{FieldElementwise::TYPE}}{Sub-record selection.}
\KeyItem{\hyperB{FieldElementwise::gmsh-file}{gmsh\_file}}{input file name}{\textless\it obligatory\textgreater}{\AddDoc{FieldElementwise::gmsh\_file}}{Input file with ASCII GMSH file format.}
\KeyItem{\hyperB{FieldElementwise::field-name}{field\_name}}{String (generic)}{\textless\it obligatory\textgreater}{\AddDoc{FieldElementwise::field\_name}}{The values of the Field are read from the \$ElementData section with field name given by this key.}
\end{RecordType}

\begin{AbstractType}{\HTRaised{IT::Field:R3 - Real}{Field:R3 -> Real}}{\hyperlink{IT::FieldConstant}{FieldConstant}}{\AddDoc{Field:R3 -> Real}}{Abstract record for all time-space functions.}
\Descendant{\hyperlink{IT::FieldConstant}{FieldConstant}}
\Descendant{\hyperlink{IT::FieldPython}{FieldPython}}
\Descendant{\hyperlink{IT::FieldFormula}{FieldFormula}}
\Descendant{\hyperlink{IT::FieldElementwise}{FieldElementwise}}
\Descendant{\hyperlink{IT::FieldInterpolatedP0}{FieldInterpolatedP0}}
\end{AbstractType}

\begin{RecordType}{\HTRaised{IT::FieldConstant}{FieldConstant}}{\hyperlink{IT::Field:R3 - Real}{Field:R3 -> Real}}{\hyperlink{FieldConstant::value}{value}}{\AddDoc{FieldConstant}}{R3 -> Real Field constant in space.}
\KeyItem{\hyperB{FieldConstant::TYPE}{TYPE}}{selection: Field:R3 -> Real\_TYPE\_selection}{FieldConstant}{\AddDoc{FieldConstant::TYPE}}{Sub-record selection.}
\KeyItem{\hyperB{FieldConstant::value}{value}}{Double }{\textless\it obligatory\textgreater}{\AddDoc{FieldConstant::value}}{Value of the constant field.
For vector values, you can use scalar value to enter constant vector.
For square NxN-matrix values, you can use:
* vector of size N to enter diagonal matrix
* vector of size (N+1)*N/2 to enter symmetric matrix (upper triangle, row by row)
* scalar to enter multiple of the unit matrix.}
\end{RecordType}

\begin{RecordType}{\HTRaised{IT::FieldPython}{FieldPython}}{\hyperlink{IT::Field:R3 - Real}{Field:R3 -> Real}}{}{\AddDoc{FieldPython}}{R3 -> Real Field given by a Python script.}
\KeyItem{\hyperB{FieldPython::TYPE}{TYPE}}{selection: Field:R3 -> Real\_TYPE\_selection}{FieldPython}{\AddDoc{FieldPython::TYPE}}{Sub-record selection.}
\KeyItem{\hyperB{FieldPython::script-string}{script\_string}}{String (generic)}{"Obligatory if 'script\_file' is not given."}{\AddDoc{FieldPython::script\_string}}{Python script given as in place string}
\KeyItem{\hyperB{FieldPython::script-file}{script\_file}}{input file name}{"Obligatory if 'script\_striong' is not given."}{\AddDoc{FieldPython::script\_file}}{Python script given as external file}
\KeyItem{\hyperB{FieldPython::function}{function}}{String (generic)}{\textless\it obligatory\textgreater}{\AddDoc{FieldPython::function}}{Function in the given script that returns tuple containing components of the return type.
For NxM tensor values: tensor(row,col) = tuple( M*row + col ).}
\end{RecordType}

\begin{RecordType}{\HTRaised{IT::FieldFormula}{FieldFormula}}{\hyperlink{IT::Field:R3 - Real}{Field:R3 -> Real}}{}{\AddDoc{FieldFormula}}{R3 -> Real Field given by runtime interpreted formula.}
\KeyItem{\hyperB{FieldFormula::TYPE}{TYPE}}{selection: Field:R3 -> Real\_TYPE\_selection}{FieldFormula}{\AddDoc{FieldFormula::TYPE}}{Sub-record selection.}
\KeyItem{\hyperB{FieldFormula::value}{value}}{String (generic)}{\textless\it obligatory\textgreater}{\AddDoc{FieldFormula::value}}{String, array of strings, or matrix of strings with formulas for individual entries of scalar, vector, or tensor value respectively.
For vector values, you can use just one string to enter homogeneous vector.
For square NxN-matrix values, you can use:
* array of strings of size N to enter diagonal matrix
* array of strings of size (N+1)*N/2 to enter symmetric matrix (upper triangle, row by row)
* just one string to enter (spatially variable) multiple of the unit matrix.
Formula can contain variables x,y,z,t and usual operators and functions.}
\end{RecordType}

\begin{RecordType}{\HTRaised{IT::FieldElementwise}{FieldElementwise}}{\hyperlink{IT::Field:R3 - Real}{Field:R3 -> Real}}{}{\AddDoc{FieldElementwise}}{R3 -> Real Field constant in space.}
\KeyItem{\hyperB{FieldElementwise::TYPE}{TYPE}}{selection: Field:R3 -> Real\_TYPE\_selection}{FieldElementwise}{\AddDoc{FieldElementwise::TYPE}}{Sub-record selection.}
\KeyItem{\hyperB{FieldElementwise::gmsh-file}{gmsh\_file}}{input file name}{\textless\it obligatory\textgreater}{\AddDoc{FieldElementwise::gmsh\_file}}{Input file with ASCII GMSH file format.}
\KeyItem{\hyperB{FieldElementwise::field-name}{field\_name}}{String (generic)}{\textless\it obligatory\textgreater}{\AddDoc{FieldElementwise::field\_name}}{The values of the Field are read from the \$ElementData section with field name given by this key.}
\end{RecordType}

\begin{RecordType}{\HTRaised{IT::FieldInterpolatedP0}{FieldInterpolatedP0}}{\hyperlink{IT::Field:R3 - Real}{Field:R3 -> Real}}{}{\AddDoc{FieldInterpolatedP0}}{R3 -> Real Field constant in space.}
\KeyItem{\hyperB{FieldInterpolatedP0::TYPE}{TYPE}}{selection: Field:R3 -> Real\_TYPE\_selection}{FieldInterpolatedP0}{\AddDoc{FieldInterpolatedP0::TYPE}}{Sub-record selection.}
\KeyItem{\hyperB{FieldInterpolatedP0::gmsh-file}{gmsh\_file}}{input file name}{\textless\it obligatory\textgreater}{\AddDoc{FieldInterpolatedP0::gmsh\_file}}{Input file with ASCII GMSH file format.}
\KeyItem{\hyperB{FieldInterpolatedP0::field-name}{field\_name}}{String (generic)}{\textless\it obligatory\textgreater}{\AddDoc{FieldInterpolatedP0::field\_name}}{The values of the Field are read from the \$ElementData section with field name given by this key.}
\end{RecordType}

\begin{RecordType}{\HTRaised{IT::DarcyFlowMH-Steady-BulkData}{DarcyFlowMH\_Steady\_BulkData}}{}{}{\AddDoc{DarcyFlowMH\_Steady\_BulkData}}{Record to set BULK fields of the equation 'DarcyFlowMH\_Steady'.
The fields are set only on the domain specified by one of the keys: 'region', 'rid', 'r\_set'
and after the time given by the key 'time'. The field setting can be overridden by
 any DarcyFlowMH\_Steady\_BulkData record that comes later in the bulk data array.}
\KeyItem{\hyperB{DarcyFlowMH-Steady-BulkData::r-set}{r\_set}}{String (generic)}{\textless\it optional\textgreater}{\AddDoc{DarcyFlowMH\_Steady\_BulkData::r\_set}}{Name of region set where to set fields.}
\KeyItem{\hyperB{DarcyFlowMH-Steady-BulkData::region}{region}}{String (generic)}{\textless\it optional\textgreater}{\AddDoc{DarcyFlowMH\_Steady\_BulkData::region}}{Label of the region where to set fields. }
\KeyItem{\hyperB{DarcyFlowMH-Steady-BulkData::rid}{rid}}{Integer [0, ]}{\textless\it optional\textgreater}{\AddDoc{DarcyFlowMH\_Steady\_BulkData::rid}}{ID of the region where to set fields.}
\KeyItem{\hyperB{DarcyFlowMH-Steady-BulkData::time}{time}}{Double [0, ]}{0.0}{\AddDoc{DarcyFlowMH\_Steady\_BulkData::time}}{Apply field setting in this record after this time.
These times have to form an increasing sequence.}
\KeyItem{\hyperB{DarcyFlowMH-Steady-BulkData::anisotropy}{anisotropy}}{abstract type: \hyperlink{IT::Field:R3 - Real[3,3]}{Field:R3 -> Real[3,3]}}{\textless\it optional\textgreater}{\AddDoc{DarcyFlowMH\_Steady\_BulkData::anisotropy}}{Anisotropy of the conductivity tensor.}
\KeyItem{\hyperB{DarcyFlowMH-Steady-BulkData::cross-section}{cross\_section}}{abstract type: \hyperlink{IT::Field:R3 - Real}{Field:R3 -> Real}}{\textless\it optional\textgreater}{\AddDoc{DarcyFlowMH\_Steady\_BulkData::cross\_section}}{Complement dimension parameter (cross section for 1D, thickness for 2D).}
\KeyItem{\hyperB{DarcyFlowMH-Steady-BulkData::conductivity}{conductivity}}{abstract type: \hyperlink{IT::Field:R3 - Real}{Field:R3 -> Real}}{\textless\it optional\textgreater}{\AddDoc{DarcyFlowMH\_Steady\_BulkData::conductivity}}{Isotropic conductivity scalar.}
\KeyItem{\hyperB{DarcyFlowMH-Steady-BulkData::sigma}{sigma}}{abstract type: \hyperlink{IT::Field:R3 - Real}{Field:R3 -> Real}}{\textless\it optional\textgreater}{\AddDoc{DarcyFlowMH\_Steady\_BulkData::sigma}}{Transition coefficient between dimensions.}
\KeyItem{\hyperB{DarcyFlowMH-Steady-BulkData::water-source-density}{water\_source\_density}}{abstract type: \hyperlink{IT::Field:R3 - Real}{Field:R3 -> Real}}{\textless\it optional\textgreater}{\AddDoc{DarcyFlowMH\_Steady\_BulkData::water\_source\_density}}{Water source density.}
\KeyItem{\hyperB{DarcyFlowMH-Steady-BulkData::init-pressure}{init\_pressure}}{abstract type: \hyperlink{IT::Field:R3 - Real}{Field:R3 -> Real}}{\textless\it optional\textgreater}{\AddDoc{DarcyFlowMH\_Steady\_BulkData::init\_pressure}}{Initial condition as pressure}
\KeyItem{\hyperB{DarcyFlowMH-Steady-BulkData::storativity}{storativity}}{abstract type: \hyperlink{IT::Field:R3 - Real}{Field:R3 -> Real}}{\textless\it optional\textgreater}{\AddDoc{DarcyFlowMH\_Steady\_BulkData::storativity}}{Storativity.}
\KeyItem{\hyperB{DarcyFlowMH-Steady-BulkData::init-piezo-head}{init\_piezo\_head}}{abstract type: \hyperlink{IT::Field:R3 - Real}{Field:R3 -> Real}}{\textless\it optional\textgreater}{\AddDoc{DarcyFlowMH\_Steady\_BulkData::init\_piezo\_head}}{Initial condition for pressure as piezometric head.}
\end{RecordType}

\begin{AbstractType}{\HTRaised{IT::Field:R3 - Real[3,3]}{Field:R3 -> Real[3,3]}}{\hyperlink{IT::FieldConstant}{FieldConstant}}{\AddDoc{Field:R3 -> Real[3,3]}}{Abstract record for all time-space functions.}
\Descendant{\hyperlink{IT::FieldConstant}{FieldConstant}}
\Descendant{\hyperlink{IT::FieldPython}{FieldPython}}
\Descendant{\hyperlink{IT::FieldFormula}{FieldFormula}}
\Descendant{\hyperlink{IT::FieldElementwise}{FieldElementwise}}
\Descendant{\hyperlink{IT::FieldInterpolatedP0}{FieldInterpolatedP0}}
\end{AbstractType}

\begin{RecordType}{\HTRaised{IT::FieldConstant}{FieldConstant}}{\hyperlink{IT::Field:R3 - Real[3,3]}{Field:R3 -> Real[3,3]}}{\hyperlink{FieldConstant::value}{value}}{\AddDoc{FieldConstant}}{R3 -> Real[3,3] Field constant in space.}
\KeyItem{\hyperB{FieldConstant::TYPE}{TYPE}}{selection: Field:R3 -> Real[3,3]\_TYPE\_selection}{FieldConstant}{\AddDoc{FieldConstant::TYPE}}{Sub-record selection.}
\KeyItem{\hyperB{FieldConstant::value}{value}}{Array [1, ] of Array [1, ] of Double }{\textless\it obligatory\textgreater}{\AddDoc{FieldConstant::value}}{Value of the constant field.
For vector values, you can use scalar value to enter constant vector.
For square NxN-matrix values, you can use:
* vector of size N to enter diagonal matrix
* vector of size (N+1)*N/2 to enter symmetric matrix (upper triangle, row by row)
* scalar to enter multiple of the unit matrix.}
\end{RecordType}

\begin{RecordType}{\HTRaised{IT::FieldPython}{FieldPython}}{\hyperlink{IT::Field:R3 - Real[3,3]}{Field:R3 -> Real[3,3]}}{}{\AddDoc{FieldPython}}{R3 -> Real[3,3] Field given by a Python script.}
\KeyItem{\hyperB{FieldPython::TYPE}{TYPE}}{selection: Field:R3 -> Real[3,3]\_TYPE\_selection}{FieldPython}{\AddDoc{FieldPython::TYPE}}{Sub-record selection.}
\KeyItem{\hyperB{FieldPython::script-string}{script\_string}}{String (generic)}{"Obligatory if 'script\_file' is not given."}{\AddDoc{FieldPython::script\_string}}{Python script given as in place string}
\KeyItem{\hyperB{FieldPython::script-file}{script\_file}}{input file name}{"Obligatory if 'script\_striong' is not given."}{\AddDoc{FieldPython::script\_file}}{Python script given as external file}
\KeyItem{\hyperB{FieldPython::function}{function}}{String (generic)}{\textless\it obligatory\textgreater}{\AddDoc{FieldPython::function}}{Function in the given script that returns tuple containing components of the return type.
For NxM tensor values: tensor(row,col) = tuple( M*row + col ).}
\end{RecordType}

\begin{RecordType}{\HTRaised{IT::FieldFormula}{FieldFormula}}{\hyperlink{IT::Field:R3 - Real[3,3]}{Field:R3 -> Real[3,3]}}{}{\AddDoc{FieldFormula}}{R3 -> Real[3,3] Field given by runtime interpreted formula.}
\KeyItem{\hyperB{FieldFormula::TYPE}{TYPE}}{selection: Field:R3 -> Real[3,3]\_TYPE\_selection}{FieldFormula}{\AddDoc{FieldFormula::TYPE}}{Sub-record selection.}
\KeyItem{\hyperB{FieldFormula::value}{value}}{Array [1, ] of Array [1, ] of String (generic)}{\textless\it obligatory\textgreater}{\AddDoc{FieldFormula::value}}{String, array of strings, or matrix of strings with formulas for individual entries of scalar, vector, or tensor value respectively.
For vector values, you can use just one string to enter homogeneous vector.
For square NxN-matrix values, you can use:
* array of strings of size N to enter diagonal matrix
* array of strings of size (N+1)*N/2 to enter symmetric matrix (upper triangle, row by row)
* just one string to enter (spatially variable) multiple of the unit matrix.
Formula can contain variables x,y,z,t and usual operators and functions.}
\end{RecordType}

\begin{RecordType}{\HTRaised{IT::FieldElementwise}{FieldElementwise}}{\hyperlink{IT::Field:R3 - Real[3,3]}{Field:R3 -> Real[3,3]}}{}{\AddDoc{FieldElementwise}}{R3 -> Real[3,3] Field constant in space.}
\KeyItem{\hyperB{FieldElementwise::TYPE}{TYPE}}{selection: Field:R3 -> Real[3,3]\_TYPE\_selection}{FieldElementwise}{\AddDoc{FieldElementwise::TYPE}}{Sub-record selection.}
\KeyItem{\hyperB{FieldElementwise::gmsh-file}{gmsh\_file}}{input file name}{\textless\it obligatory\textgreater}{\AddDoc{FieldElementwise::gmsh\_file}}{Input file with ASCII GMSH file format.}
\KeyItem{\hyperB{FieldElementwise::field-name}{field\_name}}{String (generic)}{\textless\it obligatory\textgreater}{\AddDoc{FieldElementwise::field\_name}}{The values of the Field are read from the \$ElementData section with field name given by this key.}
\end{RecordType}

\begin{RecordType}{\HTRaised{IT::FieldInterpolatedP0}{FieldInterpolatedP0}}{\hyperlink{IT::Field:R3 - Real[3,3]}{Field:R3 -> Real[3,3]}}{}{\AddDoc{FieldInterpolatedP0}}{R3 -> Real[3,3] Field constant in space.}
\KeyItem{\hyperB{FieldInterpolatedP0::TYPE}{TYPE}}{selection: Field:R3 -> Real[3,3]\_TYPE\_selection}{FieldInterpolatedP0}{\AddDoc{FieldInterpolatedP0::TYPE}}{Sub-record selection.}
\KeyItem{\hyperB{FieldInterpolatedP0::gmsh-file}{gmsh\_file}}{input file name}{\textless\it obligatory\textgreater}{\AddDoc{FieldInterpolatedP0::gmsh\_file}}{Input file with ASCII GMSH file format.}
\KeyItem{\hyperB{FieldInterpolatedP0::field-name}{field\_name}}{String (generic)}{\textless\it obligatory\textgreater}{\AddDoc{FieldInterpolatedP0::field\_name}}{The values of the Field are read from the \$ElementData section with field name given by this key.}
\end{RecordType}

\begin{RecordType}{\HTRaised{IT::Unsteady-MH}{Unsteady\_MH}}{\hyperlink{IT::DarcyFlowMH}{DarcyFlowMH}}{}{\AddDoc{Unsteady\_MH}}{Mixed-Hybrid solver for unsteady saturated Darcy flow.}
\KeyItem{\hyperB{Unsteady-MH::TYPE}{TYPE}}{selection: DarcyFlowMH\_TYPE\_selection}{Unsteady\_MH}{\AddDoc{Unsteady\_MH::TYPE}}{Sub-record selection.}
\KeyItem{\hyperB{Unsteady-MH::n-schurs}{n\_schurs}}{Integer [0, 2]}{2}{\AddDoc{Unsteady\_MH::n\_schurs}}{Number of Schur complements to perform when solving MH sytem.}
\KeyItem{\hyperB{Unsteady-MH::solver}{solver}}{abstract type: \hyperlink{IT::Solver}{Solver}}{\textless\it obligatory\textgreater}{\AddDoc{Unsteady\_MH::solver}}{Linear solver for MH problem.}
\KeyItem{\hyperB{Unsteady-MH::output}{output}}{record: \hyperlink{IT::DarcyMHOutput}{DarcyMHOutput}}{\textless\it obligatory\textgreater}{\AddDoc{Unsteady\_MH::output}}{Parameters of output form MH module.}
\KeyItem{\hyperB{Unsteady-MH::mortar-method}{mortar\_method}}{selection: \hyperlink{IT::MH-MortarMethod}{MH\_MortarMethod}}{None}{\AddDoc{Unsteady\_MH::mortar\_method}}{Method for coupling Darcy flow between dimensions.}
\KeyItem{\hyperB{Unsteady-MH::time}{time}}{record: \hyperlink{IT::TimeGovernor}{TimeGovernor}}{\textless\it obligatory\textgreater}{\AddDoc{Unsteady\_MH::time}}{Time governor setting for the unsteady Darcy flow model.}
\KeyItem{\hyperB{Unsteady-MH::bc-data}{bc\_data}}{Array  of record: \hyperlink{IT::DarcyFlowMH-Steady-BoundaryData}{DarcyFlowMH\_Steady\_BoundaryData}}{\textless\it obligatory\textgreater}{\AddDoc{Unsteady\_MH::bc\_data}}{}
\KeyItem{\hyperB{Unsteady-MH::bulk-data}{bulk\_data}}{Array  of record: \hyperlink{IT::DarcyFlowMH-Steady-BulkData}{DarcyFlowMH\_Steady\_BulkData}}{\textless\it obligatory\textgreater}{\AddDoc{Unsteady\_MH::bulk\_data}}{}
\end{RecordType}

\begin{RecordType}{\HTRaised{IT::Unsteady-LMH}{Unsteady\_LMH}}{\hyperlink{IT::DarcyFlowMH}{DarcyFlowMH}}{}{\AddDoc{Unsteady\_LMH}}{Lumped Mixed-Hybrid solver for unsteady saturated Darcy flow.}
\KeyItem{\hyperB{Unsteady-LMH::TYPE}{TYPE}}{selection: DarcyFlowMH\_TYPE\_selection}{Unsteady\_LMH}{\AddDoc{Unsteady\_LMH::TYPE}}{Sub-record selection.}
\KeyItem{\hyperB{Unsteady-LMH::n-schurs}{n\_schurs}}{Integer [0, 2]}{2}{\AddDoc{Unsteady\_LMH::n\_schurs}}{Number of Schur complements to perform when solving MH sytem.}
\KeyItem{\hyperB{Unsteady-LMH::solver}{solver}}{abstract type: \hyperlink{IT::Solver}{Solver}}{\textless\it obligatory\textgreater}{\AddDoc{Unsteady\_LMH::solver}}{Linear solver for MH problem.}
\KeyItem{\hyperB{Unsteady-LMH::output}{output}}{record: \hyperlink{IT::DarcyMHOutput}{DarcyMHOutput}}{\textless\it obligatory\textgreater}{\AddDoc{Unsteady\_LMH::output}}{Parameters of output form MH module.}
\KeyItem{\hyperB{Unsteady-LMH::mortar-method}{mortar\_method}}{selection: \hyperlink{IT::MH-MortarMethod}{MH\_MortarMethod}}{None}{\AddDoc{Unsteady\_LMH::mortar\_method}}{Method for coupling Darcy flow between dimensions.}
\KeyItem{\hyperB{Unsteady-LMH::time}{time}}{record: \hyperlink{IT::TimeGovernor}{TimeGovernor}}{\textless\it obligatory\textgreater}{\AddDoc{Unsteady\_LMH::time}}{Time governor setting for the unsteady Darcy flow model.}
\KeyItem{\hyperB{Unsteady-LMH::bc-data}{bc\_data}}{Array  of record: \hyperlink{IT::DarcyFlowMH-Steady-BoundaryData}{DarcyFlowMH\_Steady\_BoundaryData}}{\textless\it obligatory\textgreater}{\AddDoc{Unsteady\_LMH::bc\_data}}{}
\KeyItem{\hyperB{Unsteady-LMH::bulk-data}{bulk\_data}}{Array  of record: \hyperlink{IT::DarcyFlowMH-Steady-BulkData}{DarcyFlowMH\_Steady\_BulkData}}{\textless\it obligatory\textgreater}{\AddDoc{Unsteady\_LMH::bulk\_data}}{}
\end{RecordType}

\begin{AbstractType}{\HTRaised{IT::Transport}{Transport}}{}{\AddDoc{Transport}}{Secondary equation for transport of substances.}
\Descendant{\hyperlink{IT::TransportOperatorSplitting}{TransportOperatorSplitting}}
\Descendant{\hyperlink{IT::AdvectionDiffusion-DG}{AdvectionDiffusion\_DG}}
\end{AbstractType}

\begin{RecordType}{\HTRaised{IT::TransportOperatorSplitting}{TransportOperatorSplitting}}{\hyperlink{IT::Transport}{Transport}}{}{\AddDoc{TransportOperatorSplitting}}{Explicit FVM transport (no diffusion)
coupled with reaction and sorption model (ODE per element)
 via operator splitting.}
\KeyItem{\hyperB{TransportOperatorSplitting::TYPE}{TYPE}}{selection: Transport\_TYPE\_selection}{TransportOperatorSplitting}{\AddDoc{TransportOperatorSplitting::TYPE}}{Sub-record selection.}
\KeyItem{\hyperB{TransportOperatorSplitting::time}{time}}{record: \hyperlink{IT::TimeGovernor}{TimeGovernor}}{\textless\it obligatory\textgreater}{\AddDoc{TransportOperatorSplitting::time}}{Time governor setting for the transport model.}
\KeyItem{\hyperB{TransportOperatorSplitting::substances}{substances}}{Array  of String (generic)}{\textless\it obligatory\textgreater}{\AddDoc{TransportOperatorSplitting::substances}}{Names of transported substances.}
\KeyItem{\hyperB{TransportOperatorSplitting::sorption-enable}{sorption\_enable}}{Bool}{false}{\AddDoc{TransportOperatorSplitting::sorption\_enable}}{Model of sorption.}
\KeyItem{\hyperB{TransportOperatorSplitting::dual-porosity}{dual\_porosity}}{Bool}{false}{\AddDoc{TransportOperatorSplitting::dual\_porosity}}{Dual porosity model.}
\KeyItem{\hyperB{TransportOperatorSplitting::output}{output}}{record: \hyperlink{IT::TransportOutput}{TransportOutput}}{\textless\it obligatory\textgreater}{\AddDoc{TransportOperatorSplitting::output}}{Parameters of output stream.}
\KeyItem{\hyperB{TransportOperatorSplitting::reactions}{reactions}}{abstract type: \hyperlink{IT::Reactions}{Reactions}}{\textless\it optional\textgreater}{\AddDoc{TransportOperatorSplitting::reactions}}{Initialization of per element reactions.}
\KeyItem{\hyperB{TransportOperatorSplitting::adsorptions}{adsorptions}}{record: \hyperlink{IT::Sorptions}{Sorptions}}{\textless\it optional\textgreater}{\AddDoc{TransportOperatorSplitting::adsorptions}}{Initialization of per element sorptions.}
\KeyItem{\hyperB{TransportOperatorSplitting::bc-data}{bc\_data}}{Array  of record: \hyperlink{IT::TransportOperatorSplitting-BoundaryData}{TransportOperatorSplitting\_BoundaryData}}{\textless\it obligatory\textgreater}{\AddDoc{TransportOperatorSplitting::bc\_data}}{}
\KeyItem{\hyperB{TransportOperatorSplitting::bulk-data}{bulk\_data}}{Array  of record: \hyperlink{IT::TransportOperatorSplitting-BulkData}{TransportOperatorSplitting\_BulkData}}{\textless\it obligatory\textgreater}{\AddDoc{TransportOperatorSplitting::bulk\_data}}{}
\end{RecordType}

\begin{RecordType}{\HTRaised{IT::TransportOutput}{TransportOutput}}{}{}{\AddDoc{TransportOutput}}{Output setting for transport equations.}
\KeyItem{\hyperB{TransportOutput::output-stream}{output\_stream}}{record: \hyperlink{IT::OutputStream}{OutputStream}}{\textless\it obligatory\textgreater}{\AddDoc{TransportOutput::output\_stream}}{Parameters of output stream.}
\KeyItem{\hyperB{TransportOutput::save-step}{save\_step}}{Double [0, ]}{\textless\it obligatory\textgreater}{\AddDoc{TransportOutput::save\_step}}{Interval between outputs.}
\KeyItem{\hyperB{TransportOutput::output-times}{output\_times}}{Array  of Double [0, ]}{\textless\it optional\textgreater}{\AddDoc{TransportOutput::output\_times}}{Explicit array of output times (can be combined with 'save\_step'.}
\KeyItem{\hyperB{TransportOutput::conc-mobile-p0}{conc\_mobile\_p0}}{String (generic)}{\textless\it optional\textgreater}{\AddDoc{TransportOutput::conc\_mobile\_p0}}{Name of output stream for P0 approximation of the concentration in mobile phase.}
\KeyItem{\hyperB{TransportOutput::conc-immobile-p0}{conc\_immobile\_p0}}{String (generic)}{\textless\it optional\textgreater}{\AddDoc{TransportOutput::conc\_immobile\_p0}}{Name of output stream for P0 approximation of the concentration in immobile phase.}
\KeyItem{\hyperB{TransportOutput::conc-mobile-sorbed-p0}{conc\_mobile\_sorbed\_p0}}{String (generic)}{\textless\it optional\textgreater}{\AddDoc{TransportOutput::conc\_mobile\_sorbed\_p0}}{Name of output stream for P0 approximation of the surface concentration of sorbed mobile phase.}
\KeyItem{\hyperB{TransportOutput::conc-immobile-sorbed-p0}{conc\_immobile\_sorbed\_p0}}{String (generic)}{\textless\it optional\textgreater}{\AddDoc{TransportOutput::conc\_immobile\_sorbed\_p0}}{Name of output stream for P0 approximation of the surface concentration of sorbed immobile phase.}
\end{RecordType}

\begin{AbstractType}{\HTRaised{IT::Reactions}{Reactions}}{}{\AddDoc{Reactions}}{Equation for reading information about simple chemical reactions.}
\Descendant{\hyperlink{IT::Sorptions}{Sorptions}}
\Descendant{\hyperlink{IT::LinearReactions}{LinearReactions}}
\Descendant{\hyperlink{IT::PadeApproximant}{PadeApproximant}}
\Descendant{\hyperlink{IT::Isotope}{Isotope}}
\end{AbstractType}

\begin{RecordType}{\HTRaised{IT::Sorptions}{Sorptions}}{\hyperlink{IT::Reactions}{Reactions}}{}{\AddDoc{Sorptions}}{Information about all the limited solubility affected sorptions.}
\KeyItem{\hyperB{Sorptions::TYPE}{TYPE}}{selection: Reactions\_TYPE\_selection}{Sorptions}{\AddDoc{Sorptions::TYPE}}{Sub-record selection.}
\KeyItem{\hyperB{Sorptions::solvent-dens}{solvent\_dens}}{Double }{1.0}{\AddDoc{Sorptions::solvent\_dens}}{Density of the solvent.}
\KeyItem{\hyperB{Sorptions::substeps}{substeps}}{Integer }{100}{\AddDoc{Sorptions::substeps}}{Number of equidistant substeps, molar mass and isotherm intersections}
\KeyItem{\hyperB{Sorptions::species}{species}}{Array  of String (generic)}{\textless\it obligatory\textgreater}{\AddDoc{Sorptions::species}}{Names of all the sorbing species}
\KeyItem{\hyperB{Sorptions::molar-masses}{molar\_masses}}{Array  of Double }{\textless\it obligatory\textgreater}{\AddDoc{Sorptions::molar\_masses}}{Specifies molar masses of all the sorbing species}
\KeyItem{\hyperB{Sorptions::solubility}{solubility}}{Array  of Double }{\textless\it obligatory\textgreater}{\AddDoc{Sorptions::solubility}}{Specifies solubility limits of all the sorbing species}
\KeyItem{\hyperB{Sorptions::bulk-data}{bulk\_data}}{Array  of record: \hyperlink{IT::Sorption-BulkData}{Sorption\_BulkData}}{\textless\it obligatory\textgreater}{\AddDoc{Sorptions::bulk\_data}}{Containes region specific data necessery to construct isotherms.}
\end{RecordType}

\begin{RecordType}{\HTRaised{IT::Sorption-BulkData}{Sorption\_BulkData}}{}{}{\AddDoc{Sorption\_BulkData}}{Record to set BULK fields of the equation 'Sorption'.
The fields are set only on the domain specified by one of the keys: 'region', 'rid', 'r\_set'
and after the time given by the key 'time'. The field setting can be overridden by
 any Sorption\_BulkData record that comes later in the bulk data array.}
\KeyItem{\hyperB{Sorption-BulkData::r-set}{r\_set}}{String (generic)}{\textless\it optional\textgreater}{\AddDoc{Sorption\_BulkData::r\_set}}{Name of region set where to set fields.}
\KeyItem{\hyperB{Sorption-BulkData::region}{region}}{String (generic)}{\textless\it optional\textgreater}{\AddDoc{Sorption\_BulkData::region}}{Label of the region where to set fields. }
\KeyItem{\hyperB{Sorption-BulkData::rid}{rid}}{Integer [0, ]}{\textless\it optional\textgreater}{\AddDoc{Sorption\_BulkData::rid}}{ID of the region where to set fields.}
\KeyItem{\hyperB{Sorption-BulkData::time}{time}}{Double [0, ]}{0.0}{\AddDoc{Sorption\_BulkData::time}}{Apply field setting in this record after this time.
These times have to form an increasing sequence.}
\KeyItem{\hyperB{Sorption-BulkData::rock-density}{rock\_density}}{abstract type: \hyperlink{IT::Field:R3 - Real}{Field:R3 -> Real}}{\textless\it optional\textgreater}{\AddDoc{Sorption\_BulkData::rock\_density}}{Rock matrix density.}
\KeyItem{\hyperB{Sorption-BulkData::sorption-types}{sorption\_types}}{abstract type: \hyperlink{IT::Field:R3 - Enum[n]}{Field:R3 -> Enum[n]}}{\textless\it optional\textgreater}{\AddDoc{Sorption\_BulkData::sorption\_types}}{Considered adsorption is described by selected isotherm.}
\KeyItem{\hyperB{Sorption-BulkData::mult-coefs}{mult\_coefs}}{abstract type: \hyperlink{IT::Field:R3 - Real[n]}{Field:R3 -> Real[n]}}{\textless\it optional\textgreater}{\AddDoc{Sorption\_BulkData::mult\_coefs}}{Multiplication parameters (k, omega) in either Langmuir c\_s = omega * (alpha*c\_a)/(1- alpha*c\_a) or in linear c\_s = k * c\_a isothermal description.}
\KeyItem{\hyperB{Sorption-BulkData::second-params}{second\_params}}{abstract type: \hyperlink{IT::Field:R3 - Real[n]}{Field:R3 -> Real[n]}}{\textless\it optional\textgreater}{\AddDoc{Sorption\_BulkData::second\_params}}{Second parameters (alpha, ...) defining isotherm  c\_s = omega * (alpha*c\_a)/(1- alpha*c\_a).}
\KeyItem{\hyperB{Sorption-BulkData::mob-porosity}{mob\_porosity}}{abstract type: \hyperlink{IT::Field:R3 - Real}{Field:R3 -> Real}}{\textless\it optional\textgreater}{\AddDoc{Sorption\_BulkData::mob\_porosity}}{Mobile porosity of the rock matrix.}
\end{RecordType}

\begin{AbstractType}{\HTRaised{IT::Field:R3 - Enum[n]}{Field:R3 -> Enum[n]}}{\hyperlink{IT::FieldConstant}{FieldConstant}}{\AddDoc{Field:R3 -> Enum[n]}}{Abstract record for all time-space functions.}
\Descendant{\hyperlink{IT::FieldConstant}{FieldConstant}}
\Descendant{\hyperlink{IT::FieldFormula}{FieldFormula}}
\Descendant{\hyperlink{IT::FieldPython}{FieldPython}}
\Descendant{\hyperlink{IT::FieldElementwise}{FieldElementwise}}
\end{AbstractType}

\begin{RecordType}{\HTRaised{IT::FieldConstant}{FieldConstant}}{\hyperlink{IT::Field:R3 - Enum[n]}{Field:R3 -> Enum[n]}}{\hyperlink{FieldConstant::value}{value}}{\AddDoc{FieldConstant}}{R3 -> Enum[n] Field constant in space.}
\KeyItem{\hyperB{FieldConstant::TYPE}{TYPE}}{selection: Field:R3 -> Enum[n]\_TYPE\_selection}{FieldConstant}{\AddDoc{FieldConstant::TYPE}}{Sub-record selection.}
\KeyItem{\hyperB{FieldConstant::value}{value}}{Array [1, ] of selection: \hyperlink{IT::SorptionType}{SorptionType}}{\textless\it obligatory\textgreater}{\AddDoc{FieldConstant::value}}{Value of the constant field.
For vector values, you can use scalar value to enter constant vector.
For square NxN-matrix values, you can use:
* vector of size N to enter diagonal matrix
* vector of size (N+1)*N/2 to enter symmetric matrix (upper triangle, row by row)
* scalar to enter multiple of the unit matrix.}
\end{RecordType}

\begin{SelectionType}{\HTRaised{IT::SorptionType}{SorptionType}}{}
\KeyItem{none}{No sorption considered}
\KeyItem{linear}{Linear isotherm described sorption considered.}
\KeyItem{langmuir}{Langmuir isotherm described sorption considered}
\KeyItem{freundlich}{Freundlich isotherm described sorption considered}
\end{SelectionType}

\begin{RecordType}{\HTRaised{IT::FieldFormula}{FieldFormula}}{\hyperlink{IT::Field:R3 - Enum[n]}{Field:R3 -> Enum[n]}}{}{\AddDoc{FieldFormula}}{R3 -> Enum[n] Field given by runtime interpreted formula.}
\KeyItem{\hyperB{FieldFormula::TYPE}{TYPE}}{selection: Field:R3 -> Enum[n]\_TYPE\_selection}{FieldFormula}{\AddDoc{FieldFormula::TYPE}}{Sub-record selection.}
\KeyItem{\hyperB{FieldFormula::value}{value}}{Array [1, ] of String (generic)}{\textless\it obligatory\textgreater}{\AddDoc{FieldFormula::value}}{String, array of strings, or matrix of strings with formulas for individual entries of scalar, vector, or tensor value respectively.
For vector values, you can use just one string to enter homogeneous vector.
For square NxN-matrix values, you can use:
* array of strings of size N to enter diagonal matrix
* array of strings of size (N+1)*N/2 to enter symmetric matrix (upper triangle, row by row)
* just one string to enter (spatially variable) multiple of the unit matrix.
Formula can contain variables x,y,z,t and usual operators and functions.}
\end{RecordType}

\begin{RecordType}{\HTRaised{IT::FieldPython}{FieldPython}}{\hyperlink{IT::Field:R3 - Enum[n]}{Field:R3 -> Enum[n]}}{}{\AddDoc{FieldPython}}{R3 -> Enum[n] Field given by a Python script.}
\KeyItem{\hyperB{FieldPython::TYPE}{TYPE}}{selection: Field:R3 -> Enum[n]\_TYPE\_selection}{FieldPython}{\AddDoc{FieldPython::TYPE}}{Sub-record selection.}
\KeyItem{\hyperB{FieldPython::script-string}{script\_string}}{String (generic)}{"Obligatory if 'script\_file' is not given."}{\AddDoc{FieldPython::script\_string}}{Python script given as in place string}
\KeyItem{\hyperB{FieldPython::script-file}{script\_file}}{input file name}{"Obligatory if 'script\_striong' is not given."}{\AddDoc{FieldPython::script\_file}}{Python script given as external file}
\KeyItem{\hyperB{FieldPython::function}{function}}{String (generic)}{\textless\it obligatory\textgreater}{\AddDoc{FieldPython::function}}{Function in the given script that returns tuple containing components of the return type.
For NxM tensor values: tensor(row,col) = tuple( M*row + col ).}
\end{RecordType}

\begin{RecordType}{\HTRaised{IT::FieldElementwise}{FieldElementwise}}{\hyperlink{IT::Field:R3 - Enum[n]}{Field:R3 -> Enum[n]}}{}{\AddDoc{FieldElementwise}}{R3 -> Enum[n] Field constant in space.}
\KeyItem{\hyperB{FieldElementwise::TYPE}{TYPE}}{selection: Field:R3 -> Enum[n]\_TYPE\_selection}{FieldElementwise}{\AddDoc{FieldElementwise::TYPE}}{Sub-record selection.}
\KeyItem{\hyperB{FieldElementwise::gmsh-file}{gmsh\_file}}{input file name}{\textless\it obligatory\textgreater}{\AddDoc{FieldElementwise::gmsh\_file}}{Input file with ASCII GMSH file format.}
\KeyItem{\hyperB{FieldElementwise::field-name}{field\_name}}{String (generic)}{\textless\it obligatory\textgreater}{\AddDoc{FieldElementwise::field\_name}}{The values of the Field are read from the \$ElementData section with field name given by this key.}
\end{RecordType}

\begin{AbstractType}{\HTRaised{IT::Field:R3 - Real[n]}{Field:R3 -> Real[n]}}{\hyperlink{IT::FieldConstant}{FieldConstant}}{\AddDoc{Field:R3 -> Real[n]}}{Abstract record for all time-space functions.}
\Descendant{\hyperlink{IT::FieldConstant}{FieldConstant}}
\Descendant{\hyperlink{IT::FieldPython}{FieldPython}}
\Descendant{\hyperlink{IT::FieldFormula}{FieldFormula}}
\Descendant{\hyperlink{IT::FieldElementwise}{FieldElementwise}}
\Descendant{\hyperlink{IT::FieldInterpolatedP0}{FieldInterpolatedP0}}
\end{AbstractType}

\begin{RecordType}{\HTRaised{IT::FieldConstant}{FieldConstant}}{\hyperlink{IT::Field:R3 - Real[n]}{Field:R3 -> Real[n]}}{\hyperlink{FieldConstant::value}{value}}{\AddDoc{FieldConstant}}{R3 -> Real[n] Field constant in space.}
\KeyItem{\hyperB{FieldConstant::TYPE}{TYPE}}{selection: Field:R3 -> Real[n]\_TYPE\_selection}{FieldConstant}{\AddDoc{FieldConstant::TYPE}}{Sub-record selection.}
\KeyItem{\hyperB{FieldConstant::value}{value}}{Array [1, ] of Double }{\textless\it obligatory\textgreater}{\AddDoc{FieldConstant::value}}{Value of the constant field.
For vector values, you can use scalar value to enter constant vector.
For square NxN-matrix values, you can use:
* vector of size N to enter diagonal matrix
* vector of size (N+1)*N/2 to enter symmetric matrix (upper triangle, row by row)
* scalar to enter multiple of the unit matrix.}
\end{RecordType}

\begin{RecordType}{\HTRaised{IT::FieldPython}{FieldPython}}{\hyperlink{IT::Field:R3 - Real[n]}{Field:R3 -> Real[n]}}{}{\AddDoc{FieldPython}}{R3 -> Real[n] Field given by a Python script.}
\KeyItem{\hyperB{FieldPython::TYPE}{TYPE}}{selection: Field:R3 -> Real[n]\_TYPE\_selection}{FieldPython}{\AddDoc{FieldPython::TYPE}}{Sub-record selection.}
\KeyItem{\hyperB{FieldPython::script-string}{script\_string}}{String (generic)}{"Obligatory if 'script\_file' is not given."}{\AddDoc{FieldPython::script\_string}}{Python script given as in place string}
\KeyItem{\hyperB{FieldPython::script-file}{script\_file}}{input file name}{"Obligatory if 'script\_striong' is not given."}{\AddDoc{FieldPython::script\_file}}{Python script given as external file}
\KeyItem{\hyperB{FieldPython::function}{function}}{String (generic)}{\textless\it obligatory\textgreater}{\AddDoc{FieldPython::function}}{Function in the given script that returns tuple containing components of the return type.
For NxM tensor values: tensor(row,col) = tuple( M*row + col ).}
\end{RecordType}

\begin{RecordType}{\HTRaised{IT::FieldFormula}{FieldFormula}}{\hyperlink{IT::Field:R3 - Real[n]}{Field:R3 -> Real[n]}}{}{\AddDoc{FieldFormula}}{R3 -> Real[n] Field given by runtime interpreted formula.}
\KeyItem{\hyperB{FieldFormula::TYPE}{TYPE}}{selection: Field:R3 -> Real[n]\_TYPE\_selection}{FieldFormula}{\AddDoc{FieldFormula::TYPE}}{Sub-record selection.}
\KeyItem{\hyperB{FieldFormula::value}{value}}{Array [1, ] of String (generic)}{\textless\it obligatory\textgreater}{\AddDoc{FieldFormula::value}}{String, array of strings, or matrix of strings with formulas for individual entries of scalar, vector, or tensor value respectively.
For vector values, you can use just one string to enter homogeneous vector.
For square NxN-matrix values, you can use:
* array of strings of size N to enter diagonal matrix
* array of strings of size (N+1)*N/2 to enter symmetric matrix (upper triangle, row by row)
* just one string to enter (spatially variable) multiple of the unit matrix.
Formula can contain variables x,y,z,t and usual operators and functions.}
\end{RecordType}

\begin{RecordType}{\HTRaised{IT::FieldElementwise}{FieldElementwise}}{\hyperlink{IT::Field:R3 - Real[n]}{Field:R3 -> Real[n]}}{}{\AddDoc{FieldElementwise}}{R3 -> Real[n] Field constant in space.}
\KeyItem{\hyperB{FieldElementwise::TYPE}{TYPE}}{selection: Field:R3 -> Real[n]\_TYPE\_selection}{FieldElementwise}{\AddDoc{FieldElementwise::TYPE}}{Sub-record selection.}
\KeyItem{\hyperB{FieldElementwise::gmsh-file}{gmsh\_file}}{input file name}{\textless\it obligatory\textgreater}{\AddDoc{FieldElementwise::gmsh\_file}}{Input file with ASCII GMSH file format.}
\KeyItem{\hyperB{FieldElementwise::field-name}{field\_name}}{String (generic)}{\textless\it obligatory\textgreater}{\AddDoc{FieldElementwise::field\_name}}{The values of the Field are read from the \$ElementData section with field name given by this key.}
\end{RecordType}

\begin{RecordType}{\HTRaised{IT::FieldInterpolatedP0}{FieldInterpolatedP0}}{\hyperlink{IT::Field:R3 - Real[n]}{Field:R3 -> Real[n]}}{}{\AddDoc{FieldInterpolatedP0}}{R3 -> Real[n] Field constant in space.}
\KeyItem{\hyperB{FieldInterpolatedP0::TYPE}{TYPE}}{selection: Field:R3 -> Real[n]\_TYPE\_selection}{FieldInterpolatedP0}{\AddDoc{FieldInterpolatedP0::TYPE}}{Sub-record selection.}
\KeyItem{\hyperB{FieldInterpolatedP0::gmsh-file}{gmsh\_file}}{input file name}{\textless\it obligatory\textgreater}{\AddDoc{FieldInterpolatedP0::gmsh\_file}}{Input file with ASCII GMSH file format.}
\KeyItem{\hyperB{FieldInterpolatedP0::field-name}{field\_name}}{String (generic)}{\textless\it obligatory\textgreater}{\AddDoc{FieldInterpolatedP0::field\_name}}{The values of the Field are read from the \$ElementData section with field name given by this key.}
\end{RecordType}

\begin{RecordType}{\HTRaised{IT::LinearReactions}{LinearReactions}}{\hyperlink{IT::Reactions}{Reactions}}{}{\AddDoc{LinearReactions}}{Information for a decision about the way to simulate radioactive decay.}
\KeyItem{\hyperB{LinearReactions::TYPE}{TYPE}}{selection: Reactions\_TYPE\_selection}{LinearReactions}{\AddDoc{LinearReactions::TYPE}}{Sub-record selection.}
\KeyItem{\hyperB{LinearReactions::decays}{decays}}{Array  of record: \hyperlink{IT::Substep}{Substep}}{\textless\it obligatory\textgreater}{\AddDoc{LinearReactions::decays}}{Description of particular decay chain substeps.}
\KeyItem{\hyperB{LinearReactions::matrix-exp-on}{matrix\_exp\_on}}{Bool}{false}{\AddDoc{LinearReactions::matrix\_exp\_on}}{Enables to use Pade approximant of matrix exponential.}
\end{RecordType}

\begin{RecordType}{\HTRaised{IT::Substep}{Substep}}{}{}{\AddDoc{Substep}}{Equation for reading information about radioactive decays.}
\KeyItem{\hyperB{Substep::parent}{parent}}{String (generic)}{\textless\it obligatory\textgreater}{\AddDoc{Substep::parent}}{Identifier of an isotope.}
\KeyItem{\hyperB{Substep::half-life}{half\_life}}{Double }{\textless\it optional\textgreater}{\AddDoc{Substep::half\_life}}{Half life of the parent substance.}
\KeyItem{\hyperB{Substep::kinetic}{kinetic}}{Double }{\textless\it optional\textgreater}{\AddDoc{Substep::kinetic}}{Kinetic constants describing first order reactions.}
\KeyItem{\hyperB{Substep::products}{products}}{Array  of String (generic)}{\textless\it obligatory\textgreater}{\AddDoc{Substep::products}}{Identifies isotopes which decays parental atom to.}
\KeyItem{\hyperB{Substep::branch-ratios}{branch\_ratios}}{Array  of Double }{1.0}{\AddDoc{Substep::branch\_ratios}}{Decay chain branching percentage.}
\end{RecordType}

\begin{RecordType}{\HTRaised{IT::PadeApproximant}{PadeApproximant}}{\hyperlink{IT::Reactions}{Reactions}}{}{\AddDoc{PadeApproximant}}{Abstract record with an information about pade approximant parameters.}
\KeyItem{\hyperB{PadeApproximant::TYPE}{TYPE}}{selection: Reactions\_TYPE\_selection}{PadeApproximant}{\AddDoc{PadeApproximant::TYPE}}{Sub-record selection.}
\KeyItem{\hyperB{PadeApproximant::decays}{decays}}{Array  of record: \hyperlink{IT::Substep}{Substep}}{\textless\it obligatory\textgreater}{\AddDoc{PadeApproximant::decays}}{Description of particular decay chain substeps.}
\KeyItem{\hyperB{PadeApproximant::nom-pol-deg}{nom\_pol\_deg}}{Integer }{2}{\AddDoc{PadeApproximant::nom\_pol\_deg}}{Polynomial degree of the nominator of Pade approximant.}
\KeyItem{\hyperB{PadeApproximant::den-pol-deg}{den\_pol\_deg}}{Integer }{2}{\AddDoc{PadeApproximant::den\_pol\_deg}}{Polynomial degree of the nominator of Pade approximant}
\end{RecordType}

\begin{RecordType}{\HTRaised{IT::Isotope}{Isotope}}{\hyperlink{IT::Reactions}{Reactions}}{}{\AddDoc{Isotope}}{Definition of information about a single isotope.}
\KeyItem{\hyperB{Isotope::TYPE}{TYPE}}{selection: Reactions\_TYPE\_selection}{Isotope}{\AddDoc{Isotope::TYPE}}{Sub-record selection.}
\KeyItem{\hyperB{Isotope::identifier}{identifier}}{Integer }{\textless\it obligatory\textgreater}{\AddDoc{Isotope::identifier}}{Identifier of the isotope.}
\KeyItem{\hyperB{Isotope::half-life}{half\_life}}{Double }{\textless\it obligatory\textgreater}{\AddDoc{Isotope::half\_life}}{Half life parameter.}
\end{RecordType}

\begin{RecordType}{\HTRaised{IT::TransportOperatorSplitting-BoundaryData}{TransportOperatorSplitting\_BoundaryData}}{}{}{\AddDoc{TransportOperatorSplitting\_BoundaryData}}{Record to set BOUNDARY fields of the equation 'TransportOperatorSplitting'.
The fields are set only on the domain specified by one of the keys: 'region', 'rid', 'r\_set'
and after the time given by the key 'time'. The field setting can be overridden by
 any TransportOperatorSplitting\_BoundaryData record that comes later in the boundary data array.}
\KeyItem{\hyperB{TransportOperatorSplitting-BoundaryData::r-set}{r\_set}}{String (generic)}{\textless\it optional\textgreater}{\AddDoc{TransportOperatorSplitting\_BoundaryData::r\_set}}{Name of region set where to set fields.}
\KeyItem{\hyperB{TransportOperatorSplitting-BoundaryData::region}{region}}{String (generic)}{\textless\it optional\textgreater}{\AddDoc{TransportOperatorSplitting\_BoundaryData::region}}{Label of the region where to set fields. }
\KeyItem{\hyperB{TransportOperatorSplitting-BoundaryData::rid}{rid}}{Integer [0, ]}{\textless\it optional\textgreater}{\AddDoc{TransportOperatorSplitting\_BoundaryData::rid}}{ID of the region where to set fields.}
\KeyItem{\hyperB{TransportOperatorSplitting-BoundaryData::time}{time}}{Double [0, ]}{0.0}{\AddDoc{TransportOperatorSplitting\_BoundaryData::time}}{Apply field setting in this record after this time.
These times have to form an increasing sequence.}
\KeyItem{\hyperB{TransportOperatorSplitting-BoundaryData::bc-conc}{bc\_conc}}{abstract type: \hyperlink{IT::Field:R3 - Real[n]}{Field:R3 -> Real[n]}}{\textless\it optional\textgreater}{\AddDoc{TransportOperatorSplitting\_BoundaryData::bc\_conc}}{Boundary conditions for concentrations.}
\KeyItem{\hyperB{TransportOperatorSplitting-BoundaryData::old-boundary-file}{old\_boundary\_file}}{input file name}{\textless\it optional\textgreater}{\AddDoc{TransportOperatorSplitting\_BoundaryData::old\_boundary\_file}}{Input file with boundary conditions (obsolete).}
\KeyItem{\hyperB{TransportOperatorSplitting-BoundaryData::bc-times}{bc\_times}}{Array  of Double }{\textless\it optional\textgreater}{\AddDoc{TransportOperatorSplitting\_BoundaryData::bc\_times}}{Times for changing the boundary conditions (obsolete).}
\end{RecordType}

\begin{RecordType}{\HTRaised{IT::TransportOperatorSplitting-BulkData}{TransportOperatorSplitting\_BulkData}}{}{}{\AddDoc{TransportOperatorSplitting\_BulkData}}{Record to set BULK fields of the equation 'TransportOperatorSplitting'.
The fields are set only on the domain specified by one of the keys: 'region', 'rid', 'r\_set'
and after the time given by the key 'time'. The field setting can be overridden by
 any TransportOperatorSplitting\_BulkData record that comes later in the bulk data array.}
\KeyItem{\hyperB{TransportOperatorSplitting-BulkData::r-set}{r\_set}}{String (generic)}{\textless\it optional\textgreater}{\AddDoc{TransportOperatorSplitting\_BulkData::r\_set}}{Name of region set where to set fields.}
\KeyItem{\hyperB{TransportOperatorSplitting-BulkData::region}{region}}{String (generic)}{\textless\it optional\textgreater}{\AddDoc{TransportOperatorSplitting\_BulkData::region}}{Label of the region where to set fields. }
\KeyItem{\hyperB{TransportOperatorSplitting-BulkData::rid}{rid}}{Integer [0, ]}{\textless\it optional\textgreater}{\AddDoc{TransportOperatorSplitting\_BulkData::rid}}{ID of the region where to set fields.}
\KeyItem{\hyperB{TransportOperatorSplitting-BulkData::time}{time}}{Double [0, ]}{0.0}{\AddDoc{TransportOperatorSplitting\_BulkData::time}}{Apply field setting in this record after this time.
These times have to form an increasing sequence.}
\KeyItem{\hyperB{TransportOperatorSplitting-BulkData::init-conc}{init\_conc}}{abstract type: \hyperlink{IT::Field:R3 - Real[n]}{Field:R3 -> Real[n]}}{\textless\it optional\textgreater}{\AddDoc{TransportOperatorSplitting\_BulkData::init\_conc}}{Initial concentrations.}
\KeyItem{\hyperB{TransportOperatorSplitting-BulkData::por-m}{por\_m}}{abstract type: \hyperlink{IT::Field:R3 - Real}{Field:R3 -> Real}}{\textless\it optional\textgreater}{\AddDoc{TransportOperatorSplitting\_BulkData::por\_m}}{Mobile porosity}
\KeyItem{\hyperB{TransportOperatorSplitting-BulkData::sources-density}{sources\_density}}{abstract type: \hyperlink{IT::Field:R3 - Real[n]}{Field:R3 -> Real[n]}}{\textless\it optional\textgreater}{\AddDoc{TransportOperatorSplitting\_BulkData::sources\_density}}{Density of concentration sources.}
\KeyItem{\hyperB{TransportOperatorSplitting-BulkData::sources-sigma}{sources\_sigma}}{abstract type: \hyperlink{IT::Field:R3 - Real[n]}{Field:R3 -> Real[n]}}{\textless\it optional\textgreater}{\AddDoc{TransportOperatorSplitting\_BulkData::sources\_sigma}}{Concentration flux.}
\KeyItem{\hyperB{TransportOperatorSplitting-BulkData::sources-conc}{sources\_conc}}{abstract type: \hyperlink{IT::Field:R3 - Real[n]}{Field:R3 -> Real[n]}}{\textless\it optional\textgreater}{\AddDoc{TransportOperatorSplitting\_BulkData::sources\_conc}}{Concentration sources threshold.}
\KeyItem{\hyperB{TransportOperatorSplitting-BulkData::por-imm}{por\_imm}}{abstract type: \hyperlink{IT::Field:R3 - Real}{Field:R3 -> Real}}{\textless\it optional\textgreater}{\AddDoc{TransportOperatorSplitting\_BulkData::por\_imm}}{Porosity material parameter of the immobile zone. Vector, one value for every substance.}
\KeyItem{\hyperB{TransportOperatorSplitting-BulkData::alpha}{alpha}}{abstract type: \hyperlink{IT::Field:R3 - Real[n]}{Field:R3 -> Real[n]}}{\textless\it optional\textgreater}{\AddDoc{TransportOperatorSplitting\_BulkData::alpha}}{Diffusion coefficient of non-equilibrium linear exchange between mobile and immobile zone (dual porosity). Vector, one value for every substance.}
\KeyItem{\hyperB{TransportOperatorSplitting-BulkData::sorp-type}{sorp\_type}}{abstract type: \hyperlink{IT::Field:R3 - Enum[n]}{Field:R3 -> Enum[n]}}{\textless\it optional\textgreater}{\AddDoc{TransportOperatorSplitting\_BulkData::sorp\_type}}{Type of sorption isotherm.}
\KeyItem{\hyperB{TransportOperatorSplitting-BulkData::sorp-coef0}{sorp\_coef0}}{abstract type: \hyperlink{IT::Field:R3 - Real[n]}{Field:R3 -> Real[n]}}{\textless\it optional\textgreater}{\AddDoc{TransportOperatorSplitting\_BulkData::sorp\_coef0}}{First parameter of sorption: Scaling of the isothem for all types. Vector, one value for every substance. }
\KeyItem{\hyperB{TransportOperatorSplitting-BulkData::sorp-coef1}{sorp\_coef1}}{abstract type: \hyperlink{IT::Field:R3 - Real[n]}{Field:R3 -> Real[n]}}{\textless\it optional\textgreater}{\AddDoc{TransportOperatorSplitting\_BulkData::sorp\_coef1}}{Second parameter of sorption: exponent( Freundlich isotherm), limit concentration (Langmuir isotherm). Vector, one value for every substance.}
\KeyItem{\hyperB{TransportOperatorSplitting-BulkData::phi}{phi}}{abstract type: \hyperlink{IT::Field:R3 - Real}{Field:R3 -> Real}}{\textless\it optional\textgreater}{\AddDoc{TransportOperatorSplitting\_BulkData::phi}}{Fraction of the total sorption surface exposed to the mobile zone, in interval (0,1). Used only in combination with dual porosity model. Vector, one value for every substance.}
\end{RecordType}

\begin{RecordType}{\HTRaised{IT::AdvectionDiffusion-DG}{AdvectionDiffusion\_DG}}{\hyperlink{IT::Transport}{Transport}}{}{\AddDoc{AdvectionDiffusion\_DG}}{DG solver for transport with diffusion.}
\KeyItem{\hyperB{AdvectionDiffusion-DG::TYPE}{TYPE}}{selection: Transport\_TYPE\_selection}{AdvectionDiffusion\_DG}{\AddDoc{AdvectionDiffusion\_DG::TYPE}}{Sub-record selection.}
\KeyItem{\hyperB{AdvectionDiffusion-DG::time}{time}}{record: \hyperlink{IT::TimeGovernor}{TimeGovernor}}{\textless\it obligatory\textgreater}{\AddDoc{AdvectionDiffusion\_DG::time}}{Time governor setting for the transport model.}
\KeyItem{\hyperB{AdvectionDiffusion-DG::substances}{substances}}{Array  of String (generic)}{\textless\it obligatory\textgreater}{\AddDoc{AdvectionDiffusion\_DG::substances}}{Names of transported substances.}
\KeyItem{\hyperB{AdvectionDiffusion-DG::sorption-enable}{sorption\_enable}}{Bool}{false}{\AddDoc{AdvectionDiffusion\_DG::sorption\_enable}}{Model of sorption.}
\KeyItem{\hyperB{AdvectionDiffusion-DG::dual-porosity}{dual\_porosity}}{Bool}{false}{\AddDoc{AdvectionDiffusion\_DG::dual\_porosity}}{Dual porosity model.}
\KeyItem{\hyperB{AdvectionDiffusion-DG::output}{output}}{record: \hyperlink{IT::TransportOutput}{TransportOutput}}{\textless\it obligatory\textgreater}{\AddDoc{AdvectionDiffusion\_DG::output}}{Parameters of output stream.}
\KeyItem{\hyperB{AdvectionDiffusion-DG::solver}{solver}}{abstract type: \hyperlink{IT::Solver}{Solver}}{\textless\it obligatory\textgreater}{\AddDoc{AdvectionDiffusion\_DG::solver}}{Linear solver for MH problem.}
\KeyItem{\hyperB{AdvectionDiffusion-DG::bc-data}{bc\_data}}{Array  of record: \hyperlink{IT::TransportDG-BoundaryData}{TransportDG\_BoundaryData}}{\textless\it obligatory\textgreater}{\AddDoc{AdvectionDiffusion\_DG::bc\_data}}{}
\KeyItem{\hyperB{AdvectionDiffusion-DG::bulk-data}{bulk\_data}}{Array  of record: \hyperlink{IT::TransportDG-BulkData}{TransportDG\_BulkData}}{\textless\it obligatory\textgreater}{\AddDoc{AdvectionDiffusion\_DG::bulk\_data}}{}
\KeyItem{\hyperB{AdvectionDiffusion-DG::dg-variant}{dg\_variant}}{selection: \hyperlink{IT::DG-variant}{DG\_variant}}{non-symmetric}{\AddDoc{AdvectionDiffusion\_DG::dg\_variant}}{Variant of interior penalty discontinuous Galerkin method.}
\KeyItem{\hyperB{AdvectionDiffusion-DG::dg-order}{dg\_order}}{Integer [0, 2]}{1}{\AddDoc{AdvectionDiffusion\_DG::dg\_order}}{Polynomial order for finite element in DG method (order 0 is suitable if there is no diffusion/dispersion).}
\end{RecordType}

\begin{RecordType}{\HTRaised{IT::TransportDG-BoundaryData}{TransportDG\_BoundaryData}}{}{}{\AddDoc{TransportDG\_BoundaryData}}{Record to set BOUNDARY fields of the equation 'TransportDG'.
The fields are set only on the domain specified by one of the keys: 'region', 'rid', 'r\_set'
and after the time given by the key 'time'. The field setting can be overridden by
 any TransportDG\_BoundaryData record that comes later in the boundary data array.}
\KeyItem{\hyperB{TransportDG-BoundaryData::r-set}{r\_set}}{String (generic)}{\textless\it optional\textgreater}{\AddDoc{TransportDG\_BoundaryData::r\_set}}{Name of region set where to set fields.}
\KeyItem{\hyperB{TransportDG-BoundaryData::region}{region}}{String (generic)}{\textless\it optional\textgreater}{\AddDoc{TransportDG\_BoundaryData::region}}{Label of the region where to set fields. }
\KeyItem{\hyperB{TransportDG-BoundaryData::rid}{rid}}{Integer [0, ]}{\textless\it optional\textgreater}{\AddDoc{TransportDG\_BoundaryData::rid}}{ID of the region where to set fields.}
\KeyItem{\hyperB{TransportDG-BoundaryData::time}{time}}{Double [0, ]}{0.0}{\AddDoc{TransportDG\_BoundaryData::time}}{Apply field setting in this record after this time.
These times have to form an increasing sequence.}
\KeyItem{\hyperB{TransportDG-BoundaryData::bc-conc}{bc\_conc}}{abstract type: \hyperlink{IT::Field:R3 - Real[n]}{Field:R3 -> Real[n]}}{\textless\it optional\textgreater}{\AddDoc{TransportDG\_BoundaryData::bc\_conc}}{Boundary conditions for concentrations.}
\KeyItem{\hyperB{TransportDG-BoundaryData::old-boundary-file}{old\_boundary\_file}}{input file name}{\textless\it optional\textgreater}{\AddDoc{TransportDG\_BoundaryData::old\_boundary\_file}}{Input file with boundary conditions (obsolete).}
\KeyItem{\hyperB{TransportDG-BoundaryData::bc-times}{bc\_times}}{Array  of Double }{\textless\it optional\textgreater}{\AddDoc{TransportDG\_BoundaryData::bc\_times}}{Times for changing the boundary conditions (obsolete).}
\end{RecordType}

\begin{RecordType}{\HTRaised{IT::TransportDG-BulkData}{TransportDG\_BulkData}}{}{}{\AddDoc{TransportDG\_BulkData}}{Record to set BULK fields of the equation 'TransportDG'.
The fields are set only on the domain specified by one of the keys: 'region', 'rid', 'r\_set'
and after the time given by the key 'time'. The field setting can be overridden by
 any TransportDG\_BulkData record that comes later in the bulk data array.}
\KeyItem{\hyperB{TransportDG-BulkData::r-set}{r\_set}}{String (generic)}{\textless\it optional\textgreater}{\AddDoc{TransportDG\_BulkData::r\_set}}{Name of region set where to set fields.}
\KeyItem{\hyperB{TransportDG-BulkData::region}{region}}{String (generic)}{\textless\it optional\textgreater}{\AddDoc{TransportDG\_BulkData::region}}{Label of the region where to set fields. }
\KeyItem{\hyperB{TransportDG-BulkData::rid}{rid}}{Integer [0, ]}{\textless\it optional\textgreater}{\AddDoc{TransportDG\_BulkData::rid}}{ID of the region where to set fields.}
\KeyItem{\hyperB{TransportDG-BulkData::time}{time}}{Double [0, ]}{0.0}{\AddDoc{TransportDG\_BulkData::time}}{Apply field setting in this record after this time.
These times have to form an increasing sequence.}
\KeyItem{\hyperB{TransportDG-BulkData::init-conc}{init\_conc}}{abstract type: \hyperlink{IT::Field:R3 - Real[n]}{Field:R3 -> Real[n]}}{\textless\it optional\textgreater}{\AddDoc{TransportDG\_BulkData::init\_conc}}{Initial concentrations.}
\KeyItem{\hyperB{TransportDG-BulkData::por-m}{por\_m}}{abstract type: \hyperlink{IT::Field:R3 - Real}{Field:R3 -> Real}}{\textless\it optional\textgreater}{\AddDoc{TransportDG\_BulkData::por\_m}}{Mobile porosity}
\KeyItem{\hyperB{TransportDG-BulkData::sources-density}{sources\_density}}{abstract type: \hyperlink{IT::Field:R3 - Real[n]}{Field:R3 -> Real[n]}}{\textless\it optional\textgreater}{\AddDoc{TransportDG\_BulkData::sources\_density}}{Density of concentration sources.}
\KeyItem{\hyperB{TransportDG-BulkData::sources-sigma}{sources\_sigma}}{abstract type: \hyperlink{IT::Field:R3 - Real[n]}{Field:R3 -> Real[n]}}{\textless\it optional\textgreater}{\AddDoc{TransportDG\_BulkData::sources\_sigma}}{Concentration flux.}
\KeyItem{\hyperB{TransportDG-BulkData::sources-conc}{sources\_conc}}{abstract type: \hyperlink{IT::Field:R3 - Real[n]}{Field:R3 -> Real[n]}}{\textless\it optional\textgreater}{\AddDoc{TransportDG\_BulkData::sources\_conc}}{Concentration sources threshold.}
\KeyItem{\hyperB{TransportDG-BulkData::disp-l}{disp\_l}}{abstract type: \hyperlink{IT::Field:R3 - Real[n]}{Field:R3 -> Real[n]}}{\textless\it optional\textgreater}{\AddDoc{TransportDG\_BulkData::disp\_l}}{Longitudal dispersivity (for each substance).}
\KeyItem{\hyperB{TransportDG-BulkData::disp-t}{disp\_t}}{abstract type: \hyperlink{IT::Field:R3 - Real[n]}{Field:R3 -> Real[n]}}{\textless\it optional\textgreater}{\AddDoc{TransportDG\_BulkData::disp\_t}}{Transversal dispersivity (for each substance).}
\KeyItem{\hyperB{TransportDG-BulkData::diff-m}{diff\_m}}{abstract type: \hyperlink{IT::Field:R3 - Real[n]}{Field:R3 -> Real[n]}}{\textless\it optional\textgreater}{\AddDoc{TransportDG\_BulkData::diff\_m}}{Molecular diffusivity (for each substance).}
\KeyItem{\hyperB{TransportDG-BulkData::sigma-c}{sigma\_c}}{abstract type: \hyperlink{IT::Field:R3 - Real[n]}{Field:R3 -> Real[n]}}{\textless\it optional\textgreater}{\AddDoc{TransportDG\_BulkData::sigma\_c}}{Coefficient of diffusive transfer through fractures (for each substance).}
\KeyItem{\hyperB{TransportDG-BulkData::dg-penalty}{dg\_penalty}}{abstract type: \hyperlink{IT::Field:R3 - Real[n]}{Field:R3 -> Real[n]}}{\textless\it optional\textgreater}{\AddDoc{TransportDG\_BulkData::dg\_penalty}}{Penalty parameter influencing the discontinuity of the solution (for each substance). Its default value 1 is sufficient in most cases. Higher value diminishes the inter-element jumps.}
\end{RecordType}

\begin{SelectionType}{\HTRaised{IT::DG-variant}{DG\_variant}}{Type of penalty term.}
\KeyItem{non-symmetric}{non-symmetric weighted interior penalty DG method}
\KeyItem{incomplete}{incomplete weighted interior penalty DG method}
\KeyItem{symmetric}{symmetric weighted interior penalty DG method}
\end{SelectionType}
