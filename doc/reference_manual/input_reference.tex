\begin{TupleType}
	{IT::IndependentValue-7}
	{IndependentValue}
	{} % implements
	{} % reducible to key
	{{{Value of Field for independent variable.}}}
		\RecKey
			{IndependentValue-7::t}
			{t}
			{{Double [0, +inf)}}
			{ \it{Obligatory} }
			{{{Independent variable of stamp.}}}
		\RecKey
			{IndependentValue-7::value}
			{value}
			{{Parameter}}
			{ \it{Obligatory} }
			{{{Value of the field in given stamp.}}}
\end{TupleType}
\begin{TupleType}
	{IT::IndependentValue-8}
	{IndependentValue}
	{} % implements
	{} % reducible to key
	{{{Value of Field for independent variable.}}}
		\RecKey
			{IndependentValue-8::t}
			{t}
			{{Double [0, +inf)}}
			{ \it{Obligatory} }
			{{{Independent variable of stamp.}}}
		\RecKey
			{IndependentValue-8::value}
			{value}
			{{Array [1, UINT] of }{Array [1, UINT] of }{Parameter}}
			{ \it{Obligatory} }
			{{{Value of the field in given stamp.}}}
\end{TupleType}
\begin{RecordType}
	{IT::Balance}
	{Balance}
	{} % implements
	{} % reducible to key
	{{{Balance of a conservative quantity, boundary fluxes and sources.}}}
		\RecKey
			{Balance::times}
			{times}
			{{Array [0, UINT] of }{Record}{: }\Alink{IT::TimeGrid}{TimeGrid}}
			{ \it{[]} }
			{}
		\RecKey
			{Balance::add-output-times}
			{add{\_}output{\_}times}
			{{Bool}}
			{ \it{True} }
			{{{Add all output times of the balanced equation to the balance output times set. Note that this is not the time set of the output stream.}}}
		\RecKey
			{Balance::format}
			{format}
			{{Selection}{: }\Alink{IT::Balance-output-format}{Balance{\_}output{\_}format}}
			{ \it{Txt} }
			{{{Format of output file.}}}
		\RecKey
			{Balance::cumulative}
			{cumulative}
			{{Bool}}
			{ \it{False} }
			{{{Compute cumulative balance over time. If true, then balance is calculated at each computational time step, which can slow down the program.}}}
		\RecKey
			{Balance::file}
			{file}
			{{Filename}}
			{implicit value: "{File name generated from the balanced quantity: {\textless}quantity{\_}name{\textgreater}{\_}balance.*}"}
			{{{File name for output of balance.}}}
\end{RecordType}
\begin{RecordType}
	{IT::Bddc}
	{Bddc}
	{\Alink{IT::LinSys}{LinSys}} % implements
	{} % reducible to key
	{{{Solver setting.}}}
		\RecKey
			{Bddc::TYPE}
			{TYPE}
			{{String}}
			{ \it{Bddc} }
			{{{Sub-record Selection.}}}
		\RecKey
			{Bddc::r-tol}
			{r{\_}tol}
			{{Double [0, 1]}}
			{implicit value: "{Defalut value set by nonlinear solver or equation. If not we use value 1.0e-7.}"}
			{{{Relative residual tolerance,  (to initial error).}}}
		\RecKey
			{Bddc::max-it}
			{max{\_}it}
			{{Integer [0, INT]}}
			{implicit value: "{Defalut value set by nonlinear solver or equation. If not we use value 1000.}"}
			{{{Maximum number of outer iterations of the linear solver.}}}
		\RecKey
			{Bddc::max-nondecr-it}
			{max{\_}nondecr{\_}it}
			{{Integer [0, INT]}}
			{ \it{30} }
			{{{Maximum number of iterations of the linear solver with non-decreasing residual.}}}
		\RecKey
			{Bddc::number-of-levels}
			{number{\_}of{\_}levels}
			{{Integer [0, INT]}}
			{ \it{2} }
			{{{Number of levels in the multilevel method (=2 for the standard BDDC).}}}
		\RecKey
			{Bddc::use-adaptive-bddc}
			{use{\_}adaptive{\_}bddc}
			{{Bool}}
			{ \it{False} }
			{{{Use adaptive selection of constraints in BDDCML.}}}
		\RecKey
			{Bddc::bddcml-verbosity-level}
			{bddcml{\_}verbosity{\_}level}
			{{Integer [0, 2]}}
			{ \it{0} }
			{{{{Level of verbosity of the BDDCML library:}
}
\begin{itemize}
\item {0 - no output}
\item {1 - mild output}
\item {2 - detailed output.}
\end{itemize}
}}
\end{RecordType}
\begin{RecordType}
	{IT::Coupling-OperatorSplitting}
	{Coupling{\_}OperatorSplitting}
	{\Alink{IT::AdvectionProcess}{AdvectionProcess}} % implements
	{} % reducible to key
	{{{Transport by convection and/or diffusion}\\{
coupled with reaction and adsorption model (ODE per element)}\\{
 via operator splitting.}}}
		\RecKey
			{Coupling-OperatorSplitting::TYPE}
			{TYPE}
			{{String}}
			{ \it{Coupling{\_}operatorsplitting} }
			{{{Sub-record Selection.}}}
		\RecKey
			{Coupling-OperatorSplitting::time}
			{time}
			{{Record}{: }\Alink{IT::TimeGovernor}{TimeGovernor}}
			{ \it{Obligatory} }
			{{{Time governor setting for the secondary equation.}}}
		\RecKey
			{Coupling-OperatorSplitting::balance}
			{balance}
			{{Record}{: }\Alink{IT::Balance}{Balance}}
			{ \it{{\{}{\}}} }
			{{{Settings for computing balance.}}}
		\RecKey
			{Coupling-OperatorSplitting::output-stream}
			{output{\_}stream}
			{{Record}{: }\Alink{IT::OutputStream}{OutputStream}}
			{ \it{Obligatory} }
			{{{Parameters of output stream.}}}
		\RecKey
			{Coupling-OperatorSplitting::substances}
			{substances}
			{{Array [1, UINT] of }{Record}{: }\Alink{IT::Substance}{Substance}}
			{ \it{Obligatory} }
			{{{Specification of transported substances.}}}
		\RecKey
			{Coupling-OperatorSplitting::transport}
			{transport}
			{{Abstract}{: }\Alink{IT::Solute}{Solute}}
			{ \it{Obligatory} }
			{{{Type of numerical method for solute transport.}}}
		\RecKey
			{Coupling-OperatorSplitting::reaction-term}
			{reaction{\_}term}
			{{Abstract}{: }\Alink{IT::ReactionTerm}{ReactionTerm}}
			{ \it{Optional} }
			{{{Reaction model involved in transport.}}}
\end{RecordType}
\begin{RecordType}
	{IT::Coupling-Sequential}
	{Coupling{\_}Sequential}
	{\Alink{IT::Coupling-Base}{Coupling{\_}Base}} % implements
	{} % reducible to key
	{{{Record with data for a general sequential coupling.}}}
		\RecKey
			{Coupling-Sequential::TYPE}
			{TYPE}
			{{String}}
			{ \it{Coupling{\_}sequential} }
			{{{Sub-record Selection.}}}
		\RecKey
			{Coupling-Sequential::description}
			{description}
			{{String}}
			{ \it{Optional} }
			{{{Short description of the solved problem.}\\{
Is displayed in the main log, and possibly in other text output files.}}}
		\RecKey
			{Coupling-Sequential::mesh}
			{mesh}
			{{Record}{: }\Alink{IT::Mesh}{Mesh}}
			{ \it{Obligatory} }
			{{{Computational mesh common to all equations.}}}
		\RecKey
			{Coupling-Sequential::time}
			{time}
			{{Record}{: }\Alink{IT::TimeGovernor}{TimeGovernor}}
			{ \it{Optional} }
			{{{Simulation time frame and time step.}}}
		\RecKey
			{Coupling-Sequential::flow-equation}
			{flow{\_}equation}
			{{Abstract}{: }\Alink{IT::DarcyFlow}{DarcyFlow}}
			{ \it{Obligatory} }
			{{{Flow equation, provides the velocity field as a result.}}}
		\RecKey
			{Coupling-Sequential::solute-equation}
			{solute{\_}equation}
			{{Abstract}{: }\Alink{IT::AdvectionProcess}{AdvectionProcess}}
			{ \it{Optional} }
			{{{Transport of soluted substances, depends on the velocity field from a Flow equation.}}}
		\RecKey
			{Coupling-Sequential::heat-equation}
			{heat{\_}equation}
			{{Abstract}{: }\Alink{IT::AdvectionProcess}{AdvectionProcess}}
			{ \it{Optional} }
			{{{Heat transfer, depends on the velocity field from a Flow equation.}}}
\end{RecordType}
\begin{RecordType}
	{IT::Decay}
	{Decay}
	{} % implements
	{} % reducible to key
	{{{A model of a radioactive decay.}}}
		\RecKey
			{Decay::radionuclide}
			{radionuclide}
			{{String}}
			{ \it{Obligatory} }
			{{{The name of the parent radionuclide.}}}
		\RecKey
			{Decay::half-life}
			{half{\_}life}
			{{Double [0, +inf)}}
			{ \it{Obligatory} }
			{{{The half life of the parent radionuclide in seconds.}}}
		\RecKey
			{Decay::products}
			{products}
			{{Array [1, UINT] of }{Record}{: }\Alink{IT::RadioactiveDecayProduct}{RadioactiveDecayProduct}}
			{ \it{Obligatory} }
			{{{An array of the decay products (daughters).}}}
\end{RecordType}
\begin{RecordType}
	{IT::Difference}
	{Difference}
	{\Alink{IT::Region}{Region}} % implements
	{} % reducible to key
	{{{Defines region as a difference of given pair of regions.}}}
		\RecKey
			{Difference::TYPE}
			{TYPE}
			{{String}}
			{ \it{Difference} }
			{{{Sub-record Selection.}}}
		\RecKey
			{Difference::name}
			{name}
			{{String}}
			{ \it{Obligatory} }
			{{{Label (name) of the region. Has to be unique in one mesh.}}}
		\RecKey
			{Difference::regions}
			{regions}
			{{Array [2, 2] of }{String}}
			{ \it{Obligatory} }
			{{{Defines region as a difference of given pair of regions.}}}
\end{RecordType}
\begin{RecordType}
	{IT::DualPorosity}
	{DualPorosity}
	{\Alink{IT::ReactionTerm}{ReactionTerm}} % implements
	{} % reducible to key
	{{{Dual porosity model in transport problems.}\\{
Provides computing the concentration of substances in mobile and immobile zone.}}}
		\RecKey
			{DualPorosity::TYPE}
			{TYPE}
			{{String}}
			{ \it{Dualporosity} }
			{{{Sub-record Selection.}}}
		\RecKey
			{DualPorosity::input-fields}
			{input{\_}fields}
			{{Array [0, UINT] of }{Record}{: }\Alink{IT::DualPorosity-Data}{DualPorosity{\_}Data}}
			{ \it{Obligatory} }
			{{{Containes region specific data necessary to construct dual porosity model.}}}
		\RecKey
			{DualPorosity::scheme-tolerance}
			{scheme{\_}tolerance}
			{{Double [0, +inf)}}
			{ \it{0.001} }
			{{{Tolerance according to which the explicit Euler scheme is used or not.Set 0.0 to use analytic formula only (can be slower).}}}
		\RecKey
			{DualPorosity::reaction-mobile}
			{reaction{\_}mobile}
			{{Abstract}{: }\Alink{IT::ReactionTermMobile}{ReactionTermMobile}}
			{ \it{Optional} }
			{{{Reaction model in mobile zone.}}}
		\RecKey
			{DualPorosity::reaction-immobile}
			{reaction{\_}immobile}
			{{Abstract}{: }\Alink{IT::ReactionTermImmobile}{ReactionTermImmobile}}
			{ \it{Optional} }
			{{{Reaction model in immobile zone.}}}
		\RecKey
			{DualPorosity::output}
			{output}
			{{Record}{: }\Alink{IT::EquationOutput-7}{EquationOutput}}
			{ \it{{\{}u'fields': [u'conc{\_}immobile']{\}}} }
			{{{Setting of the fields output.}}}
\end{RecordType}
\begin{RecordType}
	{IT::DualPorosity-Data}
	{DualPorosity{\_}Data}
	{} % implements
	{} % reducible to key
	{{{Record to set fields of the equation.}\\{
The fields are set only on the domain specified by one of the keys: 'region', 'rid'}\\{
and after the time given by the key 'time'. The field setting can be overridden by}\\{
 any DualPorosity{\_}Data record that comes later in the boundary data array.}}}
		\RecKey
			{DualPorosity-Data::region}
			{region}
			{{Array [1, UINT] of }{String}}
			{ \it{Optional} }
			{{{Labels of the regions where to set fields. }}}
		\RecKey
			{DualPorosity-Data::rid}
			{rid}
			{{Integer [0, INT]}}
			{ \it{Optional} }
			{{{ID of the region where to set fields.}}}
		\RecKey
			{DualPorosity-Data::time}
			{time}
			{{Double [0, +inf)}}
			{ \it{0.0} }
			{{{Apply field setting in this record after this time.}\\{
These times have to form an increasing sequence.}}}
		\RecKey
			{DualPorosity-Data::diffusion-rate-immobile}
			{diffusion{\_}rate{\_}immobile}
			{{Array [1, UINT] of }{Abstract}{: }\Alink{IT::Field-R3---R}{Field:R3 -{\textgreater} R}}
			{ \it{Optional} }
			{{{Diffusion coefficient of non-equilibrium linear exchange between mobile and immobile zone. }{$[s^{-1}]$}}}
		\RecKey
			{DualPorosity-Data::porosity-immobile}
			{porosity{\_}immobile}
			{{Abstract}{: }\Alink{IT::Field-R3---R}{Field:R3 -{\textgreater} R}}
			{ \it{Optional} }
			{{{Porosity of the immobile zone. }{$[-]$}}}
		\RecKey
			{DualPorosity-Data::init-conc-immobile}
			{init{\_}conc{\_}immobile}
			{{Array [1, UINT] of }{Abstract}{: }\Alink{IT::Field-R3---R}{Field:R3 -{\textgreater} R}}
			{ \it{Optional} }
			{{{Initial concentration of substances in the immobile zone. }{$[m^{-3}kg]$}}}
\end{RecordType}
\begin{RecordType}
	{IT::EmptyRecord}
	{EmptyRecord}
	{} % implements
	{} % reducible to key
	{}
\end{RecordType}
\begin{RecordType}
	{IT::EquationOutput-9}
	{EquationOutput}
	{} % implements
	{} % reducible to key
	{{{Output of the equation's fields.The output is done through the output stream of the associated balance law equation.The stream defines output format for the full space information in selected times and observe points for the full time information. The key 'fields' select the fields for the full spatial output.The set of output times may be specified  per field otherwise common time set 'times' is used. If even this is not providedthe time set of the output{\_}stream is used. The initial time of the equation is automatically added to the time set of every selected field. The end time of the equation is automatically added to the common output time set.}}}
		\RecKey
			{EquationOutput-9::times}
			{times}
			{{Array [0, UINT] of }{Record}{: }\Alink{IT::TimeGrid}{TimeGrid}}
			{ \it{Optional} }
			{{{Output times used for the output fields without is own time series specification.}}}
		\RecKey
			{EquationOutput-9::add-input-times}
			{add{\_}input{\_}times}
			{{Bool}}
			{ \it{False} }
			{{{Add all input time points of the equation, mentioned in the 'input{\_}fields' list, also as the output points.}}}
		\RecKey
			{EquationOutput-9::fields}
			{fields}
			{{Array [0, UINT] of }{Record}{: }\Alink{IT::FieldOutputSetting-9}{FieldOutputSetting}}
			{ \it{[]} }
			{{{Array of output fields and their individual output settings.}}}
		\RecKey
			{EquationOutput-9::observe-fields}
			{observe{\_}fields}
			{{Array [0, UINT] of }{Parameter}}
			{ \it{[]} }
			{{{Array of the fields evaluated in the observe points of the associated output stream.}}}
\end{RecordType}
\begin{RecordType}
	{IT::FieldConstant-7}
	{FieldConstant}
	{\Alink{IT::Field-R3---R-6}{Field:R3 -{\textgreater} R}} % implements
	{\Alink{FieldConstant-7::value}{value}} % reducible to key
	{{{R3 -{\textgreater} R Field constant in space.}}}
		\RecKey
			{FieldConstant-7::TYPE}
			{TYPE}
			{{String}}
			{ \it{Fieldconstant} }
			{{{Sub-record Selection.}}}
		\RecKey
			{FieldConstant-7::unit}
			{unit}
			{{Record}{: }\Alink{IT::Unit}{Unit}}
			{ \it{Optional} }
			{{{Unit of the field values provided in the main input file, in the external file, orby a function (FieldPython).}}}
		\RecKey
			{FieldConstant-7::value}
			{value}
			{{Parameter}}
			{ \it{Obligatory} }
			{{{{Value of the constant field.}\\{
For vector values, you can use scalar value to enter constant vector.}\\{
For square }{$N\times N$}{-matrix values, you can use:}\\{
 - vector of size }{$N$}{ to enter diagonal matrix}
}
\begin{itemize}
\item {vector of size }{$\frac12N(N+1)$}{ to enter symmetric matrix (upper triangle, row by row)}
\item {scalar to enter multiple of the unit matrix.}
\end{itemize}
}}
\end{RecordType}
\begin{RecordType}
	{IT::FieldConstant-8}
	{FieldConstant}
	{\Alink{IT::Field-R3---R-3-3--2}{Field:R3 -{\textgreater} R[3,3]}} % implements
	{\Alink{FieldConstant-8::value}{value}} % reducible to key
	{{{R3 -{\textgreater} R[3,3] Field constant in space.}}}
		\RecKey
			{FieldConstant-8::TYPE}
			{TYPE}
			{{String}}
			{ \it{Fieldconstant} }
			{{{Sub-record Selection.}}}
		\RecKey
			{FieldConstant-8::unit}
			{unit}
			{{Record}{: }\Alink{IT::Unit}{Unit}}
			{ \it{Optional} }
			{{{Unit of the field values provided in the main input file, in the external file, orby a function (FieldPython).}}}
		\RecKey
			{FieldConstant-8::value}
			{value}
			{{Array [1, UINT] of }{Array [1, UINT] of }{Parameter}}
			{ \it{Obligatory} }
			{{{{Value of the constant field.}\\{
For vector values, you can use scalar value to enter constant vector.}\\{
For square }{$N\times N$}{-matrix values, you can use:}\\{
 - vector of size }{$N$}{ to enter diagonal matrix}
}
\begin{itemize}
\item {vector of size }{$\frac12N(N+1)$}{ to enter symmetric matrix (upper triangle, row by row)}
\item {scalar to enter multiple of the unit matrix.}
\end{itemize}
}}
\end{RecordType}
\begin{RecordType}
	{IT::FieldElementwise}
	{FieldElementwise}
	{\Alink{IT::Field-R3---R-3-3--2}{Field:R3 -{\textgreater} R[3,3]}\Alink{IT::Field-R3---R-3-3-}{Field:R3 -{\textgreater} R[3,3]}} % implements
	{} % reducible to key
	{{{R3 -{\textgreater} R[3,3] Field constant in space.}}}
		\RecKey
			{FieldElementwise::TYPE}
			{TYPE}
			{{String}}
			{ \it{Fieldelementwise} }
			{{{Sub-record Selection.}}}
		\RecKey
			{FieldElementwise::unit}
			{unit}
			{{Record}{: }\Alink{IT::Unit}{Unit}}
			{ \it{Optional} }
			{{{Unit of the field values provided in the main input file, in the external file, orby a function (FieldPython).}}}
		\RecKey
			{FieldElementwise::gmsh-file}
			{gmsh{\_}file}
			{{Filename}}
			{ \it{Obligatory} }
			{{{Input file with ASCII GMSH file format.}}}
		\RecKey
			{FieldElementwise::field-name}
			{field{\_}name}
			{{String}}
			{ \it{Obligatory} }
			{{{The values of the Field are read from the }\ttfamily {\$}ElementData}}
\end{RecordType}
\begin{RecordType}
	{IT::FieldElementwise-2}
	{FieldElementwise}
	{\Alink{IT::Field-R3---R-6}{Field:R3 -{\textgreater} R}\Alink{IT::Field-R3---R}{Field:R3 -{\textgreater} R}\Alink{IT::Field-R3---R-2}{Field:R3 -{\textgreater} R}\Alink{IT::Field-R3---R-3}{Field:R3 -{\textgreater} R}\Alink{IT::Field-R3---R-4}{Field:R3 -{\textgreater} R}\Alink{IT::Field-R3---R-5}{Field:R3 -{\textgreater} R}} % implements
	{} % reducible to key
	{{{R3 -{\textgreater} R Field constant in space.}}}
		\RecKey
			{FieldElementwise-2::TYPE}
			{TYPE}
			{{String}}
			{ \it{Fieldelementwise} }
			{{{Sub-record Selection.}}}
		\RecKey
			{FieldElementwise-2::unit}
			{unit}
			{{Record}{: }\Alink{IT::Unit}{Unit}}
			{ \it{Optional} }
			{{{Unit of the field values provided in the main input file, in the external file, orby a function (FieldPython).}}}
		\RecKey
			{FieldElementwise-2::gmsh-file}
			{gmsh{\_}file}
			{{Filename}}
			{ \it{Obligatory} }
			{{{Input file with ASCII GMSH file format.}}}
		\RecKey
			{FieldElementwise-2::field-name}
			{field{\_}name}
			{{String}}
			{ \it{Obligatory} }
			{{{The values of the Field are read from the }\ttfamily {\$}ElementData}}
\end{RecordType}
\begin{RecordType}
	{IT::FieldFormula}
	{FieldFormula}
	{\Alink{IT::Field-R3---R-3-3--2}{Field:R3 -{\textgreater} R[3,3]}\Alink{IT::Field-R3---R-3-3-}{Field:R3 -{\textgreater} R[3,3]}} % implements
	{\Alink{FieldFormula::value}{value}} % reducible to key
	{{{R3 -{\textgreater} R[3,3] Field given by runtime interpreted formula.}}}
		\RecKey
			{FieldFormula::TYPE}
			{TYPE}
			{{String}}
			{ \it{Fieldformula} }
			{{{Sub-record Selection.}}}
		\RecKey
			{FieldFormula::unit}
			{unit}
			{{Record}{: }\Alink{IT::Unit}{Unit}}
			{ \it{Optional} }
			{{{Unit of the field values provided in the main input file, in the external file, orby a function (FieldPython).}}}
		\RecKey
			{FieldFormula::value}
			{value}
			{{Array [1, UINT] of }{Array [1, UINT] of }{String}}
			{ \it{Obligatory} }
			{{{{String, array of strings, or matrix of strings with formulas for individual entries of scalar, vector, or tensor value respectively.}\\{
For vector values, you can use just one string to enter homogeneous vector.}\\{
For square }{$N\times N$}{-matrix values, you can use:}
}
\begin{itemize}
\item {array of strings of size }{$N$}{ to enter diagonal matrix}
\item {array of strings of size }{$\frac12N(N+1)$}{ to enter symmetric matrix (upper triangle, row by row)}
\item {just one string to enter (spatially variable) multiple of the unit matrix.}\\{
Formula can contain variables }\ttfamily x,y,z,t
\end{itemize}
}}
\end{RecordType}
\begin{RecordType}
	{IT::FieldFormula-2}
	{FieldFormula}
	{\Alink{IT::Field-R3---R-6}{Field:R3 -{\textgreater} R}\Alink{IT::Field-R3---R}{Field:R3 -{\textgreater} R}\Alink{IT::Field-R3---R-2}{Field:R3 -{\textgreater} R}\Alink{IT::Field-R3---R-3}{Field:R3 -{\textgreater} R}\Alink{IT::Field-R3---R-4}{Field:R3 -{\textgreater} R}\Alink{IT::Field-R3---R-5}{Field:R3 -{\textgreater} R}} % implements
	{\Alink{FieldFormula-2::value}{value}} % reducible to key
	{{{R3 -{\textgreater} R Field given by runtime interpreted formula.}}}
		\RecKey
			{FieldFormula-2::TYPE}
			{TYPE}
			{{String}}
			{ \it{Fieldformula} }
			{{{Sub-record Selection.}}}
		\RecKey
			{FieldFormula-2::unit}
			{unit}
			{{Record}{: }\Alink{IT::Unit}{Unit}}
			{ \it{Optional} }
			{{{Unit of the field values provided in the main input file, in the external file, orby a function (FieldPython).}}}
		\RecKey
			{FieldFormula-2::value}
			{value}
			{{String}}
			{ \it{Obligatory} }
			{{{{String, array of strings, or matrix of strings with formulas for individual entries of scalar, vector, or tensor value respectively.}\\{
For vector values, you can use just one string to enter homogeneous vector.}\\{
For square }{$N\times N$}{-matrix values, you can use:}
}
\begin{itemize}
\item {array of strings of size }{$N$}{ to enter diagonal matrix}
\item {array of strings of size }{$\frac12N(N+1)$}{ to enter symmetric matrix (upper triangle, row by row)}
\item {just one string to enter (spatially variable) multiple of the unit matrix.}\\{
Formula can contain variables }\ttfamily x,y,z,t
\end{itemize}
}}
\end{RecordType}
\begin{RecordType}
	{IT::FieldInterpolatedP0}
	{FieldInterpolatedP0}
	{\Alink{IT::Field-R3---R-3-3--2}{Field:R3 -{\textgreater} R[3,3]}\Alink{IT::Field-R3---R-3-3-}{Field:R3 -{\textgreater} R[3,3]}} % implements
	{} % reducible to key
	{{{R3 -{\textgreater} R[3,3] Field constant in space.}}}
		\RecKey
			{FieldInterpolatedP0::TYPE}
			{TYPE}
			{{String}}
			{ \it{Fieldinterpolatedp0} }
			{{{Sub-record Selection.}}}
		\RecKey
			{FieldInterpolatedP0::unit}
			{unit}
			{{Record}{: }\Alink{IT::Unit}{Unit}}
			{ \it{Optional} }
			{{{Unit of the field values provided in the main input file, in the external file, orby a function (FieldPython).}}}
		\RecKey
			{FieldInterpolatedP0::gmsh-file}
			{gmsh{\_}file}
			{{Filename}}
			{ \it{Obligatory} }
			{{{Input file with ASCII GMSH file format.}}}
		\RecKey
			{FieldInterpolatedP0::field-name}
			{field{\_}name}
			{{String}}
			{ \it{Obligatory} }
			{{{The values of the Field are read from the }\ttfamily {\$}ElementData}}
\end{RecordType}
\begin{RecordType}
	{IT::FieldInterpolatedP0-2}
	{FieldInterpolatedP0}
	{\Alink{IT::Field-R3---R-6}{Field:R3 -{\textgreater} R}\Alink{IT::Field-R3---R}{Field:R3 -{\textgreater} R}\Alink{IT::Field-R3---R-2}{Field:R3 -{\textgreater} R}\Alink{IT::Field-R3---R-3}{Field:R3 -{\textgreater} R}\Alink{IT::Field-R3---R-4}{Field:R3 -{\textgreater} R}\Alink{IT::Field-R3---R-5}{Field:R3 -{\textgreater} R}} % implements
	{} % reducible to key
	{{{R3 -{\textgreater} R Field constant in space.}}}
		\RecKey
			{FieldInterpolatedP0-2::TYPE}
			{TYPE}
			{{String}}
			{ \it{Fieldinterpolatedp0} }
			{{{Sub-record Selection.}}}
		\RecKey
			{FieldInterpolatedP0-2::unit}
			{unit}
			{{Record}{: }\Alink{IT::Unit}{Unit}}
			{ \it{Optional} }
			{{{Unit of the field values provided in the main input file, in the external file, orby a function (FieldPython).}}}
		\RecKey
			{FieldInterpolatedP0-2::gmsh-file}
			{gmsh{\_}file}
			{{Filename}}
			{ \it{Obligatory} }
			{{{Input file with ASCII GMSH file format.}}}
		\RecKey
			{FieldInterpolatedP0-2::field-name}
			{field{\_}name}
			{{String}}
			{ \it{Obligatory} }
			{{{The values of the Field are read from the }\ttfamily {\$}ElementData}}
\end{RecordType}
\begin{RecordType}
	{IT::FieldOutputSetting-9}
	{FieldOutputSetting}
	{} % implements
	{\Alink{FieldOutputSetting-9::field}{field}} % reducible to key
	{{{Setting of the field output. The field name, output times, output interpolation (future).}}}
		\RecKey
			{FieldOutputSetting-9::field}
			{field}
			{{Parameter}}
			{ \it{Obligatory} }
			{{{The field name (from selection).}}}
		\RecKey
			{FieldOutputSetting-9::times}
			{times}
			{{Array [0, UINT] of }{Record}{: }\Alink{IT::TimeGrid}{TimeGrid}}
			{ \it{Optional} }
			{{{Output times specific to particular field.}}}
\end{RecordType}
\begin{RecordType}
	{IT::FieldPython}
	{FieldPython}
	{\Alink{IT::Field-R3---R-3-3--2}{Field:R3 -{\textgreater} R[3,3]}\Alink{IT::Field-R3---R-3-3-}{Field:R3 -{\textgreater} R[3,3]}} % implements
	{} % reducible to key
	{{{R3 -{\textgreater} R[3,3] Field given by a Python script.}}}
		\RecKey
			{FieldPython::TYPE}
			{TYPE}
			{{String}}
			{ \it{Fieldpython} }
			{{{Sub-record Selection.}}}
		\RecKey
			{FieldPython::unit}
			{unit}
			{{Record}{: }\Alink{IT::Unit}{Unit}}
			{ \it{Optional} }
			{{{Unit of the field values provided in the main input file, in the external file, orby a function (FieldPython).}}}
		\RecKey
			{FieldPython::script-string}
			{script{\_}string}
			{{String}}
			{implicit value: "{Obligatory if 'script{\_}file' is not given.}"}
			{{{Python script given as in place string}}}
		\RecKey
			{FieldPython::script-file}
			{script{\_}file}
			{{Filename}}
			{implicit value: "{Obligatory if 'script{\_}striong' is not given.}"}
			{{{Python script given as external file}}}
		\RecKey
			{FieldPython::function}
			{function}
			{{String}}
			{ \it{Obligatory} }
			{{{Function in the given script that returns tuple containing components of the return type.}\\{
For NxM tensor values: tensor(row,col) = tuple( M*row + col ).}}}
\end{RecordType}
\begin{RecordType}
	{IT::FieldPython-2}
	{FieldPython}
	{\Alink{IT::Field-R3---R-6}{Field:R3 -{\textgreater} R}\Alink{IT::Field-R3---R}{Field:R3 -{\textgreater} R}\Alink{IT::Field-R3---R-2}{Field:R3 -{\textgreater} R}\Alink{IT::Field-R3---R-3}{Field:R3 -{\textgreater} R}\Alink{IT::Field-R3---R-4}{Field:R3 -{\textgreater} R}\Alink{IT::Field-R3---R-5}{Field:R3 -{\textgreater} R}} % implements
	{} % reducible to key
	{{{R3 -{\textgreater} R Field given by a Python script.}}}
		\RecKey
			{FieldPython-2::TYPE}
			{TYPE}
			{{String}}
			{ \it{Fieldpython} }
			{{{Sub-record Selection.}}}
		\RecKey
			{FieldPython-2::unit}
			{unit}
			{{Record}{: }\Alink{IT::Unit}{Unit}}
			{ \it{Optional} }
			{{{Unit of the field values provided in the main input file, in the external file, orby a function (FieldPython).}}}
		\RecKey
			{FieldPython-2::script-string}
			{script{\_}string}
			{{String}}
			{implicit value: "{Obligatory if 'script{\_}file' is not given.}"}
			{{{Python script given as in place string}}}
		\RecKey
			{FieldPython-2::script-file}
			{script{\_}file}
			{{Filename}}
			{implicit value: "{Obligatory if 'script{\_}striong' is not given.}"}
			{{{Python script given as external file}}}
		\RecKey
			{FieldPython-2::function}
			{function}
			{{String}}
			{ \it{Obligatory} }
			{{{Function in the given script that returns tuple containing components of the return type.}\\{
For NxM tensor values: tensor(row,col) = tuple( M*row + col ).}}}
\end{RecordType}
\begin{RecordType}
	{IT::FieldTimeFunction-7}
	{FieldTimeFunction}
	{\Alink{IT::Field-R3---R-6}{Field:R3 -{\textgreater} R}} % implements
	{\Alink{FieldTimeFunction-7::time-function}{time{\_}function}} % reducible to key
	{{{R3 -{\textgreater} R Field time-dependent function in space.}}}
		\RecKey
			{FieldTimeFunction-7::TYPE}
			{TYPE}
			{{String}}
			{ \it{Fieldtimefunction} }
			{{{Sub-record Selection.}}}
		\RecKey
			{FieldTimeFunction-7::unit}
			{unit}
			{{Record}{: }\Alink{IT::Unit}{Unit}}
			{ \it{Optional} }
			{{{Unit of the field values provided in the main input file, in the external file, orby a function (FieldPython).}}}
		\RecKey
			{FieldTimeFunction-7::time-function}
			{time{\_}function}
			{{Record}{: }\Alink{IT::TableFunction-7}{TableFunction}}
			{ \it{Obligatory} }
			{{{Values of time series initialization of Field.}}}
\end{RecordType}
\begin{RecordType}
	{IT::FieldTimeFunction-8}
	{FieldTimeFunction}
	{\Alink{IT::Field-R3---R-3-3--2}{Field:R3 -{\textgreater} R[3,3]}} % implements
	{\Alink{FieldTimeFunction-8::time-function}{time{\_}function}} % reducible to key
	{{{R3 -{\textgreater} R[3,3] Field time-dependent function in space.}}}
		\RecKey
			{FieldTimeFunction-8::TYPE}
			{TYPE}
			{{String}}
			{ \it{Fieldtimefunction} }
			{{{Sub-record Selection.}}}
		\RecKey
			{FieldTimeFunction-8::unit}
			{unit}
			{{Record}{: }\Alink{IT::Unit}{Unit}}
			{ \it{Optional} }
			{{{Unit of the field values provided in the main input file, in the external file, orby a function (FieldPython).}}}
		\RecKey
			{FieldTimeFunction-8::time-function}
			{time{\_}function}
			{{Record}{: }\Alink{IT::TableFunction-8}{TableFunction}}
			{ \it{Obligatory} }
			{{{Values of time series initialization of Field.}}}
\end{RecordType}
\begin{RecordType}
	{IT::FirstOrderReaction}
	{FirstOrderReaction}
	{\Alink{IT::ReactionTerm}{ReactionTerm}\Alink{IT::GenericReaction}{GenericReaction}\Alink{IT::ReactionTermMobile}{ReactionTermMobile}\Alink{IT::ReactionTermImmobile}{ReactionTermImmobile}} % implements
	{} % reducible to key
	{{{A model of first order chemical reactions (decompositions of a reactant into products).}}}
		\RecKey
			{FirstOrderReaction::TYPE}
			{TYPE}
			{{String}}
			{ \it{Firstorderreaction} }
			{{{Sub-record Selection.}}}
		\RecKey
			{FirstOrderReaction::reactions}
			{reactions}
			{{Array [0, UINT] of }{Record}{: }\Alink{IT::Reaction}{Reaction}}
			{ \it{Obligatory} }
			{{{An array of first order chemical reactions.}}}
		\RecKey
			{FirstOrderReaction::ode-solver}
			{ode{\_}solver}
			{{Record}{: }\Alink{IT::PadeApproximant}{PadeApproximant}}
			{ \it{{\{}{\}}} }
			{{{Numerical solver for the system of first order ordinary differential equations coming from the model.}}}
\end{RecordType}
\begin{RecordType}
	{IT::FirstOrderReactionProduct}
	{FirstOrderReactionProduct}
	{} % implements
	{\Alink{FirstOrderReactionProduct::name}{name}} % reducible to key
	{{{A record describing a product of a reaction.}}}
		\RecKey
			{FirstOrderReactionProduct::name}
			{name}
			{{String}}
			{ \it{Obligatory} }
			{{{The name of the product.}}}
		\RecKey
			{FirstOrderReactionProduct::branching-ratio}
			{branching{\_}ratio}
			{{Double [0, +inf)}}
			{ \it{1.0} }
			{{{The branching ratio of the product when there are more products.}\\{
The value must be positive. Further, the branching ratios of all products are normalized in order to sum to one.}\\{
The default value 1.0, should only be used in the case of single product.}}}
\end{RecordType}
\begin{RecordType}
	{IT::FirstOrderReactionReactant}
	{FirstOrderReactionReactant}
	{} % implements
	{\Alink{FirstOrderReactionReactant::name}{name}} % reducible to key
	{{{A record describing a reactant of a reaction.}}}
		\RecKey
			{FirstOrderReactionReactant::name}
			{name}
			{{String}}
			{ \it{Obligatory} }
			{{{The name of the reactant.}}}
\end{RecordType}
\begin{RecordType}
	{IT::Flow-Darcy-MH}
	{Flow{\_}Darcy{\_}MH}
	{\Alink{IT::DarcyFlow}{DarcyFlow}} % implements
	{} % reducible to key
	{{{Mixed-Hybrid  solver for STEADY saturated Darcy flow.}}}
		\RecKey
			{Flow-Darcy-MH::TYPE}
			{TYPE}
			{{String}}
			{ \it{Flow{\_}darcy{\_}mh} }
			{{{Sub-record Selection.}}}
		\RecKey
			{Flow-Darcy-MH::gravity}
			{gravity}
			{{Array [3, 3] of }{Double (-inf, +inf)}}
			{ \it{[0, 0, -1]} }
			{{{Vector of the gravitational acceleration (divided by the acceleration). Dimensionless, magnitude one for the Earth conditions.}}}
		\RecKey
			{Flow-Darcy-MH::input-fields}
			{input{\_}fields}
			{{Array [0, UINT] of }{Record}{: }\Alink{IT::Flow-Darcy-MH-Data}{Flow{\_}Darcy{\_}MH{\_}Data}}
			{ \it{Obligatory} }
			{{{Input data for Darcy flow model.}}}
		\RecKey
			{Flow-Darcy-MH::nonlinear-solver}
			{nonlinear{\_}solver}
			{{Record}{: }\Alink{IT::NonlinearSolver}{NonlinearSolver}}
			{ \it{Obligatory} }
			{{{Non-linear solver for MH problem.}}}
		\RecKey
			{Flow-Darcy-MH::output-stream}
			{output{\_}stream}
			{{Record}{: }\Alink{IT::OutputStream}{OutputStream}}
			{ \it{Obligatory} }
			{{{Parameters of output stream.}}}
		\RecKey
			{Flow-Darcy-MH::output}
			{output}
			{{Record}{: }\Alink{IT::EquationOutput}{EquationOutput}}
			{ \it{{\{}u'fields': [u'pressure{\_}p0', u'velocity{\_}p0']{\}}} }
			{{{Parameters of output from MH module.}}}
		\RecKey
			{Flow-Darcy-MH::output-specific}
			{output{\_}specific}
			{{Record}{: }\Alink{IT::Output-DarcyMHSpecific}{Output{\_}DarcyMHSpecific}}
			{ \it{Optional} }
			{{{Parameters of output form MH module.}}}
		\RecKey
			{Flow-Darcy-MH::balance}
			{balance}
			{{Record}{: }\Alink{IT::Balance}{Balance}}
			{ \it{{\{}{\}}} }
			{{{Settings for computing mass balance.}}}
		\RecKey
			{Flow-Darcy-MH::time}
			{time}
			{{Record}{: }\Alink{IT::TimeGovernor}{TimeGovernor}}
			{ \it{{\{}{\}}} }
			{{{Time governor setting for the unsteady Darcy flow model.}}}
		\RecKey
			{Flow-Darcy-MH::n-schurs}
			{n{\_}schurs}
			{{Integer [0, 2]}}
			{ \it{2} }
			{{{Number of Schur complements to perform when solving MH system.}}}
		\RecKey
			{Flow-Darcy-MH::mortar-method}
			{mortar{\_}method}
			{{Selection}{: }\Alink{IT::MH-MortarMethod}{MH{\_}MortarMethod}}
			{ \it{None} }
			{{{Method for coupling Darcy flow between dimensions.}}}
\end{RecordType}
\begin{RecordType}
	{IT::Flow-Darcy-MH-Data}
	{Flow{\_}Darcy{\_}MH{\_}Data}
	{} % implements
	{} % reducible to key
	{{{Record to set fields of the equation.}\\{
The fields are set only on the domain specified by one of the keys: 'region', 'rid'}\\{
and after the time given by the key 'time'. The field setting can be overridden by}\\{
 any Flow{\_}Darcy{\_}MH{\_}Data record that comes later in the boundary data array.}}}
		\RecKey
			{Flow-Darcy-MH-Data::region}
			{region}
			{{Array [1, UINT] of }{String}}
			{ \it{Optional} }
			{{{Labels of the regions where to set fields. }}}
		\RecKey
			{Flow-Darcy-MH-Data::rid}
			{rid}
			{{Integer [0, INT]}}
			{ \it{Optional} }
			{{{ID of the region where to set fields.}}}
		\RecKey
			{Flow-Darcy-MH-Data::time}
			{time}
			{{Double [0, +inf)}}
			{ \it{0.0} }
			{{{Apply field setting in this record after this time.}\\{
These times have to form an increasing sequence.}}}
		\RecKey
			{Flow-Darcy-MH-Data::anisotropy}
			{anisotropy}
			{{Abstract}{: }\Alink{IT::Field-R3---R-3-3-}{Field:R3 -{\textgreater} R[3,3]}}
			{ \it{Optional} }
			{{{Anisotropy of the conductivity tensor. }{$[-]$}}}
		\RecKey
			{Flow-Darcy-MH-Data::cross-section}
			{cross{\_}section}
			{{Abstract}{: }\Alink{IT::Field-R3---R}{Field:R3 -{\textgreater} R}}
			{ \it{Optional} }
			{{{Complement dimension parameter (cross section for 1D, thickness for 2D). }{$[m^{3-d}]$}}}
		\RecKey
			{Flow-Darcy-MH-Data::conductivity}
			{conductivity}
			{{Abstract}{: }\Alink{IT::Field-R3---R}{Field:R3 -{\textgreater} R}}
			{ \it{Optional} }
			{{{Isotropic conductivity scalar. }{$[ms^{-1}]$}}}
		\RecKey
			{Flow-Darcy-MH-Data::sigma}
			{sigma}
			{{Abstract}{: }\Alink{IT::Field-R3---R}{Field:R3 -{\textgreater} R}}
			{ \it{Optional} }
			{{{Transition coefficient between dimensions. }{$[-]$}}}
		\RecKey
			{Flow-Darcy-MH-Data::water-source-density}
			{water{\_}source{\_}density}
			{{Abstract}{: }\Alink{IT::Field-R3---R}{Field:R3 -{\textgreater} R}}
			{ \it{Optional} }
			{{{Water source density. }{$[s^{-1}]$}}}
		\RecKey
			{Flow-Darcy-MH-Data::bc-type}
			{bc{\_}type}
			{{Abstract}{: }\Alink{IT::Field-R3---R-2}{Field:R3 -{\textgreater} R}}
			{ \it{Optional} }
			{{{Boundary condition type, possible values: }{$[-]$}}}
		\RecKey
			{Flow-Darcy-MH-Data::bc-pressure}
			{bc{\_}pressure}
			{{Abstract}{: }\Alink{IT::Field-R3---R}{Field:R3 -{\textgreater} R}}
			{ \it{Optional} }
			{{{Prescribed pressure value on the boundary. Used for all values of 'bc{\_}type' save the bc{\_}type='none'.See documentation of 'bc{\_}type' for exact meaning of 'bc{\_}pressure' in individual boundary condition types. }{$[m]$}}}
		\RecKey
			{Flow-Darcy-MH-Data::bc-flux}
			{bc{\_}flux}
			{{Abstract}{: }\Alink{IT::Field-R3---R}{Field:R3 -{\textgreater} R}}
			{ \it{Optional} }
			{{{Incoming water boundary flux. Used for bc{\_}types : 'none', 'total{\_}flux', 'seepage', 'river'. }{$[m^{4-d}s^{-1}]$}}}
		\RecKey
			{Flow-Darcy-MH-Data::bc-robin-sigma}
			{bc{\_}robin{\_}sigma}
			{{Abstract}{: }\Alink{IT::Field-R3---R}{Field:R3 -{\textgreater} R}}
			{ \it{Optional} }
			{{{Conductivity coefficient in the 'total{\_}flux' or the 'river' boundary condition type. }{$[m^{3-d}s^{-1}]$}}}
		\RecKey
			{Flow-Darcy-MH-Data::bc-switch-pressure}
			{bc{\_}switch{\_}pressure}
			{{Abstract}{: }\Alink{IT::Field-R3---R}{Field:R3 -{\textgreater} R}}
			{ \it{Optional} }
			{{{Critical switch pressure for 'seepage' and 'river' boundary conditions. }{$[m]$}}}
		\RecKey
			{Flow-Darcy-MH-Data::init-pressure}
			{init{\_}pressure}
			{{Abstract}{: }\Alink{IT::Field-R3---R}{Field:R3 -{\textgreater} R}}
			{ \it{Optional} }
			{{{Initial condition as pressure }{$[m]$}}}
		\RecKey
			{Flow-Darcy-MH-Data::storativity}
			{storativity}
			{{Abstract}{: }\Alink{IT::Field-R3---R}{Field:R3 -{\textgreater} R}}
			{ \it{Optional} }
			{{{Storativity. }{$[m^{-1}]$}}}
		\RecKey
			{Flow-Darcy-MH-Data::bc-piezo-head}
			{bc{\_}piezo{\_}head}
			{{Abstract}{: }\Alink{IT::Field-R3---R}{Field:R3 -{\textgreater} R}}
			{ \it{Optional} }
			{{{Boundary piezometric head for BC types: dirichlet, robin, and river.}}}
		\RecKey
			{Flow-Darcy-MH-Data::bc-switch-piezo-head}
			{bc{\_}switch{\_}piezo{\_}head}
			{{Abstract}{: }\Alink{IT::Field-R3---R}{Field:R3 -{\textgreater} R}}
			{ \it{Optional} }
			{{{Boundary switch piezometric head for BC types: seepage, river.}}}
		\RecKey
			{Flow-Darcy-MH-Data::init-piezo-head}
			{init{\_}piezo{\_}head}
			{{Abstract}{: }\Alink{IT::Field-R3---R}{Field:R3 -{\textgreater} R}}
			{ \it{Optional} }
			{{{Initial condition for the pressure given as the piezometric head.}}}
\end{RecordType}
\begin{RecordType}
	{IT::Flow-Richards-LMH}
	{Flow{\_}Richards{\_}LMH}
	{\Alink{IT::DarcyFlow}{DarcyFlow}} % implements
	{} % reducible to key
	{{{Lumped Mixed-Hybrid solver for unsteady saturated Darcy flow.}}}
		\RecKey
			{Flow-Richards-LMH::TYPE}
			{TYPE}
			{{String}}
			{ \it{Flow{\_}richards{\_}lmh} }
			{{{Sub-record Selection.}}}
		\RecKey
			{Flow-Richards-LMH::gravity}
			{gravity}
			{{Array [3, 3] of }{Double (-inf, +inf)}}
			{ \it{[0, 0, -1]} }
			{{{Vector of the gravitational acceleration (divided by the acceleration). Dimensionless, magnitude one for the Earth conditions.}}}
		\RecKey
			{Flow-Richards-LMH::input-fields}
			{input{\_}fields}
			{{Array [0, UINT] of }{Record}{: }\Alink{IT::RichardsLMH-Data}{RichardsLMH{\_}Data}}
			{ \it{Obligatory} }
			{{{Input data for Darcy flow model.}}}
		\RecKey
			{Flow-Richards-LMH::nonlinear-solver}
			{nonlinear{\_}solver}
			{{Record}{: }\Alink{IT::NonlinearSolver}{NonlinearSolver}}
			{ \it{Obligatory} }
			{{{Non-linear solver for MH problem.}}}
		\RecKey
			{Flow-Richards-LMH::output-stream}
			{output{\_}stream}
			{{Record}{: }\Alink{IT::OutputStream}{OutputStream}}
			{ \it{Obligatory} }
			{{{Parameters of output stream.}}}
		\RecKey
			{Flow-Richards-LMH::output}
			{output}
			{{Record}{: }\Alink{IT::EquationOutput}{EquationOutput}}
			{ \it{{\{}u'fields': [u'pressure{\_}p0', u'velocity{\_}p0']{\}}} }
			{{{Parameters of output from MH module.}}}
		\RecKey
			{Flow-Richards-LMH::output-specific}
			{output{\_}specific}
			{{Record}{: }\Alink{IT::Output-DarcyMHSpecific}{Output{\_}DarcyMHSpecific}}
			{ \it{Optional} }
			{{{Parameters of output form MH module.}}}
		\RecKey
			{Flow-Richards-LMH::balance}
			{balance}
			{{Record}{: }\Alink{IT::Balance}{Balance}}
			{ \it{{\{}{\}}} }
			{{{Settings for computing mass balance.}}}
		\RecKey
			{Flow-Richards-LMH::time}
			{time}
			{{Record}{: }\Alink{IT::TimeGovernor}{TimeGovernor}}
			{ \it{{\{}{\}}} }
			{{{Time governor setting for the unsteady Darcy flow model.}}}
		\RecKey
			{Flow-Richards-LMH::n-schurs}
			{n{\_}schurs}
			{{Integer [0, 2]}}
			{ \it{2} }
			{{{Number of Schur complements to perform when solving MH system.}}}
		\RecKey
			{Flow-Richards-LMH::mortar-method}
			{mortar{\_}method}
			{{Selection}{: }\Alink{IT::MH-MortarMethod}{MH{\_}MortarMethod}}
			{ \it{None} }
			{{{Method for coupling Darcy flow between dimensions.}}}
		\RecKey
			{Flow-Richards-LMH::soil-model}
			{soil{\_}model}
			{{Record}{: }\Alink{IT::SoilModel}{SoilModel}}
			{ \it{Van{\_}genuchten} }
			{{{Setting for the soil model.}}}
\end{RecordType}
\begin{RecordType}
	{IT::From-Elements}
	{From{\_}Elements}
	{\Alink{IT::Region}{Region}} % implements
	{} % reducible to key
	{{{Region declared by name, ID and enum of elements.}\\{
Allows to create new region and assign elements to its.}\\{
Elements are specified by ids.}}}
		\RecKey
			{From-Elements::TYPE}
			{TYPE}
			{{String}}
			{ \it{From{\_}elements} }
			{{{Sub-record Selection.}}}
		\RecKey
			{From-Elements::name}
			{name}
			{{String}}
			{ \it{Obligatory} }
			{{{Label (name) of the region. Has to be unique in one mesh.}}}
		\RecKey
			{From-Elements::id}
			{id}
			{{Integer [0, INT]}}
			{ \it{Optional} }
			{{{The ID of the region to which you assign label.}\\{
If new region is created and ID is not set, unique ID will be generated automatically.}}}
		\RecKey
			{From-Elements::element-list}
			{element{\_}list}
			{{Array [1, UINT] of }{Integer [0, INT]}}
			{ \it{Obligatory} }
			{{{Specification of the region by the list of elements.}}}
\end{RecordType}
\begin{RecordType}
	{IT::From-Id}
	{From{\_}Id}
	{\Alink{IT::Region}{Region}} % implements
	{} % reducible to key
	{{{Region declared by id and name.}\\{
Allows to create new region with given id and label}\\{
or specify existing region by id which will be renamed.}}}
		\RecKey
			{From-Id::TYPE}
			{TYPE}
			{{String}}
			{ \it{From{\_}id} }
			{{{Sub-record Selection.}}}
		\RecKey
			{From-Id::name}
			{name}
			{{String}}
			{ \it{Obligatory} }
			{{{Label (name) of the region. Has to be unique in one mesh.}}}
		\RecKey
			{From-Id::id}
			{id}
			{{Integer [0, INT]}}
			{ \it{Obligatory} }
			{{{The ID of the region to which you assign label.}}}
		\RecKey
			{From-Id::dim}
			{dim}
			{{Integer [0, INT]}}
			{ \it{Optional} }
			{{{The dim of the region to which you assign label. Value is taken into account only if new region is created.}}}
\end{RecordType}
\begin{RecordType}
	{IT::From-Label}
	{From{\_}Label}
	{\Alink{IT::Region}{Region}} % implements
	{} % reducible to key
	{{{Allows to rename existing region specified by mesh{\_}label.}}}
		\RecKey
			{From-Label::TYPE}
			{TYPE}
			{{String}}
			{ \it{From{\_}label} }
			{{{Sub-record Selection.}}}
		\RecKey
			{From-Label::name}
			{name}
			{{String}}
			{ \it{Obligatory} }
			{{{New label (name) of the region. Has to be unique in one mesh.}}}
		\RecKey
			{From-Label::mesh-label}
			{mesh{\_}label}
			{{String}}
			{ \it{Obligatory} }
			{{{The mesh{\_}label is e.g. physical volume name in GMSH format.}}}
\end{RecordType}
\begin{RecordType}
	{IT::Heat-AdvectionDiffusion-DG}
	{Heat{\_}AdvectionDiffusion{\_}DG}
	{\Alink{IT::AdvectionProcess}{AdvectionProcess}} % implements
	{} % reducible to key
	{{{DG solver for heat transfer.}}}
		\RecKey
			{Heat-AdvectionDiffusion-DG::TYPE}
			{TYPE}
			{{String}}
			{ \it{Heat{\_}advectiondiffusion{\_}dg} }
			{{{Sub-record Selection.}}}
		\RecKey
			{Heat-AdvectionDiffusion-DG::time}
			{time}
			{{Record}{: }\Alink{IT::TimeGovernor}{TimeGovernor}}
			{ \it{Obligatory} }
			{{{Time governor setting for the secondary equation.}}}
		\RecKey
			{Heat-AdvectionDiffusion-DG::balance}
			{balance}
			{{Record}{: }\Alink{IT::Balance}{Balance}}
			{ \it{{\{}{\}}} }
			{{{Settings for computing balance.}}}
		\RecKey
			{Heat-AdvectionDiffusion-DG::output-stream}
			{output{\_}stream}
			{{Record}{: }\Alink{IT::OutputStream}{OutputStream}}
			{ \it{Obligatory} }
			{{{Parameters of output stream.}}}
		\RecKey
			{Heat-AdvectionDiffusion-DG::solver}
			{solver}
			{{Record}{: }\Alink{IT::Petsc}{Petsc}}
			{ \it{Obligatory} }
			{{{Linear solver for MH problem.}}}
		\RecKey
			{Heat-AdvectionDiffusion-DG::input-fields}
			{input{\_}fields}
			{{Array [0, UINT] of }{Record}{: }\Alink{IT::Heat-AdvectionDiffusion-DG-Data}{Heat{\_}AdvectionDiffusion{\_}DG{\_}Data}}
			{ \it{Obligatory} }
			{{{Input fields of the equation.}}}
		\RecKey
			{Heat-AdvectionDiffusion-DG::dg-variant}
			{dg{\_}variant}
			{{Selection}{: }\Alink{IT::DG-variant}{DG{\_}variant}}
			{ \it{Non-symmetric} }
			{{{Variant of interior penalty discontinuous Galerkin method.}}}
		\RecKey
			{Heat-AdvectionDiffusion-DG::dg-order}
			{dg{\_}order}
			{{Integer [0, 3]}}
			{ \it{1} }
			{{{Polynomial order for finite element in DG method (order 0 is suitable if there is no diffusion/dispersion).}}}
		\RecKey
			{Heat-AdvectionDiffusion-DG::output}
			{output}
			{{Record}{: }\Alink{IT::EquationOutput-8}{EquationOutput}}
			{ \it{{\{}u'fields': [u'temperature']{\}}} }
			{{{Setting of the field output.}}}
\end{RecordType}
\begin{RecordType}
	{IT::Heat-AdvectionDiffusion-DG-Data}
	{Heat{\_}AdvectionDiffusion{\_}DG{\_}Data}
	{} % implements
	{} % reducible to key
	{{{Record to set fields of the equation.}\\{
The fields are set only on the domain specified by one of the keys: 'region', 'rid'}\\{
and after the time given by the key 'time'. The field setting can be overridden by}\\{
 any Heat{\_}AdvectionDiffusion{\_}DG{\_}Data record that comes later in the boundary data array.}}}
		\RecKey
			{Heat-AdvectionDiffusion-DG-Data::region}
			{region}
			{{Array [1, UINT] of }{String}}
			{ \it{Optional} }
			{{{Labels of the regions where to set fields. }}}
		\RecKey
			{Heat-AdvectionDiffusion-DG-Data::rid}
			{rid}
			{{Integer [0, INT]}}
			{ \it{Optional} }
			{{{ID of the region where to set fields.}}}
		\RecKey
			{Heat-AdvectionDiffusion-DG-Data::time}
			{time}
			{{Double [0, +inf)}}
			{ \it{0.0} }
			{{{Apply field setting in this record after this time.}\\{
These times have to form an increasing sequence.}}}
		\RecKey
			{Heat-AdvectionDiffusion-DG-Data::bc-type}
			{bc{\_}type}
			{{Abstract}{: }\Alink{IT::Field-R3---R-5}{Field:R3 -{\textgreater} R}}
			{ \it{Optional} }
			{{{Type of boundary condition. }{$[-]$}}}
		\RecKey
			{Heat-AdvectionDiffusion-DG-Data::bc-temperature}
			{bc{\_}temperature}
			{{Array [1, UINT] of }{Abstract}{: }\Alink{IT::Field-R3---R}{Field:R3 -{\textgreater} R}}
			{ \it{Optional} }
			{{{Boundary value of temperature. }{$[K]$}}}
		\RecKey
			{Heat-AdvectionDiffusion-DG-Data::bc-flux}
			{bc{\_}flux}
			{{Abstract}{: }\Alink{IT::Field-R3---R}{Field:R3 -{\textgreater} R}}
			{ \it{Optional} }
			{{{Flux in Neumann boundary condition. }{$[m^{1-d}kgs^{-1}]$}}}
		\RecKey
			{Heat-AdvectionDiffusion-DG-Data::bc-robin-sigma}
			{bc{\_}robin{\_}sigma}
			{{Abstract}{: }\Alink{IT::Field-R3---R}{Field:R3 -{\textgreater} R}}
			{ \it{Optional} }
			{{{Conductivity coefficient in Robin boundary condition. }{$[m^{4-d}s^{-1}]$}}}
		\RecKey
			{Heat-AdvectionDiffusion-DG-Data::init-temperature}
			{init{\_}temperature}
			{{Abstract}{: }\Alink{IT::Field-R3---R}{Field:R3 -{\textgreater} R}}
			{ \it{Optional} }
			{{{Initial temperature. }{$[K]$}}}
		\RecKey
			{Heat-AdvectionDiffusion-DG-Data::porosity}
			{porosity}
			{{Abstract}{: }\Alink{IT::Field-R3---R}{Field:R3 -{\textgreater} R}}
			{ \it{Optional} }
			{{{Porosity. }{$[-]$}}}
		\RecKey
			{Heat-AdvectionDiffusion-DG-Data::fluid-density}
			{fluid{\_}density}
			{{Abstract}{: }\Alink{IT::Field-R3---R}{Field:R3 -{\textgreater} R}}
			{ \it{Optional} }
			{{{Density of fluid. }{$[m^{-3}kg]$}}}
		\RecKey
			{Heat-AdvectionDiffusion-DG-Data::fluid-heat-capacity}
			{fluid{\_}heat{\_}capacity}
			{{Abstract}{: }\Alink{IT::Field-R3---R}{Field:R3 -{\textgreater} R}}
			{ \it{Optional} }
			{{{Heat capacity of fluid. }{$[m^{2}s^{-2}K^{-1}]$}}}
		\RecKey
			{Heat-AdvectionDiffusion-DG-Data::fluid-heat-conductivity}
			{fluid{\_}heat{\_}conductivity}
			{{Abstract}{: }\Alink{IT::Field-R3---R}{Field:R3 -{\textgreater} R}}
			{ \it{Optional} }
			{{{Heat conductivity of fluid. }{$[mkgs^{-3}K^{-1}]$}}}
		\RecKey
			{Heat-AdvectionDiffusion-DG-Data::solid-density}
			{solid{\_}density}
			{{Abstract}{: }\Alink{IT::Field-R3---R}{Field:R3 -{\textgreater} R}}
			{ \it{Optional} }
			{{{Density of solid (rock). }{$[m^{-3}kg]$}}}
		\RecKey
			{Heat-AdvectionDiffusion-DG-Data::solid-heat-capacity}
			{solid{\_}heat{\_}capacity}
			{{Abstract}{: }\Alink{IT::Field-R3---R}{Field:R3 -{\textgreater} R}}
			{ \it{Optional} }
			{{{Heat capacity of solid (rock). }{$[m^{2}s^{-2}K^{-1}]$}}}
		\RecKey
			{Heat-AdvectionDiffusion-DG-Data::solid-heat-conductivity}
			{solid{\_}heat{\_}conductivity}
			{{Abstract}{: }\Alink{IT::Field-R3---R}{Field:R3 -{\textgreater} R}}
			{ \it{Optional} }
			{{{Heat conductivity of solid (rock). }{$[mkgs^{-3}K^{-1}]$}}}
		\RecKey
			{Heat-AdvectionDiffusion-DG-Data::disp-l}
			{disp{\_}l}
			{{Abstract}{: }\Alink{IT::Field-R3---R}{Field:R3 -{\textgreater} R}}
			{ \it{Optional} }
			{{{Longitudal heat dispersivity in fluid. }{$[m]$}}}
		\RecKey
			{Heat-AdvectionDiffusion-DG-Data::disp-t}
			{disp{\_}t}
			{{Abstract}{: }\Alink{IT::Field-R3---R}{Field:R3 -{\textgreater} R}}
			{ \it{Optional} }
			{{{Transversal heat dispersivity in fluid. }{$[m]$}}}
		\RecKey
			{Heat-AdvectionDiffusion-DG-Data::fluid-thermal-source}
			{fluid{\_}thermal{\_}source}
			{{Abstract}{: }\Alink{IT::Field-R3---R}{Field:R3 -{\textgreater} R}}
			{ \it{Optional} }
			{{{Thermal source density in fluid. }{$[m^{-1}kgs^{-3}]$}}}
		\RecKey
			{Heat-AdvectionDiffusion-DG-Data::solid-thermal-source}
			{solid{\_}thermal{\_}source}
			{{Abstract}{: }\Alink{IT::Field-R3---R}{Field:R3 -{\textgreater} R}}
			{ \it{Optional} }
			{{{Thermal source density in solid. }{$[m^{-1}kgs^{-3}]$}}}
		\RecKey
			{Heat-AdvectionDiffusion-DG-Data::fluid-heat-exchange-rate}
			{fluid{\_}heat{\_}exchange{\_}rate}
			{{Abstract}{: }\Alink{IT::Field-R3---R}{Field:R3 -{\textgreater} R}}
			{ \it{Optional} }
			{{{Heat exchange rate in fluid. }{$[s^{-1}]$}}}
		\RecKey
			{Heat-AdvectionDiffusion-DG-Data::solid-heat-exchange-rate}
			{solid{\_}heat{\_}exchange{\_}rate}
			{{Abstract}{: }\Alink{IT::Field-R3---R}{Field:R3 -{\textgreater} R}}
			{ \it{Optional} }
			{{{Heat exchange rate of source in solid. }{$[s^{-1}]$}}}
		\RecKey
			{Heat-AdvectionDiffusion-DG-Data::fluid-ref-temperature}
			{fluid{\_}ref{\_}temperature}
			{{Abstract}{: }\Alink{IT::Field-R3---R}{Field:R3 -{\textgreater} R}}
			{ \it{Optional} }
			{{{Reference temperature of source in fluid. }{$[K]$}}}
		\RecKey
			{Heat-AdvectionDiffusion-DG-Data::solid-ref-temperature}
			{solid{\_}ref{\_}temperature}
			{{Abstract}{: }\Alink{IT::Field-R3---R}{Field:R3 -{\textgreater} R}}
			{ \it{Optional} }
			{{{Reference temperature in solid. }{$[K]$}}}
		\RecKey
			{Heat-AdvectionDiffusion-DG-Data::fracture-sigma}
			{fracture{\_}sigma}
			{{Array [1, UINT] of }{Abstract}{: }\Alink{IT::Field-R3---R}{Field:R3 -{\textgreater} R}}
			{ \it{Optional} }
			{{{Coefficient of diffusive transfer through fractures (for each substance). }{$[-]$}}}
		\RecKey
			{Heat-AdvectionDiffusion-DG-Data::dg-penalty}
			{dg{\_}penalty}
			{{Array [1, UINT] of }{Abstract}{: }\Alink{IT::Field-R3---R}{Field:R3 -{\textgreater} R}}
			{ \it{Optional} }
			{{{Penalty parameter influencing the discontinuity of the solution (for each substance). Its default value 1 is sufficient in most cases. Higher value diminishes the inter-element jumps. }{$[-]$}}}
\end{RecordType}
\begin{RecordType}
	{IT::Intersection}
	{Intersection}
	{\Alink{IT::Region}{Region}} % implements
	{} % reducible to key
	{{{Defines region as an intersection of given two or more regions.}}}
		\RecKey
			{Intersection::TYPE}
			{TYPE}
			{{String}}
			{ \it{Intersection} }
			{{{Sub-record Selection.}}}
		\RecKey
			{Intersection::name}
			{name}
			{{String}}
			{ \it{Obligatory} }
			{{{Label (name) of the region. Has to be unique in one mesh.}}}
		\RecKey
			{Intersection::regions}
			{regions}
			{{Array [2, UINT] of }{String}}
			{ \it{Obligatory} }
			{{{Defines region as an intersection of given pair of regions.}}}
\end{RecordType}
\begin{RecordType}
	{IT::Mesh}
	{Mesh}
	{} % implements
	{\Alink{Mesh::mesh-file}{mesh{\_}file}} % reducible to key
	{{{Record with mesh related data.}}}
		\RecKey
			{Mesh::mesh-file}
			{mesh{\_}file}
			{{Filename}}
			{ \it{Obligatory} }
			{{{Input file with mesh description.}}}
		\RecKey
			{Mesh::regions}
			{regions}
			{{Array [0, UINT] of }{Abstract}{: }\Alink{IT::Region}{Region}}
			{ \it{Optional} }
			{{{{List of additional region and region set definitions not contained in the mesh.}\\{
There are three region sets implicitly defined:}
}
\begin{itemize}
\item {ALL (all regions of the mesh)}
\item {.BOUNDARY (all boundary regions)}
\item {and BULK (all bulk regions)}
\end{itemize}
}}
		\RecKey
			{Mesh::partitioning}
			{partitioning}
			{{Record}{: }\Alink{IT::Partition}{Partition}}
			{ \it{Any{\_}neighboring} }
			{{{Parameters of mesh partitioning algorithms.}}}
		\RecKey
			{Mesh::print-regions}
			{print{\_}regions}
			{{Bool}}
			{ \it{False} }
			{{{If true, print table of all used regions.}}}
\end{RecordType}
\begin{RecordType}
	{IT::NonlinearSolver}
	{NonlinearSolver}
	{} % implements
	{} % reducible to key
	{{{Parameters to a non-linear solver.}}}
		\RecKey
			{NonlinearSolver::linear-solver}
			{linear{\_}solver}
			{{Abstract}{: }\Alink{IT::LinSys}{LinSys}}
			{ \it{{\{}{\}}} }
			{{{Linear solver for MH problem.}}}
		\RecKey
			{NonlinearSolver::tolerance}
			{tolerance}
			{{Double [0, +inf)}}
			{ \it{1e-06} }
			{{{Residual tolerance.}}}
		\RecKey
			{NonlinearSolver::min-it}
			{min{\_}it}
			{{Integer [0, INT]}}
			{ \it{1} }
			{{{Minimum number of iterations (linear solves) to use. This is usefull if the convergence criteria does not characterize your goal well enough so it converges prematurely possibly without the single linear solve.If greater then 'max{\_}it' the value is set to 'max{\_}it'.}}}
		\RecKey
			{NonlinearSolver::max-it}
			{max{\_}it}
			{{Integer [0, INT]}}
			{ \it{100} }
			{{{Maximum number of iterations (linear solves) of the non-linear solver.}}}
		\RecKey
			{NonlinearSolver::converge-on-stagnation}
			{converge{\_}on{\_}stagnation}
			{{Bool}}
			{ \it{False} }
			{{{If a stagnation of the nonlinear solver is detected the solver stops. A divergence is reported by default forcing the end of the simulation. Setting this flag to 'true', the solverends with convergence success on stagnation, but report warning about it.}}}
\end{RecordType}
\begin{RecordType}
	{IT::ObservePoint}
	{ObservePoint}
	{} % implements
	{\Alink{ObservePoint::point}{point}} % reducible to key
	{{{{Specification of the observation point. The actual observe element and the observe point on it is determined as follows:}
}
\begin{enumerate}
\item {Find an initial element containing the initial point. If no such element exists we report the error.}
\item {Use BFS starting from the inital element to find the 'observe element'. The observe element is the closest element 3. Find the closest projection of the inital point on the observe element and snap this projection according to the 'snap{\_}dim'.}
\end{enumerate}
}}
		\RecKey
			{ObservePoint::name}
			{name}
			{{String}}
			{implicit value: "{Default name have the form 'obs{\_}{\textless}id{\textgreater}', where 'id' is the rank of the point on the input.}"}
			{{{Optional point name. Has to be unique. Any string that is valid YAML key in record without any quoting can be used howeverusing just alpha-numerical characters and underscore instead of the space is recommended. }}}
		\RecKey
			{ObservePoint::point}
			{point}
			{{Array [3, 3] of }{Double (-inf, +inf)}}
			{ \it{Obligatory} }
			{{{Initial point for the observe point search.}}}
		\RecKey
			{ObservePoint::snap-dim}
			{snap{\_}dim}
			{{Integer [0, 4]}}
			{ \it{4} }
			{{{The dimension of the sub-element to which center we snap. For value 4 no snapping is done. For values 0 up to 3 the element containing the initial point is found and then the observepoint is snapped to the nearest center of the sub-element of the given dimension. E.g. for dimension 2 we snap to the nearest center of the face of the initial element.}}}
		\RecKey
			{ObservePoint::snap-region}
			{snap{\_}region}
			{{String}}
			{ \it{All} }
			{{{The region of the initial element for snapping. Without snapping we make a projection to the initial element.}}}
		\RecKey
			{ObservePoint::n-search-levels}
			{n{\_}search{\_}levels}
			{{Integer [0, INT]}}
			{ \it{1} }
			{{{Maximum number of levels of the breadth first search used to find the observe element from the initial element. Value zero means to search only the initial element itself.}}}
\end{RecordType}
\begin{RecordType}
	{IT::OutputMesh}
	{OutputMesh}
	{} % implements
	{} % reducible to key
	{{{Parameters of the refined output mesh.}}}
		\RecKey
			{OutputMesh::max-level}
			{max{\_}level}
			{{Integer [1, 20]}}
			{ \it{3} }
			{{{Maximal level of refinement of the output mesh.}}}
		\RecKey
			{OutputMesh::refine-by-error}
			{refine{\_}by{\_}error}
			{{Bool}}
			{ \it{False} }
			{{{Set true for using error{\_}control{\_}field. Set false for global uniform refinement to max{\_}level.}}}
		\RecKey
			{OutputMesh::error-control-field}
			{error{\_}control{\_}field}
			{{String}}
			{ \it{Optional} }
			{{{Name of an output field, according to which the output mesh will be refined. The field must be a SCALAR one.}}}
\end{RecordType}
\begin{RecordType}
	{IT::OutputStream}
	{OutputStream}
	{} % implements
	{} % reducible to key
	{{{Configuration of the spatial output of a single balance equation.}}}
		\RecKey
			{OutputStream::file}
			{file}
			{{Filename}}
			{implicit value: "{Name of the equation associated with the output stream.}"}
			{{{File path to the connected output file.}}}
		\RecKey
			{OutputStream::format}
			{format}
			{{Abstract}{: }\Alink{IT::OutputTime}{OutputTime}}
			{ \it{{\{}{\}}} }
			{{{Format of output stream and possible parameters.}}}
		\RecKey
			{OutputStream::times}
			{times}
			{{Array [0, UINT] of }{Record}{: }\Alink{IT::TimeGrid}{TimeGrid}}
			{ \it{Optional} }
			{{{Output times used for equations without is own output times key.}}}
		\RecKey
			{OutputStream::output-mesh}
			{output{\_}mesh}
			{{Record}{: }\Alink{IT::OutputMesh}{OutputMesh}}
			{ \it{Optional} }
			{{{Output mesh record enables output on a refined mesh.}}}
		\RecKey
			{OutputStream::precision}
			{precision}
			{{Integer [0, INT]}}
			{ \it{5} }
			{{{The number of decimal digits used in output of floating point values.}}}
		\RecKey
			{OutputStream::observe-points}
			{observe{\_}points}
			{{Array [0, UINT] of }{Record}{: }\Alink{IT::ObservePoint}{ObservePoint}}
			{ \it{[]} }
			{{{Array of observe points.}}}
\end{RecordType}
\begin{RecordType}
	{IT::Output-DarcyMHSpecific}
	{Output{\_}DarcyMHSpecific}
	{} % implements
	{} % reducible to key
	{{{Specific Darcy flow MH output.}}}
		\RecKey
			{Output-DarcyMHSpecific::compute-errors}
			{compute{\_}errors}
			{{Bool}}
			{ \it{False} }
			{{{SPECIAL PURPOSE. Computing errors pro non-compatible coupling.}}}
		\RecKey
			{Output-DarcyMHSpecific::raw-flow-output}
			{raw{\_}flow{\_}output}
			{{Filename}}
			{ \it{Optional} }
			{{{Output file with raw data form MH module.}}}
\end{RecordType}
\begin{RecordType}
	{IT::PadeApproximant}
	{PadeApproximant}
	{} % implements
	{} % reducible to key
	{{{Record with an information about pade approximant parameters.Note that stable method is guaranteed only if d-n=1 or d-n=2, where d=degree of denominator and n=degree of nominator. In those cases the Pade approximant corresponds to an implicit Runge-Kutta method which is both A- and L-stable. The default values n=2, d=3 yield relatively good precision while keeping the order moderately low.}}}
		\RecKey
			{PadeApproximant::pade-nominator-degree}
			{pade{\_}nominator{\_}degree}
			{{Integer [1, INT]}}
			{ \it{1} }
			{{{Polynomial degree of the nominator of Pade approximant.}}}
		\RecKey
			{PadeApproximant::pade-denominator-degree}
			{pade{\_}denominator{\_}degree}
			{{Integer [1, INT]}}
			{ \it{3} }
			{{{Polynomial degree of the denominator of Pade approximant}}}
\end{RecordType}
\begin{RecordType}
	{IT::Partition}
	{Partition}
	{} % implements
	{\Alink{Partition::graph-type}{graph{\_}type}} % reducible to key
	{{{Setting for various types of mesh partitioning.}}}
		\RecKey
			{Partition::tool}
			{tool}
			{{Selection}{: }\Alink{IT::PartTool}{PartTool}}
			{ \it{Metis} }
			{{{Software package used for partitioning. See corresponding selection.}}}
		\RecKey
			{Partition::graph-type}
			{graph{\_}type}
			{{Selection}{: }\Alink{IT::GraphType}{GraphType}}
			{ \it{Any{\_}neighboring} }
			{{{Algorithm for generating graph and its weights from a multidimensional mesh.}}}
\end{RecordType}
\begin{RecordType}
	{IT::Petsc}
	{Petsc}
	{\Alink{IT::LinSys}{LinSys}} % implements
	{} % reducible to key
	{{{Interface to PETSc solvers. Convergence criteria is:}\\
\ttfamily norm( res{\_}n )  {\textless} max( norm( res{\_}0 ) * r{\_}tol, a{\_}tol )\\{
where res{\_}i is the residuum vector after i-th iteration of the solver and res{\_}0 is an estimate of the norm of initial residual.}\\{
If the initial guess of the solution is provided (usually only for transient equations) the residual of this estimate is used,}\\{
otherwise the norm of preconditioned RHS is used.}\\{
The default norm is L2 norm of preconditioned residual: }{$ P^{-1}(Ax-b)$}{, usage of other norm may be prescribed using the 'option' key.}\\{
See also PETSc documentation for KSPSetNormType.}}}
		\RecKey
			{Petsc::TYPE}
			{TYPE}
			{{String}}
			{ \it{Petsc} }
			{{{Sub-record Selection.}}}
		\RecKey
			{Petsc::r-tol}
			{r{\_}tol}
			{{Double [0, 1]}}
			{implicit value: "{Defalut value set by nonlinear solver or equation. If not we use value 1.0e-7.}"}
			{{{Relative residual tolerance,  (to initial error).}}}
		\RecKey
			{Petsc::a-tol}
			{a{\_}tol}
			{{Double [0, +inf)}}
			{implicit value: "{Defalut value set by nonlinear solver or equation. If not we use value 1.0e-11.}"}
			{{{Absolute residual tolerance.}}}
		\RecKey
			{Petsc::max-it}
			{max{\_}it}
			{{Integer [0, INT]}}
			{implicit value: "{Defalut value set by nonlinear solver or equation. If not we use value 1000.}"}
			{{{Maximum number of outer iterations of the linear solver.}}}
		\RecKey
			{Petsc::options}
			{options}
			{{String}}
			{ \it{} }
			{{{Options passed to PETSC before creating KSP instead of default setting.}}}
\end{RecordType}
\begin{RecordType}
	{IT::RadioactiveDecay}
	{RadioactiveDecay}
	{\Alink{IT::ReactionTerm}{ReactionTerm}\Alink{IT::GenericReaction}{GenericReaction}\Alink{IT::ReactionTermMobile}{ReactionTermMobile}\Alink{IT::ReactionTermImmobile}{ReactionTermImmobile}} % implements
	{} % reducible to key
	{{{A model of a radioactive decay and possibly of a decay chain.}}}
		\RecKey
			{RadioactiveDecay::TYPE}
			{TYPE}
			{{String}}
			{ \it{Radioactivedecay} }
			{{{Sub-record Selection.}}}
		\RecKey
			{RadioactiveDecay::decays}
			{decays}
			{{Array [1, UINT] of }{Record}{: }\Alink{IT::Decay}{Decay}}
			{ \it{Obligatory} }
			{{{An array of radioactive decays.}}}
		\RecKey
			{RadioactiveDecay::ode-solver}
			{ode{\_}solver}
			{{Record}{: }\Alink{IT::PadeApproximant}{PadeApproximant}}
			{ \it{{\{}{\}}} }
			{{{Numerical solver for the system of first order ordinary differential equations coming from the model.}}}
\end{RecordType}
\begin{RecordType}
	{IT::RadioactiveDecayProduct}
	{RadioactiveDecayProduct}
	{} % implements
	{\Alink{RadioactiveDecayProduct::name}{name}} % reducible to key
	{{{A record describing a product of a radioactive decay.}}}
		\RecKey
			{RadioactiveDecayProduct::name}
			{name}
			{{String}}
			{ \it{Obligatory} }
			{{{The name of the product.}}}
		\RecKey
			{RadioactiveDecayProduct::energy}
			{energy}
			{{Double [0, +inf)}}
			{ \it{0.0} }
			{{{Not used at the moment! The released energy in MeV from the decay of the radionuclide into the product.}}}
		\RecKey
			{RadioactiveDecayProduct::branching-ratio}
			{branching{\_}ratio}
			{{Double [0, +inf)}}
			{ \it{1.0} }
			{{{The branching ratio of the product when there is more than one.Considering only one product, the default ratio 1.0 is used.Its value must be positive. Further, the branching ratios of all products are normalizedby their sum, so the sum then gives 1.0 (this also resolves possible rounding errors).}}}
\end{RecordType}
\begin{RecordType}
	{IT::Reaction}
	{Reaction}
	{} % implements
	{} % reducible to key
	{{{Describes a single first order chemical reaction.}}}
		\RecKey
			{Reaction::reactants}
			{reactants}
			{{Array [1, UINT] of }{Record}{: }\Alink{IT::FirstOrderReactionReactant}{FirstOrderReactionReactant}}
			{ \it{Obligatory} }
			{{{An array of reactants. Do not use array, reactions with only one reactant (decays) are implemented at the moment!}}}
		\RecKey
			{Reaction::reaction-rate}
			{reaction{\_}rate}
			{{Double [0, +inf)}}
			{ \it{Obligatory} }
			{{{The reaction rate coefficient of the first order reaction.}}}
		\RecKey
			{Reaction::products}
			{products}
			{{Array [1, UINT] of }{Record}{: }\Alink{IT::FirstOrderReactionProduct}{FirstOrderReactionProduct}}
			{ \it{Obligatory} }
			{{{An array of products.}}}
\end{RecordType}
\begin{RecordType}
	{IT::RichardsLMH-Data}
	{RichardsLMH{\_}Data}
	{} % implements
	{} % reducible to key
	{{{Record to set fields of the equation.}\\{
The fields are set only on the domain specified by one of the keys: 'region', 'rid'}\\{
and after the time given by the key 'time'. The field setting can be overridden by}\\{
 any RichardsLMH{\_}Data record that comes later in the boundary data array.}}}
		\RecKey
			{RichardsLMH-Data::region}
			{region}
			{{Array [1, UINT] of }{String}}
			{ \it{Optional} }
			{{{Labels of the regions where to set fields. }}}
		\RecKey
			{RichardsLMH-Data::rid}
			{rid}
			{{Integer [0, INT]}}
			{ \it{Optional} }
			{{{ID of the region where to set fields.}}}
		\RecKey
			{RichardsLMH-Data::time}
			{time}
			{{Double [0, +inf)}}
			{ \it{0.0} }
			{{{Apply field setting in this record after this time.}\\{
These times have to form an increasing sequence.}}}
		\RecKey
			{RichardsLMH-Data::anisotropy}
			{anisotropy}
			{{Abstract}{: }\Alink{IT::Field-R3---R-3-3-}{Field:R3 -{\textgreater} R[3,3]}}
			{ \it{Optional} }
			{{{Anisotropy of the conductivity tensor. }{$[-]$}}}
		\RecKey
			{RichardsLMH-Data::cross-section}
			{cross{\_}section}
			{{Abstract}{: }\Alink{IT::Field-R3---R}{Field:R3 -{\textgreater} R}}
			{ \it{Optional} }
			{{{Complement dimension parameter (cross section for 1D, thickness for 2D). }{$[m^{3-d}]$}}}
		\RecKey
			{RichardsLMH-Data::conductivity}
			{conductivity}
			{{Abstract}{: }\Alink{IT::Field-R3---R}{Field:R3 -{\textgreater} R}}
			{ \it{Optional} }
			{{{Isotropic conductivity scalar. }{$[ms^{-1}]$}}}
		\RecKey
			{RichardsLMH-Data::sigma}
			{sigma}
			{{Abstract}{: }\Alink{IT::Field-R3---R}{Field:R3 -{\textgreater} R}}
			{ \it{Optional} }
			{{{Transition coefficient between dimensions. }{$[-]$}}}
		\RecKey
			{RichardsLMH-Data::water-source-density}
			{water{\_}source{\_}density}
			{{Abstract}{: }\Alink{IT::Field-R3---R}{Field:R3 -{\textgreater} R}}
			{ \it{Optional} }
			{{{Water source density. }{$[s^{-1}]$}}}
		\RecKey
			{RichardsLMH-Data::bc-type}
			{bc{\_}type}
			{{Abstract}{: }\Alink{IT::Field-R3---R-2}{Field:R3 -{\textgreater} R}}
			{ \it{Optional} }
			{{{Boundary condition type, possible values: }{$[-]$}}}
		\RecKey
			{RichardsLMH-Data::bc-pressure}
			{bc{\_}pressure}
			{{Abstract}{: }\Alink{IT::Field-R3---R}{Field:R3 -{\textgreater} R}}
			{ \it{Optional} }
			{{{Prescribed pressure value on the boundary. Used for all values of 'bc{\_}type' save the bc{\_}type='none'.See documentation of 'bc{\_}type' for exact meaning of 'bc{\_}pressure' in individual boundary condition types. }{$[m]$}}}
		\RecKey
			{RichardsLMH-Data::bc-flux}
			{bc{\_}flux}
			{{Abstract}{: }\Alink{IT::Field-R3---R}{Field:R3 -{\textgreater} R}}
			{ \it{Optional} }
			{{{Incoming water boundary flux. Used for bc{\_}types : 'none', 'total{\_}flux', 'seepage', 'river'. }{$[m^{4-d}s^{-1}]$}}}
		\RecKey
			{RichardsLMH-Data::bc-robin-sigma}
			{bc{\_}robin{\_}sigma}
			{{Abstract}{: }\Alink{IT::Field-R3---R}{Field:R3 -{\textgreater} R}}
			{ \it{Optional} }
			{{{Conductivity coefficient in the 'total{\_}flux' or the 'river' boundary condition type. }{$[m^{3-d}s^{-1}]$}}}
		\RecKey
			{RichardsLMH-Data::bc-switch-pressure}
			{bc{\_}switch{\_}pressure}
			{{Abstract}{: }\Alink{IT::Field-R3---R}{Field:R3 -{\textgreater} R}}
			{ \it{Optional} }
			{{{Critical switch pressure for 'seepage' and 'river' boundary conditions. }{$[m]$}}}
		\RecKey
			{RichardsLMH-Data::init-pressure}
			{init{\_}pressure}
			{{Abstract}{: }\Alink{IT::Field-R3---R}{Field:R3 -{\textgreater} R}}
			{ \it{Optional} }
			{{{Initial condition as pressure }{$[m]$}}}
		\RecKey
			{RichardsLMH-Data::storativity}
			{storativity}
			{{Abstract}{: }\Alink{IT::Field-R3---R}{Field:R3 -{\textgreater} R}}
			{ \it{Optional} }
			{{{Storativity. }{$[m^{-1}]$}}}
		\RecKey
			{RichardsLMH-Data::water-content-saturated}
			{water{\_}content{\_}saturated}
			{{Abstract}{: }\Alink{IT::Field-R3---R}{Field:R3 -{\textgreater} R}}
			{ \it{Optional} }
			{{{Saturated water content }{$ \theta_s $}{.}\\{
relative volume of the water in a reference volume of a saturated porous media.}\\{
 }{$[-]$}}}
		\RecKey
			{RichardsLMH-Data::water-content-residual}
			{water{\_}content{\_}residual}
			{{Abstract}{: }\Alink{IT::Field-R3---R}{Field:R3 -{\textgreater} R}}
			{ \it{Optional} }
			{{{Residual water content }{$ \theta_r $}{.}\\{
Relative volume of the water in a reference volume of an ideally dry porous media.}\\{
 }{$[-]$}}}
		\RecKey
			{RichardsLMH-Data::genuchten-p-head-scale}
			{genuchten{\_}p{\_}head{\_}scale}
			{{Abstract}{: }\Alink{IT::Field-R3---R}{Field:R3 -{\textgreater} R}}
			{ \it{Optional} }
			{{{The van Genuchten pressure head scaling parameter }{$ \alpha $}{.}\\{
The parameter of the van Genuchten's model to scale the pressure head.}\\{
Related to the inverse of the air entry pressure, i.e. the pressure where the relative water content starts to decrease below 1.}\\{
 }{$[m^{-1}]$}}}
		\RecKey
			{RichardsLMH-Data::genuchten-n-exponent}
			{genuchten{\_}n{\_}exponent}
			{{Abstract}{: }\Alink{IT::Field-R3---R}{Field:R3 -{\textgreater} R}}
			{ \it{Optional} }
			{{{The van Genuchten exponent parameter }{$ n $}{. }{$[-]$}}}
		\RecKey
			{RichardsLMH-Data::bc-piezo-head}
			{bc{\_}piezo{\_}head}
			{{Abstract}{: }\Alink{IT::Field-R3---R}{Field:R3 -{\textgreater} R}}
			{ \it{Optional} }
			{{{Boundary piezometric head for BC types: dirichlet, robin, and river.}}}
		\RecKey
			{RichardsLMH-Data::bc-switch-piezo-head}
			{bc{\_}switch{\_}piezo{\_}head}
			{{Abstract}{: }\Alink{IT::Field-R3---R}{Field:R3 -{\textgreater} R}}
			{ \it{Optional} }
			{{{Boundary switch piezometric head for BC types: seepage, river.}}}
		\RecKey
			{RichardsLMH-Data::init-piezo-head}
			{init{\_}piezo{\_}head}
			{{Abstract}{: }\Alink{IT::Field-R3---R}{Field:R3 -{\textgreater} R}}
			{ \it{Optional} }
			{{{Initial condition for the pressure given as the piezometric head.}}}
\end{RecordType}
\begin{RecordType}
	{IT::Root}
	{Root}
	{} % implements
	{} % reducible to key
	{{{Root record of JSON input for Flow123d.}}}
		\RecKey
			{Root::flow123d-version}
			{flow123d{\_}version}
			{{String}}
			{ \it{Obligatory} }
			{{{Version of Flow123d for which the input file was created.Flow123d only warn about version incompatibility. However, external tools may use this information to provide conversion of the input file to the structure required by another version of Flow123d.}}}
		\RecKey
			{Root::problem}
			{problem}
			{{Abstract}{: }\Alink{IT::Coupling-Base}{Coupling{\_}Base}}
			{ \it{Obligatory} }
			{{{Simulation problem to be solved.}}}
		\RecKey
			{Root::pause-after-run}
			{pause{\_}after{\_}run}
			{{Bool}}
			{ \it{False} }
			{{{If true, the program will wait for key press before it terminates.}}}
\end{RecordType}
\begin{RecordType}
	{IT::SoilModel}
	{SoilModel}
	{} % implements
	{\Alink{SoilModel::model-type}{model{\_}type}} % reducible to key
	{{{Setting for the soil model.}}}
		\RecKey
			{SoilModel::model-type}
			{model{\_}type}
			{{Selection}{: }\Alink{IT::Soil-Model-Type}{Soil{\_}Model{\_}Type}}
			{ \it{Van{\_}genuchten} }
			{{{Selection of the globally applied soil model. In future we replace this key by a field for selection of the model.That will allow usage of different soil model in a single simulation.}}}
		\RecKey
			{SoilModel::cut-fraction}
			{cut{\_}fraction}
			{{Double [0, 1]}}
			{ \it{0.999} }
			{{{Fraction of the water content where we cut  and rescale the curve.}}}
\end{RecordType}
\begin{RecordType}
	{IT::Solute-AdvectionDiffusion-DG}
	{Solute{\_}AdvectionDiffusion{\_}DG}
	{\Alink{IT::Solute}{Solute}} % implements
	{} % reducible to key
	{{{DG solver for solute transport.}}}
		\RecKey
			{Solute-AdvectionDiffusion-DG::TYPE}
			{TYPE}
			{{String}}
			{ \it{Solute{\_}advectiondiffusion{\_}dg} }
			{{{Sub-record Selection.}}}
		\RecKey
			{Solute-AdvectionDiffusion-DG::solvent-density}
			{solvent{\_}density}
			{{Double [0, +inf)}}
			{ \it{1.0} }
			{{{Density of the solvent [ }{$kg.m^(-3)$}{ ].}}}
		\RecKey
			{Solute-AdvectionDiffusion-DG::solver}
			{solver}
			{{Record}{: }\Alink{IT::Petsc}{Petsc}}
			{ \it{Obligatory} }
			{{{Linear solver for MH problem.}}}
		\RecKey
			{Solute-AdvectionDiffusion-DG::input-fields}
			{input{\_}fields}
			{{Array [0, UINT] of }{Record}{: }\Alink{IT::Solute-AdvectionDiffusion-DG-Data}{Solute{\_}AdvectionDiffusion{\_}DG{\_}Data}}
			{ \it{Obligatory} }
			{{{Input fields of the equation.}}}
		\RecKey
			{Solute-AdvectionDiffusion-DG::dg-variant}
			{dg{\_}variant}
			{{Selection}{: }\Alink{IT::DG-variant}{DG{\_}variant}}
			{ \it{Non-symmetric} }
			{{{Variant of interior penalty discontinuous Galerkin method.}}}
		\RecKey
			{Solute-AdvectionDiffusion-DG::dg-order}
			{dg{\_}order}
			{{Integer [0, 3]}}
			{ \it{1} }
			{{{Polynomial order for finite element in DG method (order 0 is suitable if there is no diffusion/dispersion).}}}
		\RecKey
			{Solute-AdvectionDiffusion-DG::output}
			{output}
			{{Record}{: }\Alink{IT::EquationOutput-3}{EquationOutput}}
			{ \it{{\{}u'fields': [u'conc']{\}}} }
			{{{Setting of the field output.}}}
\end{RecordType}
\begin{RecordType}
	{IT::Solute-AdvectionDiffusion-DG-Data}
	{Solute{\_}AdvectionDiffusion{\_}DG{\_}Data}
	{} % implements
	{} % reducible to key
	{{{Record to set fields of the equation.}\\{
The fields are set only on the domain specified by one of the keys: 'region', 'rid'}\\{
and after the time given by the key 'time'. The field setting can be overridden by}\\{
 any Solute{\_}AdvectionDiffusion{\_}DG{\_}Data record that comes later in the boundary data array.}}}
		\RecKey
			{Solute-AdvectionDiffusion-DG-Data::region}
			{region}
			{{Array [1, UINT] of }{String}}
			{ \it{Optional} }
			{{{Labels of the regions where to set fields. }}}
		\RecKey
			{Solute-AdvectionDiffusion-DG-Data::rid}
			{rid}
			{{Integer [0, INT]}}
			{ \it{Optional} }
			{{{ID of the region where to set fields.}}}
		\RecKey
			{Solute-AdvectionDiffusion-DG-Data::time}
			{time}
			{{Double [0, +inf)}}
			{ \it{0.0} }
			{{{Apply field setting in this record after this time.}\\{
These times have to form an increasing sequence.}}}
		\RecKey
			{Solute-AdvectionDiffusion-DG-Data::porosity}
			{porosity}
			{{Abstract}{: }\Alink{IT::Field-R3---R}{Field:R3 -{\textgreater} R}}
			{ \it{Optional} }
			{{{Mobile porosity }{$[-]$}}}
		\RecKey
			{Solute-AdvectionDiffusion-DG-Data::sources-density}
			{sources{\_}density}
			{{Array [1, UINT] of }{Abstract}{: }\Alink{IT::Field-R3---R}{Field:R3 -{\textgreater} R}}
			{ \it{Optional} }
			{{{Density of concentration sources. }{$[m^{-3}kgs^{-1}]$}}}
		\RecKey
			{Solute-AdvectionDiffusion-DG-Data::sources-sigma}
			{sources{\_}sigma}
			{{Array [1, UINT] of }{Abstract}{: }\Alink{IT::Field-R3---R}{Field:R3 -{\textgreater} R}}
			{ \it{Optional} }
			{{{Concentration flux. }{$[s^{-1}]$}}}
		\RecKey
			{Solute-AdvectionDiffusion-DG-Data::sources-conc}
			{sources{\_}conc}
			{{Array [1, UINT] of }{Abstract}{: }\Alink{IT::Field-R3---R}{Field:R3 -{\textgreater} R}}
			{ \it{Optional} }
			{{{Concentration sources threshold. }{$[m^{-3}kg]$}}}
		\RecKey
			{Solute-AdvectionDiffusion-DG-Data::bc-type}
			{bc{\_}type}
			{{Array [1, UINT] of }{Abstract}{: }\Alink{IT::Field-R3---R-3}{Field:R3 -{\textgreater} R}}
			{ \it{Optional} }
			{{{Type of boundary condition. }{$[-]$}}}
		\RecKey
			{Solute-AdvectionDiffusion-DG-Data::bc-conc}
			{bc{\_}conc}
			{{Array [1, UINT] of }{Abstract}{: }\Alink{IT::Field-R3---R}{Field:R3 -{\textgreater} R}}
			{ \it{Optional} }
			{{{Dirichlet boundary condition (for each substance). }{$[m^{-3}kg]$}}}
		\RecKey
			{Solute-AdvectionDiffusion-DG-Data::bc-flux}
			{bc{\_}flux}
			{{Array [1, UINT] of }{Abstract}{: }\Alink{IT::Field-R3---R}{Field:R3 -{\textgreater} R}}
			{ \it{Optional} }
			{{{Flux in Neumann boundary condition. }{$[m^{1-d}kgs^{-1}]$}}}
		\RecKey
			{Solute-AdvectionDiffusion-DG-Data::bc-robin-sigma}
			{bc{\_}robin{\_}sigma}
			{{Array [1, UINT] of }{Abstract}{: }\Alink{IT::Field-R3---R}{Field:R3 -{\textgreater} R}}
			{ \it{Optional} }
			{{{Conductivity coefficient in Robin boundary condition. }{$[m^{4-d}s^{-1}]$}}}
		\RecKey
			{Solute-AdvectionDiffusion-DG-Data::init-conc}
			{init{\_}conc}
			{{Array [1, UINT] of }{Abstract}{: }\Alink{IT::Field-R3---R}{Field:R3 -{\textgreater} R}}
			{ \it{Optional} }
			{{{Initial concentrations. }{$[m^{-3}kg]$}}}
		\RecKey
			{Solute-AdvectionDiffusion-DG-Data::disp-l}
			{disp{\_}l}
			{{Array [1, UINT] of }{Abstract}{: }\Alink{IT::Field-R3---R}{Field:R3 -{\textgreater} R}}
			{ \it{Optional} }
			{{{Longitudal dispersivity (for each substance). }{$[m]$}}}
		\RecKey
			{Solute-AdvectionDiffusion-DG-Data::disp-t}
			{disp{\_}t}
			{{Array [1, UINT] of }{Abstract}{: }\Alink{IT::Field-R3---R}{Field:R3 -{\textgreater} R}}
			{ \it{Optional} }
			{{{Transversal dispersivity (for each substance). }{$[m]$}}}
		\RecKey
			{Solute-AdvectionDiffusion-DG-Data::diff-m}
			{diff{\_}m}
			{{Array [1, UINT] of }{Abstract}{: }\Alink{IT::Field-R3---R}{Field:R3 -{\textgreater} R}}
			{ \it{Optional} }
			{{{Molecular diffusivity (for each substance). }{$[m^{2}s^{-1}]$}}}
		\RecKey
			{Solute-AdvectionDiffusion-DG-Data::rock-density}
			{rock{\_}density}
			{{Abstract}{: }\Alink{IT::Field-R3---R}{Field:R3 -{\textgreater} R}}
			{ \it{Optional} }
			{{{Rock matrix density. }{$[m^{-3}kg]$}}}
		\RecKey
			{Solute-AdvectionDiffusion-DG-Data::sorption-mult}
			{sorption{\_}mult}
			{{Array [1, UINT] of }{Abstract}{: }\Alink{IT::Field-R3---R}{Field:R3 -{\textgreater} R}}
			{ \it{Optional} }
			{{{Coefficient of linear sorption. }{$[kg^{-1}mol]$}}}
		\RecKey
			{Solute-AdvectionDiffusion-DG-Data::fracture-sigma}
			{fracture{\_}sigma}
			{{Array [1, UINT] of }{Abstract}{: }\Alink{IT::Field-R3---R}{Field:R3 -{\textgreater} R}}
			{ \it{Optional} }
			{{{Coefficient of diffusive transfer through fractures (for each substance). }{$[-]$}}}
		\RecKey
			{Solute-AdvectionDiffusion-DG-Data::dg-penalty}
			{dg{\_}penalty}
			{{Array [1, UINT] of }{Abstract}{: }\Alink{IT::Field-R3---R}{Field:R3 -{\textgreater} R}}
			{ \it{Optional} }
			{{{Penalty parameter influencing the discontinuity of the solution (for each substance). Its default value 1 is sufficient in most cases. Higher value diminishes the inter-element jumps. }{$[-]$}}}
\end{RecordType}
\begin{RecordType}
	{IT::Solute-Advection-FV}
	{Solute{\_}Advection{\_}FV}
	{\Alink{IT::Solute}{Solute}} % implements
	{} % reducible to key
	{{{Explicit in time finite volume method for advection only solute transport.}}}
		\RecKey
			{Solute-Advection-FV::TYPE}
			{TYPE}
			{{String}}
			{ \it{Solute{\_}advection{\_}fv} }
			{{{Sub-record Selection.}}}
		\RecKey
			{Solute-Advection-FV::input-fields}
			{input{\_}fields}
			{{Array [0, UINT] of }{Record}{: }\Alink{IT::Solute-Advection-FV-Data}{Solute{\_}Advection{\_}FV{\_}Data}}
			{ \it{Obligatory} }
			{}
		\RecKey
			{Solute-Advection-FV::output}
			{output}
			{{Record}{: }\Alink{IT::EquationOutput-2}{EquationOutput}}
			{ \it{{\{}u'fields': [u'conc']{\}}} }
			{{{Setting of the fields output.}}}
\end{RecordType}
\begin{RecordType}
	{IT::Solute-Advection-FV-Data}
	{Solute{\_}Advection{\_}FV{\_}Data}
	{} % implements
	{} % reducible to key
	{{{Record to set fields of the equation.}\\{
The fields are set only on the domain specified by one of the keys: 'region', 'rid'}\\{
and after the time given by the key 'time'. The field setting can be overridden by}\\{
 any Solute{\_}Advection{\_}FV{\_}Data record that comes later in the boundary data array.}}}
		\RecKey
			{Solute-Advection-FV-Data::region}
			{region}
			{{Array [1, UINT] of }{String}}
			{ \it{Optional} }
			{{{Labels of the regions where to set fields. }}}
		\RecKey
			{Solute-Advection-FV-Data::rid}
			{rid}
			{{Integer [0, INT]}}
			{ \it{Optional} }
			{{{ID of the region where to set fields.}}}
		\RecKey
			{Solute-Advection-FV-Data::time}
			{time}
			{{Double [0, +inf)}}
			{ \it{0.0} }
			{{{Apply field setting in this record after this time.}\\{
These times have to form an increasing sequence.}}}
		\RecKey
			{Solute-Advection-FV-Data::porosity}
			{porosity}
			{{Abstract}{: }\Alink{IT::Field-R3---R}{Field:R3 -{\textgreater} R}}
			{ \it{Optional} }
			{{{Mobile porosity }{$[-]$}}}
		\RecKey
			{Solute-Advection-FV-Data::sources-density}
			{sources{\_}density}
			{{Array [1, UINT] of }{Abstract}{: }\Alink{IT::Field-R3---R}{Field:R3 -{\textgreater} R}}
			{ \it{Optional} }
			{{{Density of concentration sources. }{$[m^{-3}kgs^{-1}]$}}}
		\RecKey
			{Solute-Advection-FV-Data::sources-sigma}
			{sources{\_}sigma}
			{{Array [1, UINT] of }{Abstract}{: }\Alink{IT::Field-R3---R}{Field:R3 -{\textgreater} R}}
			{ \it{Optional} }
			{{{Concentration flux. }{$[s^{-1}]$}}}
		\RecKey
			{Solute-Advection-FV-Data::sources-conc}
			{sources{\_}conc}
			{{Array [1, UINT] of }{Abstract}{: }\Alink{IT::Field-R3---R}{Field:R3 -{\textgreater} R}}
			{ \it{Optional} }
			{{{Concentration sources threshold. }{$[m^{-3}kg]$}}}
		\RecKey
			{Solute-Advection-FV-Data::bc-conc}
			{bc{\_}conc}
			{{Array [1, UINT] of }{Abstract}{: }\Alink{IT::Field-R3---R}{Field:R3 -{\textgreater} R}}
			{ \it{Optional} }
			{{{Boundary conditions for concentrations. }{$[m^{-3}kg]$}}}
		\RecKey
			{Solute-Advection-FV-Data::init-conc}
			{init{\_}conc}
			{{Array [1, UINT] of }{Abstract}{: }\Alink{IT::Field-R3---R}{Field:R3 -{\textgreater} R}}
			{ \it{Optional} }
			{{{Initial concentrations. }{$[m^{-3}kg]$}}}
\end{RecordType}
\begin{RecordType}
	{IT::Sorption}
	{Sorption}
	{\Alink{IT::ReactionTerm}{ReactionTerm}} % implements
	{} % reducible to key
	{{{Sorption model in the reaction term of transport.}}}
		\RecKey
			{Sorption::TYPE}
			{TYPE}
			{{String}}
			{ \it{Sorption} }
			{{{Sub-record Selection.}}}
		\RecKey
			{Sorption::substances}
			{substances}
			{{Array [1, UINT] of }{String}}
			{ \it{Obligatory} }
			{{{Names of the substances that take part in the sorption model.}}}
		\RecKey
			{Sorption::solvent-density}
			{solvent{\_}density}
			{{Double [0, +inf)}}
			{ \it{1.0} }
			{{{Density of the solvent.}}}
		\RecKey
			{Sorption::substeps}
			{substeps}
			{{Integer [1, INT]}}
			{ \it{1000} }
			{{{Number of equidistant substeps, molar mass and isotherm intersections}}}
		\RecKey
			{Sorption::solubility}
			{solubility}
			{{Array [0, UINT] of }{Double [0, +inf)}}
			{ \it{Optional} }
			{{{Specifies solubility limits of all the sorbing species.}}}
		\RecKey
			{Sorption::table-limits}
			{table{\_}limits}
			{{Array [0, UINT] of }{Double [0, +inf)}}
			{ \it{Optional} }
			{{{Specifies highest aqueous concentration in interpolation table.}}}
		\RecKey
			{Sorption::input-fields}
			{input{\_}fields}
			{{Array [0, UINT] of }{Record}{: }\Alink{IT::Sorption-Data}{Sorption{\_}Data}}
			{ \it{Obligatory} }
			{{{Containes region specific data necessary to construct isotherms.}}}
		\RecKey
			{Sorption::reaction-liquid}
			{reaction{\_}liquid}
			{{Abstract}{: }\Alink{IT::GenericReaction}{GenericReaction}}
			{ \it{Optional} }
			{{{Reaction model following the sorption in the liquid.}}}
		\RecKey
			{Sorption::reaction-solid}
			{reaction{\_}solid}
			{{Abstract}{: }\Alink{IT::GenericReaction}{GenericReaction}}
			{ \it{Optional} }
			{{{Reaction model following the sorption in the solid.}}}
		\RecKey
			{Sorption::output}
			{output}
			{{Record}{: }\Alink{IT::EquationOutput-4}{EquationOutput}}
			{ \it{{\{}u'fields': [u'conc{\_}solid']{\}}} }
			{{{Setting of the fields output.}}}
\end{RecordType}
\begin{RecordType}
	{IT::SorptionImmobile}
	{SorptionImmobile}
	{\Alink{IT::ReactionTermImmobile}{ReactionTermImmobile}} % implements
	{} % reducible to key
	{{{Sorption model in the immobile zone, following the dual porosity model.}}}
		\RecKey
			{SorptionImmobile::TYPE}
			{TYPE}
			{{String}}
			{ \it{Sorptionimmobile} }
			{{{Sub-record Selection.}}}
		\RecKey
			{SorptionImmobile::substances}
			{substances}
			{{Array [1, UINT] of }{String}}
			{ \it{Obligatory} }
			{{{Names of the substances that take part in the sorption model.}}}
		\RecKey
			{SorptionImmobile::solvent-density}
			{solvent{\_}density}
			{{Double [0, +inf)}}
			{ \it{1.0} }
			{{{Density of the solvent.}}}
		\RecKey
			{SorptionImmobile::substeps}
			{substeps}
			{{Integer [1, INT]}}
			{ \it{1000} }
			{{{Number of equidistant substeps, molar mass and isotherm intersections}}}
		\RecKey
			{SorptionImmobile::solubility}
			{solubility}
			{{Array [0, UINT] of }{Double [0, +inf)}}
			{ \it{Optional} }
			{{{Specifies solubility limits of all the sorbing species.}}}
		\RecKey
			{SorptionImmobile::table-limits}
			{table{\_}limits}
			{{Array [0, UINT] of }{Double [0, +inf)}}
			{ \it{Optional} }
			{{{Specifies highest aqueous concentration in interpolation table.}}}
		\RecKey
			{SorptionImmobile::input-fields}
			{input{\_}fields}
			{{Array [0, UINT] of }{Record}{: }\Alink{IT::Sorption-Data}{Sorption{\_}Data}}
			{ \it{Obligatory} }
			{{{Containes region specific data necessary to construct isotherms.}}}
		\RecKey
			{SorptionImmobile::reaction-liquid}
			{reaction{\_}liquid}
			{{Abstract}{: }\Alink{IT::GenericReaction}{GenericReaction}}
			{ \it{Optional} }
			{{{Reaction model following the sorption in the liquid.}}}
		\RecKey
			{SorptionImmobile::reaction-solid}
			{reaction{\_}solid}
			{{Abstract}{: }\Alink{IT::GenericReaction}{GenericReaction}}
			{ \it{Optional} }
			{{{Reaction model following the sorption in the solid.}}}
		\RecKey
			{SorptionImmobile::output}
			{output}
			{{Record}{: }\Alink{IT::EquationOutput-6}{EquationOutput}}
			{ \it{{\{}u'fields': [u'conc{\_}immobile{\_}solid']{\}}} }
			{{{Setting of the fields output.}}}
\end{RecordType}
\begin{RecordType}
	{IT::SorptionMobile}
	{SorptionMobile}
	{\Alink{IT::ReactionTermMobile}{ReactionTermMobile}} % implements
	{} % reducible to key
	{{{Sorption model in the mobile zone, following the dual porosity model.}}}
		\RecKey
			{SorptionMobile::TYPE}
			{TYPE}
			{{String}}
			{ \it{Sorptionmobile} }
			{{{Sub-record Selection.}}}
		\RecKey
			{SorptionMobile::substances}
			{substances}
			{{Array [1, UINT] of }{String}}
			{ \it{Obligatory} }
			{{{Names of the substances that take part in the sorption model.}}}
		\RecKey
			{SorptionMobile::solvent-density}
			{solvent{\_}density}
			{{Double [0, +inf)}}
			{ \it{1.0} }
			{{{Density of the solvent.}}}
		\RecKey
			{SorptionMobile::substeps}
			{substeps}
			{{Integer [1, INT]}}
			{ \it{1000} }
			{{{Number of equidistant substeps, molar mass and isotherm intersections}}}
		\RecKey
			{SorptionMobile::solubility}
			{solubility}
			{{Array [0, UINT] of }{Double [0, +inf)}}
			{ \it{Optional} }
			{{{Specifies solubility limits of all the sorbing species.}}}
		\RecKey
			{SorptionMobile::table-limits}
			{table{\_}limits}
			{{Array [0, UINT] of }{Double [0, +inf)}}
			{ \it{Optional} }
			{{{Specifies highest aqueous concentration in interpolation table.}}}
		\RecKey
			{SorptionMobile::input-fields}
			{input{\_}fields}
			{{Array [0, UINT] of }{Record}{: }\Alink{IT::Sorption-Data}{Sorption{\_}Data}}
			{ \it{Obligatory} }
			{{{Containes region specific data necessary to construct isotherms.}}}
		\RecKey
			{SorptionMobile::reaction-liquid}
			{reaction{\_}liquid}
			{{Abstract}{: }\Alink{IT::GenericReaction}{GenericReaction}}
			{ \it{Optional} }
			{{{Reaction model following the sorption in the liquid.}}}
		\RecKey
			{SorptionMobile::reaction-solid}
			{reaction{\_}solid}
			{{Abstract}{: }\Alink{IT::GenericReaction}{GenericReaction}}
			{ \it{Optional} }
			{{{Reaction model following the sorption in the solid.}}}
		\RecKey
			{SorptionMobile::output}
			{output}
			{{Record}{: }\Alink{IT::EquationOutput-5}{EquationOutput}}
			{ \it{{\{}u'fields': [u'conc{\_}solid']{\}}} }
			{{{Setting of the fields output.}}}
\end{RecordType}
\begin{RecordType}
	{IT::Sorption-Data}
	{Sorption{\_}Data}
	{} % implements
	{} % reducible to key
	{{{Record to set fields of the equation.}\\{
The fields are set only on the domain specified by one of the keys: 'region', 'rid'}\\{
and after the time given by the key 'time'. The field setting can be overridden by}\\{
 any Sorption{\_}Data record that comes later in the boundary data array.}}}
		\RecKey
			{Sorption-Data::region}
			{region}
			{{Array [1, UINT] of }{String}}
			{ \it{Optional} }
			{{{Labels of the regions where to set fields. }}}
		\RecKey
			{Sorption-Data::rid}
			{rid}
			{{Integer [0, INT]}}
			{ \it{Optional} }
			{{{ID of the region where to set fields.}}}
		\RecKey
			{Sorption-Data::time}
			{time}
			{{Double [0, +inf)}}
			{ \it{0.0} }
			{{{Apply field setting in this record after this time.}\\{
These times have to form an increasing sequence.}}}
		\RecKey
			{Sorption-Data::rock-density}
			{rock{\_}density}
			{{Abstract}{: }\Alink{IT::Field-R3---R}{Field:R3 -{\textgreater} R}}
			{ \it{Optional} }
			{{{Rock matrix density. }{$[m^{-3}kg]$}}}
		\RecKey
			{Sorption-Data::sorption-type}
			{sorption{\_}type}
			{{Array [1, UINT] of }{Abstract}{: }\Alink{IT::Field-R3---R-4}{Field:R3 -{\textgreater} R}}
			{ \it{Optional} }
			{{{Considered sorption is described by selected isotherm. If porosity on an element is equal or even higher than 1.0 (meaning no sorbing surface), then type 'none' will be selected automatically. }{$[-]$}}}
		\RecKey
			{Sorption-Data::isotherm-mult}
			{isotherm{\_}mult}
			{{Array [1, UINT] of }{Abstract}{: }\Alink{IT::Field-R3---R}{Field:R3 -{\textgreater} R}}
			{ \it{Optional} }
			{{{Multiplication parameters (k, omega) in either Langmuir c{\_}s = omega * (alpha}\textit{c{\_}a)/(1- alpha}{c{\_}a) or in linear c{\_}s = k * c{\_}a isothermal description. }{$[kg^{-1}mol]$}}}
		\RecKey
			{Sorption-Data::isotherm-other}
			{isotherm{\_}other}
			{{Array [1, UINT] of }{Abstract}{: }\Alink{IT::Field-R3---R}{Field:R3 -{\textgreater} R}}
			{ \it{Optional} }
			{{{Second parameters (alpha, ...) defining isotherm  c{\_}s = omega * (alpha}\textit{c{\_}a)/(1- alpha}{c{\_}a). }{$[-]$}}}
		\RecKey
			{Sorption-Data::init-conc-solid}
			{init{\_}conc{\_}solid}
			{{Array [1, UINT] of }{Abstract}{: }\Alink{IT::Field-R3---R}{Field:R3 -{\textgreater} R}}
			{ \it{Optional} }
			{{{Initial solid concentration of substances. Vector, one value for every substance. }{$[kg^{-1}mol]$}}}
\end{RecordType}
\begin{RecordType}
	{IT::Substance}
	{Substance}
	{} % implements
	{\Alink{Substance::name}{name}} % reducible to key
	{{{Chemical substance.}}}
		\RecKey
			{Substance::name}
			{name}
			{{String}}
			{ \it{Obligatory} }
			{{{Name of the substance.}}}
		\RecKey
			{Substance::molar-mass}
			{molar{\_}mass}
			{{Double [0, +inf)}}
			{ \it{1} }
			{{{Molar mass of the substance [kg/mol].}}}
\end{RecordType}
\begin{RecordType}
	{IT::TableFunction-7}
	{TableFunction}
	{} % implements
	{\Alink{TableFunction-7::values}{values}} % reducible to key
	{{{Allow set variable series initialization of Fields.}}}
		\RecKey
			{TableFunction-7::values}
			{values}
			{{Array [2, UINT] of }{Tuple}{: }\Alink{IT::IndependentValue-7}{IndependentValue}}
			{ \it{Obligatory} }
			{{{Initizaliation values of Field.}}}
\end{RecordType}
\begin{RecordType}
	{IT::TableFunction-8}
	{TableFunction}
	{} % implements
	{\Alink{TableFunction-8::values}{values}} % reducible to key
	{{{Allow set variable series initialization of Fields.}}}
		\RecKey
			{TableFunction-8::values}
			{values}
			{{Array [2, UINT] of }{Tuple}{: }\Alink{IT::IndependentValue-8}{IndependentValue}}
			{ \it{Obligatory} }
			{{{Initizaliation values of Field.}}}
\end{RecordType}
\begin{RecordType}
	{IT::TimeGovernor}
	{TimeGovernor}
	{} % implements
	{\Alink{TimeGovernor::max-dt}{max{\_}dt}} % reducible to key
	{{{Setting of the simulation time. (can be specific to one equation)}}}
		\RecKey
			{TimeGovernor::start-time}
			{start{\_}time}
			{{Double (-inf, +inf)}}
			{ \it{0.0} }
			{{{Start time of the simulation.}}}
		\RecKey
			{TimeGovernor::end-time}
			{end{\_}time}
			{{Double (-inf, +inf)}}
			{ \it{5e+17} }
			{{{End time of the simulation. Default value is more then age of universe in seconds.}}}
		\RecKey
			{TimeGovernor::init-dt}
			{init{\_}dt}
			{{Double [0, +inf)}}
			{ \it{0.0} }
			{{{Initial guess for the time step.}\\{
Only useful for equations that use adaptive time stepping.If set to 0.0, the time step is determined in fully autonomous way if the equation supports it.}}}
		\RecKey
			{TimeGovernor::min-dt}
			{min{\_}dt}
			{{Double [0, +inf)}}
			{implicit value: "{Machine precision.}"}
			{{{Soft lower limit for the time step. Equation using adaptive time stepping can notsuggest smaller time step, but actual time step could be smaller in order to match prescribed input or output times.}}}
		\RecKey
			{TimeGovernor::max-dt}
			{max{\_}dt}
			{{Double [0, +inf)}}
			{implicit value: "{Whole time of the simulation if specified, infinity else.}"}
			{{{Hard upper limit for the time step. Actual length of the time step is also limitedby input and output times.}}}
\end{RecordType}
\begin{RecordType}
	{IT::TimeGrid}
	{TimeGrid}
	{} % implements
	{\Alink{TimeGrid::begin}{begin}} % reducible to key
	{{{Equally spaced grid of time points.}}}
		\RecKey
			{TimeGrid::begin}
			{begin}
			{{Double [0, +inf)}}
			{implicit value: "{The initial time of the associated equation.}"}
			{{{The start time of the grid.}}}
		\RecKey
			{TimeGrid::step}
			{step}
			{{Double [0, +inf)}}
			{ \it{Optional} }
			{{{The step of the grid. If not specified, the grid consists only of the start time.}}}
		\RecKey
			{TimeGrid::end}
			{end}
			{{Double [0, +inf)}}
			{implicit value: "{The end time of the simulation.}"}
			{{{The time greater or equal to the last time in the grid.}}}
\end{RecordType}
\begin{RecordType}
	{IT::Union}
	{Union}
	{\Alink{IT::Region}{Region}} % implements
	{} % reducible to key
	{{{Defines region as a union of given two or more regions.}\\{
Regions can be given by names or IDs or both ways together.}}}
		\RecKey
			{Union::TYPE}
			{TYPE}
			{{String}}
			{ \it{Union} }
			{{{Sub-record Selection.}}}
		\RecKey
			{Union::name}
			{name}
			{{String}}
			{ \it{Obligatory} }
			{{{Label (name) of the region. Has to be unique in one mesh.}}}
		\RecKey
			{Union::region-ids}
			{region{\_}ids}
			{{Array [0, UINT] of }{Integer [0, INT]}}
			{ \it{Optional} }
			{{{List of region ID numbers that has to be added to the region set.}}}
		\RecKey
			{Union::regions}
			{regions}
			{{Array [0, UINT] of }{String}}
			{ \it{Optional} }
			{{{Defines region as a union of given pair of regions.}}}
\end{RecordType}
\begin{RecordType}
	{IT::Unit}
	{Unit}
	{} % implements
	{\Alink{Unit::unit-formula}{unit{\_}formula}} % reducible to key
	{{{Specify unit of an input value. }\\{
Evaluation of the unit formula results into a coeficient and a unit in terms of powers of base SI units. The unit must match expected SI unit of the value, while the value provided on the input is multiplied by the coefficient before further processing.The unit formula have form: {\textless}UnitExpr{\textgreater};{\textless}Variable{\textgreater}={\textless}Number{\textgreater}}\textit{{\textless}UnitExpr{\textgreater};..., where {\textless}Variable{\textgreater} is a variable name and {\textless}UnitExpr{\textgreater} is a units expression which consists of products and divisions of terms, where a term has form {\textless}Base{\textgreater}{\^{}}{\textless}N{\textgreater}, where {\textless}N{\textgreater} is an integer exponent and {\textless}Base{\textgreater} is either a base SI unit, a derived unit, or a variable defined in the same unit formula.Example, unit for the pressure head: 'MPa/rho/g{\_}; rho = 990}{kg}\textit{m{\^{}}-3; g{\_} = 9.8}{m*s{\^{}}-2'}}}
		\RecKey
			{Unit::unit-formula}
			{unit{\_}formula}
			{{String}}
			{ \it{Obligatory} }
			{{{Definition of unit.}}}
\end{RecordType}
\begin{RecordType}
	{IT::gmsh}
	{gmsh}
	{\Alink{IT::OutputTime}{OutputTime}} % implements
	{} % reducible to key
	{{{Parameters of gmsh output format.}}}
		\RecKey
			{gmsh::TYPE}
			{TYPE}
			{{String}}
			{ \it{Gmsh} }
			{{{Sub-record Selection.}}}
\end{RecordType}
\begin{RecordType}
	{IT::vtk}
	{vtk}
	{\Alink{IT::OutputTime}{OutputTime}} % implements
	{} % reducible to key
	{{{Parameters of vtk output format.}}}
		\RecKey
			{vtk::TYPE}
			{TYPE}
			{{String}}
			{ \it{Vtk} }
			{{{Sub-record Selection.}}}
		\RecKey
			{vtk::variant}
			{variant}
			{{Selection}{: }\Alink{IT::VTK-variant-ascii-or-binary-}{VTK variant (ascii or binary)}}
			{ \it{Ascii} }
			{{{Variant of output stream file format.}}}
		\RecKey
			{vtk::parallel}
			{parallel}
			{{Bool}}
			{ \it{False} }
			{{{Parallel or serial version of file format.}}}
\end{RecordType}
\begin{AbstractType}
	{IT::AdvectionProcess}
	{AdvectionProcess}
	{}
	{{{Abstract advection process. In particular: transport of substances or heat transfer.}}}
		\Descendant{\Alink{IT::Coupling-OperatorSplitting}{Coupling{\_}OperatorSplitting}}
		\Descendant{\Alink{IT::Heat-AdvectionDiffusion-DG}{Heat{\_}AdvectionDiffusion{\_}DG}}
\end{AbstractType}
\begin{AbstractType}
	{IT::Coupling-Base}
	{Coupling{\_}Base}
	{}
	{{{The root record of description of particular the problem to solve.}}}
		\Descendant{\Alink{IT::Coupling-Sequential}{Coupling{\_}Sequential}}
\end{AbstractType}
\begin{AbstractType}
	{IT::DarcyFlow}
	{DarcyFlow}
	{}
	{{{Darcy flow model. Abstraction of various porous media flow models.}}}
		\Descendant{\Alink{IT::Flow-Darcy-MH}{Flow{\_}Darcy{\_}MH}}
		\Descendant{\Alink{IT::Flow-Richards-LMH}{Flow{\_}Richards{\_}LMH}}
\end{AbstractType}
\begin{AbstractType}
	{IT::EmptyAbstract}
	{EmptyAbstract}
	{}
	{}
\end{AbstractType}
\begin{AbstractType}
	{IT::Field-R3---R-6}
	{Field:R3 -{\textgreater} R}
	{\Alink{IT::FieldConstant-7}{FieldConstant}}
	{{{Abstract for all time-space functions.}}}
		\Descendant{\Alink{IT::FieldPython-2}{FieldPython}}
		\Descendant{\Alink{IT::FieldConstant-7}{FieldConstant}}
		\Descendant{\Alink{IT::FieldFormula-2}{FieldFormula}}
		\Descendant{\Alink{IT::FieldElementwise-2}{FieldElementwise}}
		\Descendant{\Alink{IT::FieldInterpolatedP0-2}{FieldInterpolatedP0}}
		\Descendant{\Alink{IT::FieldTimeFunction-7}{FieldTimeFunction}}
\end{AbstractType}
\begin{AbstractType}
	{IT::Field-R3---R-3-3--2}
	{Field:R3 -{\textgreater} R[3,3]}
	{\Alink{IT::FieldConstant-8}{FieldConstant}}
	{{{Abstract for all time-space functions.}}}
		\Descendant{\Alink{IT::FieldPython}{FieldPython}}
		\Descendant{\Alink{IT::FieldConstant-8}{FieldConstant}}
		\Descendant{\Alink{IT::FieldFormula}{FieldFormula}}
		\Descendant{\Alink{IT::FieldElementwise}{FieldElementwise}}
		\Descendant{\Alink{IT::FieldInterpolatedP0}{FieldInterpolatedP0}}
		\Descendant{\Alink{IT::FieldTimeFunction-8}{FieldTimeFunction}}
\end{AbstractType}
\begin{AbstractType}
	{IT::GenericReaction}
	{GenericReaction}
	{}
	{{{Abstract equation for a reaction of species in single compartment (e.g. mobile solid).It can be part of: direct operator splitting coupling, dual porosity model, any sorption.}}}
		\Descendant{\Alink{IT::FirstOrderReaction}{FirstOrderReaction}}
		\Descendant{\Alink{IT::RadioactiveDecay}{RadioactiveDecay}}
\end{AbstractType}
\begin{AbstractType}
	{IT::LinSys}
	{LinSys}
	{\Alink{IT::Petsc}{Petsc}}
	{{{Linear solver setting.}}}
		\Descendant{\Alink{IT::Petsc}{Petsc}}
		\Descendant{\Alink{IT::Bddc}{Bddc}}
\end{AbstractType}
\begin{AbstractType}
	{IT::OutputTime}
	{OutputTime}
	{\Alink{IT::vtk}{vtk}}
	{{{Format of output stream and possible parameters.}}}
		\Descendant{\Alink{IT::vtk}{vtk}}
		\Descendant{\Alink{IT::gmsh}{gmsh}}
\end{AbstractType}
\begin{AbstractType}
	{IT::ReactionTerm}
	{ReactionTerm}
	{}
	{{{Abstract equation for a reaction term (dual porosity, sorption, reactions). Can be part of coupling with a transport equation via. operator splitting.}}}
		\Descendant{\Alink{IT::FirstOrderReaction}{FirstOrderReaction}}
		\Descendant{\Alink{IT::RadioactiveDecay}{RadioactiveDecay}}
		\Descendant{\Alink{IT::Sorption}{Sorption}}
		\Descendant{\Alink{IT::DualPorosity}{DualPorosity}}
\end{AbstractType}
\begin{AbstractType}
	{IT::ReactionTermImmobile}
	{ReactionTermImmobile}
	{}
	{{{Abstract equation for a reaction term of the IMMOBILE pores (sorption, reactions). Is part of dual porosity model.}}}
		\Descendant{\Alink{IT::FirstOrderReaction}{FirstOrderReaction}}
		\Descendant{\Alink{IT::RadioactiveDecay}{RadioactiveDecay}}
		\Descendant{\Alink{IT::SorptionImmobile}{SorptionImmobile}}
\end{AbstractType}
\begin{AbstractType}
	{IT::ReactionTermMobile}
	{ReactionTermMobile}
	{}
	{{{Abstract equation for a reaction term of the MOBILE pores (sorption, reactions). Is part of dual porosity model.}}}
		\Descendant{\Alink{IT::FirstOrderReaction}{FirstOrderReaction}}
		\Descendant{\Alink{IT::RadioactiveDecay}{RadioactiveDecay}}
		\Descendant{\Alink{IT::SorptionMobile}{SorptionMobile}}
\end{AbstractType}
\begin{AbstractType}
	{IT::Region}
	{Region}
	{}
	{{{Abstract record for Region.}}}
		\Descendant{\Alink{IT::From-Id}{From{\_}Id}}
		\Descendant{\Alink{IT::From-Label}{From{\_}Label}}
		\Descendant{\Alink{IT::From-Elements}{From{\_}Elements}}
		\Descendant{\Alink{IT::Union}{Union}}
		\Descendant{\Alink{IT::Difference}{Difference}}
		\Descendant{\Alink{IT::Intersection}{Intersection}}
\end{AbstractType}
\begin{AbstractType}
	{IT::Solute}
	{Solute}
	{}
	{{{Transport of soluted  substances.}}}
		\Descendant{\Alink{IT::Solute-Advection-FV}{Solute{\_}Advection{\_}FV}}
		\Descendant{\Alink{IT::Solute-AdvectionDiffusion-DG}{Solute{\_}AdvectionDiffusion{\_}DG}}
\end{AbstractType}
\begin{SelectionType}
	{IT::Balance-output-format}
	{Balance{\_}output{\_}format}
	{{{Format of output file for balance.}}}
		\SelectionItem
			{Balance-output-format::legacy}
			{legacy}
			{{{Legacy format used by previous program versions.}}}
		\SelectionItem
			{Balance-output-format::txt}
			{txt}
			{{{Excel format with tab delimiter.}}}
		\SelectionItem
			{Balance-output-format::gnuplot}
			{gnuplot}
			{{{Format compatible with GnuPlot datafile with fixed column width.}}}
\end{SelectionType}
\begin{SelectionType}
	{IT::DG-variant}
	{DG{\_}variant}
	{{{Type of penalty term.}}}
		\SelectionItem
			{DG-variant::non-symmetric}
			{non-symmetric}
			{{{non-symmetric weighted interior penalty DG method}}}
		\SelectionItem
			{DG-variant::incomplete}
			{incomplete}
			{{{incomplete weighted interior penalty DG method}}}
		\SelectionItem
			{DG-variant::symmetric}
			{symmetric}
			{{{symmetric weighted interior penalty DG method}}}
\end{SelectionType}
\begin{SelectionType}
	{IT::DualPorosity-output-fields}
	{DualPorosity{\_}output{\_}fields}
	{{{Selection of output fields for the DualPorosity model.}}}
		\SelectionItem
			{DualPorosity-output-fields::diffusion-rate-immobile}
			{diffusion{\_}rate{\_}immobile}
			{{{Output of the field diffusion{\_}rate{\_}immobile }{$[s^{-1}]$}{ (Diffusion coefficient of non-equilibrium linear exchange between mobile and immobile zone.).}}}
		\SelectionItem
			{DualPorosity-output-fields::porosity-immobile}
			{porosity{\_}immobile}
			{{{Output of the field porosity{\_}immobile }{$[-]$}{ (Porosity of the immobile zone.).}}}
		\SelectionItem
			{DualPorosity-output-fields::init-conc-immobile}
			{init{\_}conc{\_}immobile}
			{{{Output of the field init{\_}conc{\_}immobile }{$[m^{-3}kg]$}{ (Initial concentration of substances in the immobile zone.).}}}
		\SelectionItem
			{DualPorosity-output-fields::conc-immobile}
			{conc{\_}immobile}
			{{{Output of the field conc{\_}immobile }{$[m^{-3}kg]$}{.}}}
\end{SelectionType}
\begin{SelectionType}
	{IT::EmptySelection}
	{EmptySelection}
	{}
\end{SelectionType}
\begin{SelectionType}
	{IT::Flow-Darcy-BC-Type}
	{Flow{\_}Darcy{\_}BC{\_}Type}
	{}
		\SelectionItem
			{Flow-Darcy-BC-Type::none}
			{none}
			{{{Homogeneous Neumann boundary condition. Zero flux}}}
		\SelectionItem
			{Flow-Darcy-BC-Type::dirichlet}
			{dirichlet}
			{{{Dirichlet boundary condition. Specify the pressure head through the 'bc{\_}pressure' field or the piezometric head through the 'bc{\_}piezo{\_}head' field.}}}
		\SelectionItem
			{Flow-Darcy-BC-Type::total-flux}
			{total{\_}flux}
			{{{Flux boundary condition (combines Neumann and Robin type). Water inflow equal to }{$q^N + \sigma (h^R - h)$}{. Specify the water inflow by the 'bc{\_}flux' field, the transition coefficient by 'bc{\_}robin{\_}sigma' and the reference pressure head or pieozmetric head through 'bc{\_}pressure' or 'bc{\_}piezo{\_}head' respectively.}}}
		\SelectionItem
			{Flow-Darcy-BC-Type::seepage}
			{seepage}
			{{{Seepage face boundary condition. Pressure and inflow bounded from above. Boundary with potential seepage flow is described by the pair of inequalities:}{$h \le h_d^D$}{ and }{$ q \le q_d^N$}{, where the equality holds in at least one of them. Caution! Setting {\$}q{\_}d{\^{}}N{\$} strictly negativemay lead to an ill posed problem since a positive outflow is enforced.Parameters }{$h_d^D$}{ and }{$q_d^N$}{ are given by fields }\ttfamily bc{\_}pressure\ttfamily bc{\_}piezo{\_}head\ttfamily bc{\_}flux}}
		\SelectionItem
			{Flow-Darcy-BC-Type::river}
			{river}
			{{{River boundary condition. For the water level above the bedrock, }{$H > H^S$}{, the Robin boundary condition is used with the inflow given by: }{ $q^N + \sigma(H^D - H)$}{. For the water level under the bedrock, constant infiltration is used }{ $q^N + \sigma(H^D - H^S)$}{. Parameters: }\ttfamily bc{\_}pressure\ttfamily bc{\_}switch{\_}pressure\ttfamily bc{\_}sigma,}}
\end{SelectionType}
\begin{SelectionType}
	{IT::Flow-Darcy-MH-output-fields}
	{Flow{\_}Darcy{\_}MH{\_}output{\_}fields}
	{{{Selection of output fields for the Flow{\_}Darcy{\_}MH model.}}}
		\SelectionItem
			{Flow-Darcy-MH-output-fields::pressure-p0}
			{pressure{\_}p0}
			{{{Output of the field pressure{\_}p0 }{$[m]$}{.}}}
		\SelectionItem
			{Flow-Darcy-MH-output-fields::pressure-p1}
			{pressure{\_}p1}
			{{{Output of the field pressure{\_}p1 }{$[m]$}{.}}}
		\SelectionItem
			{Flow-Darcy-MH-output-fields::piezo-head-p0}
			{piezo{\_}head{\_}p0}
			{{{Output of the field piezo{\_}head{\_}p0 }{$[m]$}{.}}}
		\SelectionItem
			{Flow-Darcy-MH-output-fields::velocity-p0}
			{velocity{\_}p0}
			{{{Output of the field velocity{\_}p0 }{$[ms^{-1}]$}{.}}}
		\SelectionItem
			{Flow-Darcy-MH-output-fields::subdomain}
			{subdomain}
			{{{Output of the field subdomain }{$[-]$}{.}}}
		\SelectionItem
			{Flow-Darcy-MH-output-fields::region-id}
			{region{\_}id}
			{{{Output of the field region{\_}id }{$[-]$}{.}}}
		\SelectionItem
			{Flow-Darcy-MH-output-fields::anisotropy}
			{anisotropy}
			{{{Output of the field anisotropy }{$[-]$}{ (Anisotropy of the conductivity tensor.).}}}
		\SelectionItem
			{Flow-Darcy-MH-output-fields::cross-section}
			{cross{\_}section}
			{{{Output of the field cross{\_}section }{$[m^{3-d}]$}{ (Complement dimension parameter (cross section for 1D, thickness for 2D).).}}}
		\SelectionItem
			{Flow-Darcy-MH-output-fields::conductivity}
			{conductivity}
			{{{Output of the field conductivity }{$[ms^{-1}]$}{ (Isotropic conductivity scalar.).}}}
		\SelectionItem
			{Flow-Darcy-MH-output-fields::sigma}
			{sigma}
			{{{Output of the field sigma }{$[-]$}{ (Transition coefficient between dimensions.).}}}
		\SelectionItem
			{Flow-Darcy-MH-output-fields::water-source-density}
			{water{\_}source{\_}density}
			{{{Output of the field water{\_}source{\_}density }{$[s^{-1}]$}{ (Water source density.).}}}
		\SelectionItem
			{Flow-Darcy-MH-output-fields::init-pressure}
			{init{\_}pressure}
			{{{Output of the field init{\_}pressure }{$[m]$}{ (Initial condition as pressure).}}}
		\SelectionItem
			{Flow-Darcy-MH-output-fields::storativity}
			{storativity}
			{{{Output of the field storativity }{$[m^{-1}]$}{ (Storativity.).}}}
		\SelectionItem
			{Flow-Darcy-MH-output-fields::pressure-diff}
			{pressure{\_}diff}
			{{{Output of the field pressure{\_}diff }{$[m]$}{.}}}
		\SelectionItem
			{Flow-Darcy-MH-output-fields::velocity-diff}
			{velocity{\_}diff}
			{{{Output of the field velocity{\_}diff }{$[ms^{-1}]$}{.}}}
		\SelectionItem
			{Flow-Darcy-MH-output-fields::div-diff}
			{div{\_}diff}
			{{{Output of the field div{\_}diff }{$[s^{-1}]$}{.}}}
\end{SelectionType}
\begin{SelectionType}
	{IT::GraphType}
	{GraphType}
	{{{Different algorithms to make the sparse graph with weighted edges}\\{
from the multidimensional mesh. Main difference is dealing with }\\{
neighborings of elements of different dimension.}}}
		\SelectionItem
			{GraphType::any-neighboring}
			{any{\_}neighboring}
			{{{Add edge for any pair of neighboring elements.}}}
		\SelectionItem
			{GraphType::any-wight-lower-dim-cuts}
			{any{\_}wight{\_}lower{\_}dim{\_}cuts}
			{{{Same as before and assign higher weight to cuts of lower dimension in order to make them stick to one face.}}}
		\SelectionItem
			{GraphType::same-dimension-neghboring}
			{same{\_}dimension{\_}neghboring}
			{{{Add edge for any pair of neighboring elements of same dimension (bad for matrix multiply).}}}
\end{SelectionType}
\begin{SelectionType}
	{IT::Heat-AdvectionDiffusion-DG-output-fields}
	{Heat{\_}AdvectionDiffusion{\_}DG{\_}output{\_}fields}
	{{{Selection of output fields for the Heat{\_}AdvectionDiffusion{\_}DG model.}}}
		\SelectionItem
			{Heat-AdvectionDiffusion-DG-output-fields::init-temperature}
			{init{\_}temperature}
			{{{Output of the field init{\_}temperature }{$[K]$}{ (Initial temperature.).}}}
		\SelectionItem
			{Heat-AdvectionDiffusion-DG-output-fields::porosity}
			{porosity}
			{{{Output of the field porosity }{$[-]$}{ (Porosity.).}}}
		\SelectionItem
			{Heat-AdvectionDiffusion-DG-output-fields::water-content}
			{water{\_}content}
			{{{Output of the field water{\_}content }{$[-]$}{.}}}
		\SelectionItem
			{Heat-AdvectionDiffusion-DG-output-fields::fluid-density}
			{fluid{\_}density}
			{{{Output of the field fluid{\_}density }{$[m^{-3}kg]$}{ (Density of fluid.).}}}
		\SelectionItem
			{Heat-AdvectionDiffusion-DG-output-fields::fluid-heat-capacity}
			{fluid{\_}heat{\_}capacity}
			{{{Output of the field fluid{\_}heat{\_}capacity }{$[m^{2}s^{-2}K^{-1}]$}{ (Heat capacity of fluid.).}}}
		\SelectionItem
			{Heat-AdvectionDiffusion-DG-output-fields::fluid-heat-conductivity}
			{fluid{\_}heat{\_}conductivity}
			{{{Output of the field fluid{\_}heat{\_}conductivity }{$[mkgs^{-3}K^{-1}]$}{ (Heat conductivity of fluid.).}}}
		\SelectionItem
			{Heat-AdvectionDiffusion-DG-output-fields::solid-density}
			{solid{\_}density}
			{{{Output of the field solid{\_}density }{$[m^{-3}kg]$}{ (Density of solid (rock).).}}}
		\SelectionItem
			{Heat-AdvectionDiffusion-DG-output-fields::solid-heat-capacity}
			{solid{\_}heat{\_}capacity}
			{{{Output of the field solid{\_}heat{\_}capacity }{$[m^{2}s^{-2}K^{-1}]$}{ (Heat capacity of solid (rock).).}}}
		\SelectionItem
			{Heat-AdvectionDiffusion-DG-output-fields::solid-heat-conductivity}
			{solid{\_}heat{\_}conductivity}
			{{{Output of the field solid{\_}heat{\_}conductivity }{$[mkgs^{-3}K^{-1}]$}{ (Heat conductivity of solid (rock).).}}}
		\SelectionItem
			{Heat-AdvectionDiffusion-DG-output-fields::disp-l}
			{disp{\_}l}
			{{{Output of the field disp{\_}l }{$[m]$}{ (Longitudal heat dispersivity in fluid.).}}}
		\SelectionItem
			{Heat-AdvectionDiffusion-DG-output-fields::disp-t}
			{disp{\_}t}
			{{{Output of the field disp{\_}t }{$[m]$}{ (Transversal heat dispersivity in fluid.).}}}
		\SelectionItem
			{Heat-AdvectionDiffusion-DG-output-fields::fluid-thermal-source}
			{fluid{\_}thermal{\_}source}
			{{{Output of the field fluid{\_}thermal{\_}source }{$[m^{-1}kgs^{-3}]$}{ (Thermal source density in fluid.).}}}
		\SelectionItem
			{Heat-AdvectionDiffusion-DG-output-fields::solid-thermal-source}
			{solid{\_}thermal{\_}source}
			{{{Output of the field solid{\_}thermal{\_}source }{$[m^{-1}kgs^{-3}]$}{ (Thermal source density in solid.).}}}
		\SelectionItem
			{Heat-AdvectionDiffusion-DG-output-fields::fluid-heat-exchange-rate}
			{fluid{\_}heat{\_}exchange{\_}rate}
			{{{Output of the field fluid{\_}heat{\_}exchange{\_}rate }{$[s^{-1}]$}{ (Heat exchange rate in fluid.).}}}
		\SelectionItem
			{Heat-AdvectionDiffusion-DG-output-fields::solid-heat-exchange-rate}
			{solid{\_}heat{\_}exchange{\_}rate}
			{{{Output of the field solid{\_}heat{\_}exchange{\_}rate }{$[s^{-1}]$}{ (Heat exchange rate of source in solid.).}}}
		\SelectionItem
			{Heat-AdvectionDiffusion-DG-output-fields::fluid-ref-temperature}
			{fluid{\_}ref{\_}temperature}
			{{{Output of the field fluid{\_}ref{\_}temperature }{$[K]$}{ (Reference temperature of source in fluid.).}}}
		\SelectionItem
			{Heat-AdvectionDiffusion-DG-output-fields::solid-ref-temperature}
			{solid{\_}ref{\_}temperature}
			{{{Output of the field solid{\_}ref{\_}temperature }{$[K]$}{ (Reference temperature in solid.).}}}
		\SelectionItem
			{Heat-AdvectionDiffusion-DG-output-fields::temperature}
			{temperature}
			{{{Output of the field temperature }{$[K]$}{.}}}
		\SelectionItem
			{Heat-AdvectionDiffusion-DG-output-fields::fracture-sigma}
			{fracture{\_}sigma}
			{{{Output of the field fracture{\_}sigma }{$[-]$}{ (Coefficient of diffusive transfer through fractures (for each substance).).}}}
		\SelectionItem
			{Heat-AdvectionDiffusion-DG-output-fields::dg-penalty}
			{dg{\_}penalty}
			{{{Output of the field dg{\_}penalty }{$[-]$}{ (Penalty parameter influencing the discontinuity of the solution (for each substance). Its default value 1 is sufficient in most cases. Higher value diminishes the inter-element jumps.).}}}
		\SelectionItem
			{Heat-AdvectionDiffusion-DG-output-fields::region-id}
			{region{\_}id}
			{{{Output of the field region{\_}id }{$[-]$}{.}}}
\end{SelectionType}
\begin{SelectionType}
	{IT::Heat-BC-Type}
	{Heat{\_}BC{\_}Type}
	{{{Types of boundary conditions for heat transfer model.}}}
		\SelectionItem
			{Heat-BC-Type::inflow}
			{inflow}
			{{{Default heat transfer boundary condition.}\\{
On water inflow }{$(q_w \le 0)$}{, total energy flux is given by the reference temperature 'bc{\_}temperature'. On water outflow we prescribe zero diffusive flux, i.e. the energy flows out only due to advection.}}}
		\SelectionItem
			{Heat-BC-Type::dirichlet}
			{dirichlet}
			{{{Dirichlet boundary condition }{$T = T_D $}{.}\\{
The prescribed temperature }{$T_D$}{ is specified by the field 'bc{\_}temperature'.}}}
		\SelectionItem
			{Heat-BC-Type::total-flux}
			{total{\_}flux}
			{{{Total energy flux boundary condition.}\\{
The prescribed incoming total flux can have the general form }{$\delta(f_N+\sigma_R(T_R-T) )$}{, where the absolute flux }{$f_N$}{ is specified by the field 'bc{\_}flux', the transition parameter }{$\sigma_R$}{ by 'bc{\_}robin{\_}sigma', and the reference temperature }{$T_R$}{ by 'bc{\_}temperature'.}}}
		\SelectionItem
			{Heat-BC-Type::diffusive-flux}
			{diffusive{\_}flux}
			{{{Diffusive flux boundary condition.}\\{
The prescribed incoming energy flux due to diffusion can have the general form }{$\delta(f_N+\sigma_R(T_R-T) )$}{, where the absolute flux }{$f_N$}{ is specified by the field 'bc{\_}flux', the transition parameter }{$\sigma_R$}{ by 'bc{\_}robin{\_}sigma', and the reference temperature }{$T_R$}{ by 'bc{\_}temperature'.}}}
\end{SelectionType}
\begin{SelectionType}
	{IT::MH-MortarMethod}
	{MH{\_}MortarMethod}
	{}
		\SelectionItem
			{MH-MortarMethod::None}
			{None}
			{{{Mortar space: P0 on elements of lower dimension.}}}
		\SelectionItem
			{MH-MortarMethod::P0}
			{P0}
			{{{Mortar space: P0 on elements of lower dimension.}}}
		\SelectionItem
			{MH-MortarMethod::P1}
			{P1}
			{{{Mortar space: P1 on intersections, using non-conforming pressures.}}}
\end{SelectionType}
\begin{SelectionType}
	{IT::PartTool}
	{PartTool}
	{{{Select the partitioning tool to use.}}}
		\SelectionItem
			{PartTool::PETSc}
			{PETSc}
			{{{Use PETSc interface to various partitioning tools.}}}
		\SelectionItem
			{PartTool::METIS}
			{METIS}
			{{{Use direct interface to Metis.}}}
\end{SelectionType}
\begin{SelectionType}
	{IT::Soil-Model-Type}
	{Soil{\_}Model{\_}Type}
	{}
		\SelectionItem
			{Soil-Model-Type::van-genuchten}
			{van{\_}genuchten}
			{{{Van Genuchten soil model with cutting near zero.}}}
		\SelectionItem
			{Soil-Model-Type::irmay}
			{irmay}
			{{{Irmay model for conductivity, Van Genuchten model for the water content. Suitable for bentonite.}}}
\end{SelectionType}
\begin{SelectionType}
	{IT::Solute-AdvectionDiffusion-BC-Type}
	{Solute{\_}AdvectionDiffusion{\_}BC{\_}Type}
	{{{Types of boundary conditions for advection-diffusion solute transport model.}}}
		\SelectionItem
			{Solute-AdvectionDiffusion-BC-Type::inflow}
			{inflow}
			{{{Default transport boundary condition.}\\{
On water inflow }{$(q_w \le 0)$}{, total flux is given by the reference concentration 'bc{\_}conc'. On water outflow we prescribe zero diffusive flux, i.e. the mass flows out only due to advection.}}}
		\SelectionItem
			{Solute-AdvectionDiffusion-BC-Type::dirichlet}
			{dirichlet}
			{{{Dirichlet boundary condition }{$ c = c_D $}{.}\\{
The prescribed concentration }{$c_D$}{ is specified by the field 'bc{\_}conc'.}}}
		\SelectionItem
			{Solute-AdvectionDiffusion-BC-Type::total-flux}
			{total{\_}flux}
			{{{Total mass flux boundary condition.}\\{
The prescribed total incoming flux can have the general form }{$\delta(f_N+\sigma_R(c_R-c) )$}{, where the absolute flux }{$f_N$}{ is specified by the field 'bc{\_}flux', the transition parameter }{$\sigma_R$}{ by 'bc{\_}robin{\_}sigma', and the reference concentration }{$c_R$}{ by 'bc{\_}conc'.}}}
		\SelectionItem
			{Solute-AdvectionDiffusion-BC-Type::diffusive-flux}
			{diffusive{\_}flux}
			{{{Diffusive flux boundary condition.}\\{
The prescribed incoming mass flux due to diffusion can have the general form }{$\delta(f_N+\sigma_R(c_R-c) )$}{, where the absolute flux }{$f_N$}{ is specified by the field 'bc{\_}flux', the transition parameter }{$\sigma_R$}{ by 'bc{\_}robin{\_}sigma', and the reference concentration }{$c_R$}{ by 'bc{\_}conc'.}}}
\end{SelectionType}
\begin{SelectionType}
	{IT::Solute-AdvectionDiffusion-DG-output-fields}
	{Solute{\_}AdvectionDiffusion{\_}DG{\_}output{\_}fields}
	{{{Selection of output fields for the Solute{\_}AdvectionDiffusion{\_}DG model.}}}
		\SelectionItem
			{Solute-AdvectionDiffusion-DG-output-fields::porosity}
			{porosity}
			{{{Output of the field porosity }{$[-]$}{ (Mobile porosity).}}}
		\SelectionItem
			{Solute-AdvectionDiffusion-DG-output-fields::water-content}
			{water{\_}content}
			{{{Output of the field water{\_}content }{$[-]$}{ (INTERNAL - water content passed from unsaturated Darcy).}}}
		\SelectionItem
			{Solute-AdvectionDiffusion-DG-output-fields::sources-density}
			{sources{\_}density}
			{{{Output of the field sources{\_}density }{$[m^{-3}kgs^{-1}]$}{ (Density of concentration sources.).}}}
		\SelectionItem
			{Solute-AdvectionDiffusion-DG-output-fields::sources-sigma}
			{sources{\_}sigma}
			{{{Output of the field sources{\_}sigma }{$[s^{-1}]$}{ (Concentration flux.).}}}
		\SelectionItem
			{Solute-AdvectionDiffusion-DG-output-fields::sources-conc}
			{sources{\_}conc}
			{{{Output of the field sources{\_}conc }{$[m^{-3}kg]$}{ (Concentration sources threshold.).}}}
		\SelectionItem
			{Solute-AdvectionDiffusion-DG-output-fields::init-conc}
			{init{\_}conc}
			{{{Output of the field init{\_}conc }{$[m^{-3}kg]$}{ (Initial concentrations.).}}}
		\SelectionItem
			{Solute-AdvectionDiffusion-DG-output-fields::disp-l}
			{disp{\_}l}
			{{{Output of the field disp{\_}l }{$[m]$}{ (Longitudal dispersivity (for each substance).).}}}
		\SelectionItem
			{Solute-AdvectionDiffusion-DG-output-fields::disp-t}
			{disp{\_}t}
			{{{Output of the field disp{\_}t }{$[m]$}{ (Transversal dispersivity (for each substance).).}}}
		\SelectionItem
			{Solute-AdvectionDiffusion-DG-output-fields::diff-m}
			{diff{\_}m}
			{{{Output of the field diff{\_}m }{$[m^{2}s^{-1}]$}{ (Molecular diffusivity (for each substance).).}}}
		\SelectionItem
			{Solute-AdvectionDiffusion-DG-output-fields::rock-density}
			{rock{\_}density}
			{{{Output of the field rock{\_}density }{$[m^{-3}kg]$}{ (Rock matrix density.).}}}
		\SelectionItem
			{Solute-AdvectionDiffusion-DG-output-fields::sorption-mult}
			{sorption{\_}mult}
			{{{Output of the field sorption{\_}mult }{$[kg^{-1}mol]$}{ (Coefficient of linear sorption.).}}}
		\SelectionItem
			{Solute-AdvectionDiffusion-DG-output-fields::conc}
			{conc}
			{{{Output of the field conc }{$[m^{-3}kg]$}{.}}}
		\SelectionItem
			{Solute-AdvectionDiffusion-DG-output-fields::fracture-sigma}
			{fracture{\_}sigma}
			{{{Output of the field fracture{\_}sigma }{$[-]$}{ (Coefficient of diffusive transfer through fractures (for each substance).).}}}
		\SelectionItem
			{Solute-AdvectionDiffusion-DG-output-fields::dg-penalty}
			{dg{\_}penalty}
			{{{Output of the field dg{\_}penalty }{$[-]$}{ (Penalty parameter influencing the discontinuity of the solution (for each substance). Its default value 1 is sufficient in most cases. Higher value diminishes the inter-element jumps.).}}}
		\SelectionItem
			{Solute-AdvectionDiffusion-DG-output-fields::region-id}
			{region{\_}id}
			{{{Output of the field region{\_}id }{$[-]$}{.}}}
\end{SelectionType}
\begin{SelectionType}
	{IT::Solute-Advection-FV-output-fields}
	{Solute{\_}Advection{\_}FV{\_}output{\_}fields}
	{{{Selection of output fields for the Solute{\_}Advection{\_}FV model.}}}
		\SelectionItem
			{Solute-Advection-FV-output-fields::porosity}
			{porosity}
			{{{Output of the field porosity }{$[-]$}{ (Mobile porosity).}}}
		\SelectionItem
			{Solute-Advection-FV-output-fields::water-content}
			{water{\_}content}
			{{{Output of the field water{\_}content }{$[-]$}{ (INTERNAL - water content passed from unsaturated Darcy).}}}
		\SelectionItem
			{Solute-Advection-FV-output-fields::sources-density}
			{sources{\_}density}
			{{{Output of the field sources{\_}density }{$[m^{-3}kgs^{-1}]$}{ (Density of concentration sources.).}}}
		\SelectionItem
			{Solute-Advection-FV-output-fields::sources-sigma}
			{sources{\_}sigma}
			{{{Output of the field sources{\_}sigma }{$[s^{-1}]$}{ (Concentration flux.).}}}
		\SelectionItem
			{Solute-Advection-FV-output-fields::sources-conc}
			{sources{\_}conc}
			{{{Output of the field sources{\_}conc }{$[m^{-3}kg]$}{ (Concentration sources threshold.).}}}
		\SelectionItem
			{Solute-Advection-FV-output-fields::init-conc}
			{init{\_}conc}
			{{{Output of the field init{\_}conc }{$[m^{-3}kg]$}{ (Initial concentrations.).}}}
		\SelectionItem
			{Solute-Advection-FV-output-fields::conc}
			{conc}
			{{{Output of the field conc }{$[m^{-3}kg]$}{.}}}
		\SelectionItem
			{Solute-Advection-FV-output-fields::region-id}
			{region{\_}id}
			{{{Output of the field region{\_}id }{$[-]$}{.}}}
\end{SelectionType}
\begin{SelectionType}
	{IT::SorptionImmobile-output-fields}
	{SorptionImmobile{\_}output{\_}fields}
	{{{Selection of output fields for the SorptionImmobile model.}}}
		\SelectionItem
			{SorptionImmobile-output-fields::rock-density}
			{rock{\_}density}
			{{{Output of the field rock{\_}density }{$[m^{-3}kg]$}{ (Rock matrix density.).}}}
		\SelectionItem
			{SorptionImmobile-output-fields::sorption-type}
			{sorption{\_}type}
			{{{Output of the field sorption{\_}type }{$[-]$}{ (Considered sorption is described by selected isotherm. If porosity on an element is equal or even higher than 1.0 (meaning no sorbing surface), then type 'none' will be selected automatically.).}}}
		\SelectionItem
			{SorptionImmobile-output-fields::isotherm-mult}
			{isotherm{\_}mult}
			{{{Output of the field isotherm{\_}mult }{$[kg^{-1}mol]$}{ (Multiplication parameters (k, omega) in either Langmuir c{\_}s = omega * (alpha}\textit{c{\_}a)/(1- alpha}{c{\_}a) or in linear c{\_}s = k * c{\_}a isothermal description.).}}}
		\SelectionItem
			{SorptionImmobile-output-fields::isotherm-other}
			{isotherm{\_}other}
			{{{Output of the field isotherm{\_}other }{$[-]$}{ (Second parameters (alpha, ...) defining isotherm  c{\_}s = omega * (alpha}\textit{c{\_}a)/(1- alpha}{c{\_}a).).}}}
		\SelectionItem
			{SorptionImmobile-output-fields::init-conc-solid}
			{init{\_}conc{\_}solid}
			{{{Output of the field init{\_}conc{\_}solid }{$[kg^{-1}mol]$}{ (Initial solid concentration of substances. Vector, one value for every substance.).}}}
		\SelectionItem
			{SorptionImmobile-output-fields::conc-immobile-solid}
			{conc{\_}immobile{\_}solid}
			{{{Output of the field conc{\_}immobile{\_}solid }{$[m^{-3}kg]$}{.}}}
\end{SelectionType}
\begin{SelectionType}
	{IT::SorptionMobile-output-fields}
	{SorptionMobile{\_}output{\_}fields}
	{{{Selection of output fields for the SorptionMobile model.}}}
		\SelectionItem
			{SorptionMobile-output-fields::rock-density}
			{rock{\_}density}
			{{{Output of the field rock{\_}density }{$[m^{-3}kg]$}{ (Rock matrix density.).}}}
		\SelectionItem
			{SorptionMobile-output-fields::sorption-type}
			{sorption{\_}type}
			{{{Output of the field sorption{\_}type }{$[-]$}{ (Considered sorption is described by selected isotherm. If porosity on an element is equal or even higher than 1.0 (meaning no sorbing surface), then type 'none' will be selected automatically.).}}}
		\SelectionItem
			{SorptionMobile-output-fields::isotherm-mult}
			{isotherm{\_}mult}
			{{{Output of the field isotherm{\_}mult }{$[kg^{-1}mol]$}{ (Multiplication parameters (k, omega) in either Langmuir c{\_}s = omega * (alpha}\textit{c{\_}a)/(1- alpha}{c{\_}a) or in linear c{\_}s = k * c{\_}a isothermal description.).}}}
		\SelectionItem
			{SorptionMobile-output-fields::isotherm-other}
			{isotherm{\_}other}
			{{{Output of the field isotherm{\_}other }{$[-]$}{ (Second parameters (alpha, ...) defining isotherm  c{\_}s = omega * (alpha}\textit{c{\_}a)/(1- alpha}{c{\_}a).).}}}
		\SelectionItem
			{SorptionMobile-output-fields::init-conc-solid}
			{init{\_}conc{\_}solid}
			{{{Output of the field init{\_}conc{\_}solid }{$[kg^{-1}mol]$}{ (Initial solid concentration of substances. Vector, one value for every substance.).}}}
		\SelectionItem
			{SorptionMobile-output-fields::conc-solid}
			{conc{\_}solid}
			{{{Output of the field conc{\_}solid }{$[m^{-3}kg]$}{.}}}
\end{SelectionType}
\begin{SelectionType}
	{IT::SorptionType}
	{SorptionType}
	{}
		\SelectionItem
			{SorptionType::none}
			{none}
			{{{No sorption considered.}}}
		\SelectionItem
			{SorptionType::linear}
			{linear}
			{{{Linear isotherm runs the concentration exchange between liquid and solid.}}}
		\SelectionItem
			{SorptionType::langmuir}
			{langmuir}
			{{{Langmuir isotherm runs the concentration exchange between liquid and solid.}}}
		\SelectionItem
			{SorptionType::freundlich}
			{freundlich}
			{{{Freundlich isotherm runs the concentration exchange between liquid and solid.}}}
\end{SelectionType}
\begin{SelectionType}
	{IT::Sorption-output-fields}
	{Sorption{\_}output{\_}fields}
	{{{Selection of output fields for the Sorption model.}}}
		\SelectionItem
			{Sorption-output-fields::rock-density}
			{rock{\_}density}
			{{{Output of the field rock{\_}density }{$[m^{-3}kg]$}{ (Rock matrix density.).}}}
		\SelectionItem
			{Sorption-output-fields::sorption-type}
			{sorption{\_}type}
			{{{Output of the field sorption{\_}type }{$[-]$}{ (Considered sorption is described by selected isotherm. If porosity on an element is equal or even higher than 1.0 (meaning no sorbing surface), then type 'none' will be selected automatically.).}}}
		\SelectionItem
			{Sorption-output-fields::isotherm-mult}
			{isotherm{\_}mult}
			{{{Output of the field isotherm{\_}mult }{$[kg^{-1}mol]$}{ (Multiplication parameters (k, omega) in either Langmuir c{\_}s = omega * (alpha}\textit{c{\_}a)/(1- alpha}{c{\_}a) or in linear c{\_}s = k * c{\_}a isothermal description.).}}}
		\SelectionItem
			{Sorption-output-fields::isotherm-other}
			{isotherm{\_}other}
			{{{Output of the field isotherm{\_}other }{$[-]$}{ (Second parameters (alpha, ...) defining isotherm  c{\_}s = omega * (alpha}\textit{c{\_}a)/(1- alpha}{c{\_}a).).}}}
		\SelectionItem
			{Sorption-output-fields::init-conc-solid}
			{init{\_}conc{\_}solid}
			{{{Output of the field init{\_}conc{\_}solid }{$[kg^{-1}mol]$}{ (Initial solid concentration of substances. Vector, one value for every substance.).}}}
		\SelectionItem
			{Sorption-output-fields::conc-solid}
			{conc{\_}solid}
			{{{Output of the field conc{\_}solid }{$[m^{-3}kg]$}{.}}}
\end{SelectionType}
\begin{SelectionType}
	{IT::VTK-variant-ascii-or-binary-}
	{VTK variant (ascii or binary)}
	{}
		\SelectionItem
			{VTK-variant-ascii-or-binary-::ascii}
			{ascii}
			{{{ASCII variant of VTK file format}}}
		\SelectionItem
			{VTK-variant-ascii-or-binary-::binary}
			{binary}
			{{{Uncompressed appended binary XML VTK format without usage of base64 encoding of appended data.}}}
		\SelectionItem
			{VTK-variant-ascii-or-binary-::binary-zlib}
			{binary{\_}zlib}
			{{{Appended binary XML VTK format without usage of base64 encoding of appended data. Compressed with ZLib. (Not supported yet)}}}
\end{SelectionType}
