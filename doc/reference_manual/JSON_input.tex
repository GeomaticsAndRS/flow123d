

\chapter{New input files}
The input consists of the root input file given as the parameter on the command line and possibly several other 
files with large input data. In this section we shall describe format of the input files. At first, we specify syntax of
an extension to the JSON file format. Then we set rules for input of more specific data constructs. We continue by description of general scheme for input of
boundary conditions and material time-space variable data. And finally, we describe setting of particular equations and their solvers.

The aim of this draft is twofold. First, we want to outline the way how to translate current input file format (in version 1.6.5) into the new one without extending 
the existing functionality. Second, we want to propose a new way how to input general boundary and material data. Desired features are:
\begin{itemize}
 \item input simple data in simple way
 \item possibility to express very complex input data 
 \item possibility to generate data automatically, and input very large input sets
 \item input interface that provides uniform access to the data in program independent of the input format
\end{itemize}
[ ... something else?]

As this is a draft version there are lot of remarks, suggestions and questions in square brackets. Some keys are marked OBSOLETE, which means that 
we want to replace them by something else.

\section{Root file format}
The root input file is in the Humanized JSON file format. That is the JSON file format with few syntax extensions and several semantic rules particular to
Flow123d. The syntax extensions are
\begin{enumerate}
\item You can use one line comments using hash \verb'#'.
\item The quoting of the keys is optional if they do not contain spaces (holds for all Flow keys).
\item You can use equality sign \verb'=' instead of colon \verb':' for separation of keys and values in JSON objects.
\item You can use any whitespace to separate tokens in JSON object or JSON array.
\end{enumerate}
The aim of these extensions is to simplify writing input files manually. However these extensions can be easily filtered out and converted to 
the generic JSON format. (This way it can be also implemented in Flow123d.)

For those who are not familiar with the JSON file format, we give the brief description right here. The full description can be found at
\url{http://www.json.org/}. However, we use term {\it record} in the place of the {\it JSON object} in order to distinguish {\it JSON object}, 
which is merely a data structure written in the text, and the {\it C++ object}, i.e. instance of some class. 

\subsection{Humanized JSON}
The JSON format consists of four kind of basic entities: 
{\it null} token, {\it true} and {\it false} tokens, {\it number}, {\it string}. {\it Number} is either integer or float point number 
possibly in the exponential form and {\it string} is any sequence of characters quoted in \verb'""' (backslash \verb"\" is used as escape character and 
Unicode is supported, see full specification for details). 

In the following, we mean by white space characters:
space, tab, and new line. In particular the newline character (outside of comment or quoting) is just the white space character without any special meaning.

The basic entities can be combined in composed entities, in a {\it record} or in an {\it array}. The {\it record} is set of assignments enclosed in the curly brackets
\begin{verbatim}
{
        #basic syntax
        "some_number":124, 
        "some_string":"Hallo",
        "some_subrecord":{},
        "some_array":[],
        
        #extended syntax
        non_quoted_key_extension=123,
        separation_by_whitespace="a" sbw_1="b"
        sbw_2="c"
}
\end{verbatim}
One assignment is a pair of the {\it key} and the {\it value}. {\it Key} is {\it string} or token matching regexp\\
\verb'[a-zA-Z_][a-zA-Z_0-9]*'.
{\it Value} is basic or composed entity.
The key and the value are separated by the colon (generic syntax) or equality sign (extended syntax). Pairs are separated by a comma (generic syntax) or 
sequence of white space characters (extended syntax). The values stored in the record are accessed through the keys like in an associative array. Records are usually used for initialization of corresponding classes.

The second composed entity is the {\it array} which is sequence of (basic or composed) entities separated by comma (generic syntax) or whitespace sequence 
(extended syntax) and enclosed in the square brackets. 
The values stored in the array are accessed through the order. The Flow reader offers either initialization of a container from JSON array or
a sequential access. The latter one is the only possible access for the included arrays, which we discuss later.

On any place out of the quoted string you can use hash mark \verb'#' 
to start a one line comment. Everything up to the new line will be ignored and replaced by single white space.

[What about multiple line strings? (Should be allowed)]

\subsection{Special keys}
Apart from small extensions of JSON syntax, we impose further general rules on the interpretation of the input files by Flow123d reader.
First, the capital only keywords  
have a special meaning for Flow JSON reader. On the other hand, we use only small caps for keys interpreted through the reader.
The special keywords are:
\begin{description}

\item[TYPE]:\\ 
\begin{verbatim}
TYPE= <enum>
\end{verbatim}
The \verb'<enum>' is particular semantic construct described later on. 
When appears in the record, it specifies which particular class to instantiate. This only has meaning if the record initializes
an abstract class. In consistency with the source code, we shall call such records {\it polymorphic}. 


In fact we consider
that every record is of some {\it type} at least implicitly. The {\it type} of the record is specification of the keys that are
interpreted by the program Flow123d. At some places the program assumes a record of specific {\it type} so you need not to specify 
\verb'TYPE' key in those records.

\item[INCLUDE\_RECORD]:\\
This is a simple inclusion of another file as a content of a record:
\begin{verbatim}
{
        INCLUDE_RECORD = "<file name>"
}
\end{verbatim}

\item[INCLUDE\_ARRAY]:\\
\begin{verbatim}
array=
{
        INCLUDE_ARRAY = "<file name>"
        FORMAT = "<format string>"
}       
\end{verbatim}
The reader will substitute the include record by a sequentially accessible array. The file has fixed number of 
space separated data fields on every line. Every line becomes one element in the array of type record. Every line forms a 
record with key names given by the \verb'<format string>' 
and corresponding data taken form the line.

The key difference compared to regular JSON arrays is that included arrays can be accessed only sequentially 
within the program and thus we minimize reader memory overhead for large input data. The idea is to translate raw data into structured
format and use uniform access to the data.

Basic syntax for format string could be an array of strings --- formats of individual columns.
Every format string is an address of key that is given the column. Onother possibility is to give an arbitrary 
JSON file, where all values are numbers of columns where to take the value.

[\dots better specify format string]


[Possible extensions:
- have sections in the file for setting time dependent data
- have number of lines at the beginning
- have variable format
- allow vectors in the 'line records']

\item[REFERENCE]:\\
\begin{verbatim}
time_governor={
  REF=<address>
}
\end{verbatim}
This will set key \verb'time_governor' to the same value as the entity specified by the address.
The address is an array of strings for keys and integers for indices.
The address can be absolute or relative identification of an entity. The relative address is relative to the entity in which the reference record is contained.
One can use string \verb'".."' to move to parent entity and string \verb'"//"' to move to the root record of current file.
Indices in address starts from 0.

For example assume the file
\begin{verbatim}
mesh={
        file_name="xyz"
}
array=[
        {x=1, y=0}       
        {x=2, y=0}
        {x=3, y=0}
]               
outer_record={
        output_file="x_out"
        inner_record={
                output_file={REF=["..","output_file"]}  # value "x_out"
        }
        x={REF="/array/2/x"}                                    # value "3"
        f_name={REF=["//","mesh","file_name"]}                  # value "xyz"
}       
\end{verbatim}

Concept of addres should be better explained and used consistently in reader interface.
\end{description}

\subsection{Semantic rules}

\subsubsection{Implicit creation of composed entities}
Consider that there is a {\it type} of record in which all keys have default values (possibly except one). Then the specification
of the record {\it type} can contain a {\it default key}. Then user can use the value of the {\it default key} instead of the whole record.
All other keys apart from the {\it default key} will be initialized by default values. 
This allows to express simple tasks by simple inputs but still make complex inputs possible. 
In order to make this working, developers should provide default values everywhere it is possible.
 
Similar functionality holds for arrays. If the user sets a non-array value where an array is expected the reader provides an array with a unique element holding the given value.
See examples in the next section for application of these two rules.



\subsubsection{Enum construct}
Enum values can be integers or strings from particular set. Strings should be preferred for manual creating of input files, while 
the integer constants are suitable for automatic data preparation. 

The input reader provides a way how to define names of members of an enum class and then 
initialize this enum class from input file.  [Need better description]

\subsubsection{String types}
For purpose of this documentation we distinguish several string types with particular purpose and treatment. Those are:
\begin{description}
\item[input filename] This has to be valid absolute or relative path to an existing file. 
The string can contain variable \verb'${INPUT}' %$
which will be replaced by path given at command line through parameter \verb'-i'.

[In order to allow input of  time dependent data in individual files, we should 
 have also variable \verb'${TIME\_LEVEL}' %$
 From user point of view this is not property of general input filename string, however
 in implementation this should be done in the same way as \verb'${INPUT}'.
]
 
[? Shall we allow both Windows and UNIX slashes?]

[Developers should provide default names to all files. ]

\item[output filename] This has to be relative path. The path will be prefixed by
the path given at command line through the parameter \verb'-o'.
In some cases the path will be also postfixed by extension of particular file format.

\item[formula] Expression that will be parsed and evaluated runtime. Documentation of particular key should provide 
variables which can appear in the expression, however in general it can be function of the space coordinates $x$, $y$, $z$ and possibly also 
function of time $t$. For full specification of expression syntax see documentation of FParser library:
\url{http://warp.povusers.org/FunctionParser/fparser.html\#literals}

\item[text string] Just text without particular meaning.
\end{description}

\subsubsection{Record types}
A record type like particular definition of a class (e.g. in C++). One record type serves usually for initialization of 
one particular class. From this point of view one record type is set of keys that corresponding class can read.

For purpose of this manual the record type is given by specification of record's keys, their types, default values and meanings.
In the next two sections, we describe all record types that forms input capabilities of Flow123d. Description of a record type 
has form of table. Table heading consists of the name of the record type. Then for every key we present name, type of the value,
default value and text description of key meaning. Type of the value can be record type, array of record types, double, integer, enum or 
string type. Default value specification can be:
\begin{description} 
 \item[none] No default value given, but input is mandatory. You get an error if you don't set this key.
 \item[null] \verb'null' value. No particular default value, but you need not to set the key. Usually means feature turned off.
 \item[explicit value] For keys of type: string, double, integer, or enum, the default value is explicit value of this type.
 \item[type defaults] For keys of some record type we let that record to set its default values.
\end{description}

[? polymorphic record types]


\section{Record types for input of data fields}


\newenvironment{recordtype}[2]
{\par
 \vskip 2ex
 \noindent%
 record type: {\bf #1}#2
 \par%
 \vskip 0.5ex
 \hrule%
 \vskip 0.3ex
 \hrule
 \begingroup%records of steady field type which we describe right now 
  \addtolength{\leftskip}{3em}%
  %
  \gdef\keyitem##1##2##3{%
    \par
    \vskip 0.3ex
    %\hrule%
    \noindent%
    \hspace{-3em}{\bf\tt ##1} = {\it \textless ##2\textgreater} \hfill \makebox[0.4\textwidth][l]{DEFAULT: {##3}\hfil}%
    \par
  }%
}{%
  \vskip 0.7ex  
  \hrule%
  \vskip 0.5ex
  \endgroup%
}

\def\enumhead#1{
{\bf #1} enum cases:
}
\def\enumitem#1#2{
  \par{\tt #1=#2 \hspace{2em}}
}

In this section we describe record types used to describe general time-space scalar, vector, or tensor fields
and records for prescription of boundary conditions. Since one possibility how to prescribe input data fields is by 
discrete function spaces on computational mesh, we begin with mesh setting.

\subsection{Mesh type}
The mesh record and should provide a mesh consisting of points, lines, triangles and tetrahedrons in 3D space and further
definition of boundary segments and element connectivity.

\begin{recordtype}{Mesh}{}

\keyitem{file}{input filename}{mesh.msh}
The file with computational mesh in the ASCII GMSH format.\\
\url{http://geuz.org/gmsh/doc/texinfo/gmsh.html#MSH-ASCII-file-format}

\keyitem{boundary\_segments}{array of boundary segments}{null optional}
The set of 0,1, or 2 dimensional boundary faces of the mesh should be partitioned into boundary segments in order to prescribe unique boundary condition 
on every boundary face. 
The segments numbers are assigned to boundary faces by iterating through the array. Initially every boundary face has segment number $0$. 
Every record in the array use ``auxiliary'' physical domains, elements or direct face specification to specify some set of boundary faces. The new segment number
is assigned to each face in the set, possibly overwriting previous value.

Physical domains or its parts that appears in the boundary segment definitions are removed form the computational mesh, however, the element
numbers of removed elements are stored in the corresponding boundary face and can be used to define face-wise approximations of functions with 
support on the boundary.

\keyitem{neighbouring}{input file name}{neigbours.flw}
This should be removed as soon as we integrate ngh functionality into flow.
\end{recordtype}

\begin{recordtype}{Boundary segment}{}
 \keyitem{index}{integer}{index in outer array}
 The index of boundary segment can be used later to prescribe particular type of boundary condition on it.
 Indices must be greater or equal to $1$ and should form more or less continuous sequence. The zero boundary segment is reserved for remaining part
 of the boundary. By default we assign indices to boundary segments according to the order in their array in mesh, i.e. index (counted form $0$) plus $1$.
 \keyitem{physical\_domains}{array of integers}{null optional}
 Numbers of physical domains which form the boundary segment. All elements of these physical domains will be removed from actual computational mesh.
 \keyitem{elements}{array of integers}{null optional}
 The array contains element numbers which should be removed from computational mesh and added to boundary segment.
 \keyitem{sides}{array of integer pairs}{null optional}
 The array contains numbers of elements which outer faces will be added to the boundary segment or pairs $[element, side\_on\   _element]$ 
 identifying individual faces that will be added to the boundary segment.
\end{recordtype}

\subsection{Time-space field type}
A general time and space dependent, scalar, vector, or  tensor valued function is given by array
of {\it steady field data}, i.e. time slices. The time slice contains array of {\it space functions}
for individual materials. Then, the {\it space function} can by either analytical (given by formula)
or numerical, given by type of discrete space and array of {\it elemental functions}. {\it Elemental function} is
just array of values for every degree of freedom on one element.

The function described by this type is tensor values in general and dimensions of this tensor should be 
specified outside of the function data. For example in description of Transport record type you should specify that function
for initial condition is vector valued with vector size equal to number of substances. It is like template for Time-space field
type parametrized by shape of the value tensor given by number of lines $N$ and columnes $N$.


\begin{recordtype}{Steady field data}{}
\keyitem{time}{double}{-Inf}
  Start time for the spatial filed data. 
\keyitem{time\_interpolation}{enum}{constant}
  \enumhead{time interpolation}
  \enumitem{constant}{0} Keeps constant data until next time cut.
  \enumitem{linear}{1} Linear interpolation between current time cut and the next one.
\keyitem{materials}{array of Steady spatial functions}{null optional}
\end{recordtype}

\begin{recordtype}{Space function}{}
\keyitem{material}{integer}{0}
Material filter. The function has nonzero value only on elements with given material number. 
Function with filter $0$ takes place for all materials where no function is set.

\keyitem{analytic}{multi-array of function formulas}{null optional}
The shape of the multi-array is given by 
rank of the value of the function. Since formula parser can deal only with scalar functions, we
have to specify individual members of resulting tensor. Formulas can contain variables $x$, $y$, $z$, and $t$. 
The formula is used until the next time slice and is evaluated for every solved time step (can depend on equation).

Instead of constant formulas one can use double values.
Usage of formulas or doubles need not to be uniform over the tensor.


\keyitem{numeric\_type}{enum}{None}
  \enumhead{FE space}
  \enumitem{None}{0} Use analytic function.
  \enumitem{P0}{1} Zero order polynomial on element.

Currently we support only P0 base functions for data.

\keyitem{numeric}{array of element functions}{empty array}
Usually, this element-wise array should by included from an external file through \verb'INCLUDE_ARRAY' construct.

\end{recordtype}

\begin{recordtype}{Element function}{}
\keyitem{element\_id}{integer}{null optional}
Element ID in the mesh. By default the element is 
identified by the index in the array of element function.

\keyitem{values}{multi-array of doubles}{null mandatory}
Values for degrees of freedom of the base functions on the element. Currently we support 
only P0 functions which are given by value in the barycenter of the element. 
In general the value can be tensor, i.e. array of arrays of doubles.
However, in accordance to simplification rules, you can use only array of doubles for 
vector functions or mere double for scalar functions.
\end{recordtype}

[ tensor/vector valued element function should be given also as simple array of DOF values. But
  we have to provide ordering of tensor products of FE spaces]

[ As follows from next examples, there is no way how to simply set simply tensors.  We can introduce automatic conversion form scalar to  vector (constant vector) and vector to tensor 
(diagonal tensor). ]

[ we should also allow 'in place' array includes to simplify material tables etc.]

[How to allow parallel input of field data?]

[Should be there explicit mesh reference in the field specification?]


Examples:
\begin{verbatim}
constant_scalar_function = 1.0
# is same as
constant_scalar_function = {
  time = -Inf,
  time_interpolation= constant,
  materials = [
    {
      rank=0
      numeric_type=None
      analytic=1.0       # the only key withou default value
    }
  ]
}

conductivity_tensor = 
  [{ material = 1,
     rank = 2,
     analytic = [[1.0, 0.0, 0.0], [0.0, 1.0, 0.0], [0.0, 0.0, 1.0]]
   },
   { material = 2,
     rank = 2,
     analytic = [[10.0, 0.0, 0.0], [0.0, 10.0, 0.0], [0.0, 0.0, 10.0]]
   }
  ]
 
\end{verbatim}

\subsection{Boundary conditions}
Input of boundary conditions is similar to the Time-space fields. For description of one time slice we have 
type {\it steady boundary data}. This contains array of {\it boundary conditions} for individual boundary segments. 
{\it Boundary condition} is given by type and parameters that are analytic or numeric functions. However, numeric functions
are considered only on boundary elements.

\begin{recordtype}{steady boundary data}{}
\keyitem{time}{double}{0.0}
Time when the BC should be applied.
\keyitem{bc}{array of boundary conditions}{empty} 
\end{recordtype}

\begin{recordtype}{boundary condition}{}
 \keyitem{boundary\_segment}{integer}{0}
  Boundary segment where the boundary condition will be applied.
 \keyitem{bc\_type}{enum}{dirichlet}
  Currently there are just three types of boundary condition common to all equations, 
  but some equations can implement some specific boundary conditions. Common boundary condition  types are
  \enumhead{BC types}
  \enumitem{dirichlet}{0}
  \enumitem{neumann}{1}
  \enumitem{newton}{2} (also known as Robin boundary condition)


\keyitem{value}{space function}{type defaults}
Prescribed value for Dirichlet boundary condition and Newton boundary condition, the normal flux for the Neumann boundary condition.
Key \verb'material' of {\it space function} is irrelevant.

\keyitem{mean\_value}{scalar or vector constant}{0}
Prescribes flux for Neumann boundary condition by total flux over the boundary segment. If both {\it value} and {\it mean\_value} keys
are set, we use only {\it value} key. 
[How this interact with fracture opening?]


\keyitem{newton\_coef}{scalar space function}{type defaults}
Coefficient that appears in the Newton boundary condition. Key \verb'material' of {\it space function} is irrelevant.
\end{recordtype}

E.g. denoting $u$ the unknown scalar or vector field and $\prtl_n u$ density of the normal flux of this field,
the meaning of keys \verb'value' and \verb'newton_coef' is following:
$\vc{v}$ $\grad v$
\begin{align*}
 u &:= value &&\text{for Dirichlet boundary} \\
 A\grad u \cdot \vc{n} &:= value && \text{for Neumann  boundary} \\
A\grad u \cdot \vc{n} &:= newton_coef ( u - value ) && \text{for Newton boundary}
\end{align*}

Specific interpretation of the boundary conditions should be described in particular equations.

[How to allow both analytic  and numerical functions here?]

[ In order to allow changing BC type in time, the structure has to be: time array of BC segments array of BC type with data
alternatively we can have just one array of BC patches, where one patch has: time, BC segment, BC type, BC data
patches with non monotone time will be scratched]

[need vector valued Dirichlet (and Neuman, and Newton) for Transport boundary conditions]

[More examples...]


\section{Other record types recognized by Flow123d}

\subsection{Record types not related to equations}

\begin{recordtype}{Root record}{}
  \keyitem{system}{system type}{type defaults}
  Record with application setting.
  \keyitem{problem}{problem type}{null mandatory}
  Record with numerical problem to solve. 
  \keyitem{material}{input filename}{material.flw}
    Old material file still used for initialization of data fields in equations. 
    This should be replaced by material database type. Then certain input data fields in equations can be 
    constructed from material informations. Main obstacle are various adsorption algorithms depending on material number.
\end{recordtype}

Only these  keys are recognized directly at main level, however you can put here your own keys and then reference
to them. For example \verb'mesh' is part of problem type record, but you can put it on the main level and use reference 
inside \verb'problem'. 

[Should we put problem record to the main level?]

[Should we provide ``material database''? Possibility to specify properties of individual materials and use them to construct
 field data for equations.]

\begin{recordtype}{System}{}
\keyitem{pause\_after\_run}{bool}{no}
Wait for press of Enter after run. Good for Windows users, but dangerous for batch computations. 
Should be rather an command-line option.
\keyitem{verbosity}{bool}{no}
Turns on/off more verbose mode. 
\keyitem{output\_streams}{output stream}{null optional}
One or more output data streams.
There are two predefined output streams:


vtk ascii stream:
\begin{verbatim}
{
  name="dafault_vtk_ascii"
  file="flow_output"
  type="vtk_ascii"
}
\end{verbatim}

GMSH ascii stream:
\begin{verbatim}
{
  name="dafault_gmsh_ascii"
  file="flow_output"
  type="gmsh_ascii"
}
\end{verbatim}

\end{recordtype}

Possibly here could be variables for check-pointing, debugging, timers etc.

\begin{recordtype}{Output stream}{}
 \keyitem{name}{string}{null mandatory}
  Name of the output stream. This is used to set output stream for 
  individual output data. 
 \keyitem{file}{output filename string}{stream name}
  File name of the output file for the stream. The file name should be without extension, the correct extension
  will be appended according to the format type.
 \keyitem{format}{enum}{vtk\_ascii}
   \enumhead{output format}
   \enumitem{vtk\_ascii}{0} 
   \enumitem{gmsh\_ascii}{1}
 \keyitem{precision}{integer}{8}
 Number of valid decadic digits to output floating point data into the ascii file formats.
 \keyitem{copy\_file}{output filename string}{null optional}
 Optionally one can set copy file to output data into to different file formats.
 \keyitem{copy\_format}{enum}{null optional}
 \keyitem{copy\_precision}{integer}{8}
\end{recordtype}


\subsection{Equation related record types}
Up to now there is only one problem type: \verb'TYPE=sequential\_coupling', 
but in near future we should introduce full coupling e.g. for density driven flow.

The {\emph sequential coupling problem} has following keys:
\begin{recordtype}{Sequential coupling}{ implements {\it Problem type} }
\keyitem{description}{string}{null optional}
Short text description of solved problem. Now it is only reported on the screen, but could be 
written into output files or used somewhere else.
%
\keyitem{mesh}{mesh type}{type defaults}
The computational mesh common for both coupled equations.
%
\keyitem{time\_governor}{time governor type}{type defaults}
Common time governor  setting.
[ Future: allow different setting for each equation]
%
\keyitem{primary\_equation}{darcy flow type}{type defaults}
Independent equation.
%
\keyitem{secondary\_equation}{transport type}{null optional}
Equation with some data dependent on the primary equation.
\end{recordtype}

\begin{recordtype}{Time governor}{}
\keyitem{init\_time}{double}{0.0}
Time when an equation starts its simulation.
\keyitem{time\_step}{double}{1.0}
Initial time step.
\keyitem{end\_time}{double}{1.0}
End time for an equation. 
\end{recordtype}

This record type should initialize \verb'TimeGovernor' class, but there are still questions about steady TimeGovernor and 
if we allow user setting for other parameters.

There are three subtypes {\it steady saturated MH}, {\it unsteady saturated MH}, {\it unsteady saturated LMH}
The common keys are:

\begin{recordtype}{Darcy flow}{(abstract type)}
\keyitem{TYPE}{enum}{}
  There are three implementations of Darcy flow. Most keys are common but unsteady solvers accept some extra keys.
  \enumhead{darcy flow type}
  \enumitem{steady\_MH}{0}
  \enumitem{unsteady\_LMH}{1}
  \enumitem{unsteady\_MH}{2}
\keyitem{sources}{time-space field}{0} Density of water sources. Scalar valued field (1x1 tensor).
\keyitem{sources\_file}{input file name}{null optional}
File with sources in old format. (OBSOLETE)
%
\keyitem{coef\_tensor}{tensor steady field}{1.0}
Conductivity 3x3 tensor. [Should be always 3x3 and then restricted on 2d and 1d fractures.]
\keyitem{boundary\_condition}{array of steady boundary data}{null mandatory} 
New scheme for setting boundary conditions.
\keyitem{boundary\_file}{input file name}{null mandatory}
File with boundary conditions in old format. (OBSOLETE)
%
\keyitem{solver}{solver type}{type defaults}
\keyitem{n\_schurs}{integer}{2}
Number of Schur complements to make. Valid values are 0,1,2.
%
\keyitem{output}{darcy flow output type}{typ defaults}
This is just sub record to separate output setting. 
%
\keyitem{initial}{steady data type}{null mandatory}
Initial condition. Scalar valued field (1x1 tensor). (for unsteady only)
\keyitem{initial\_file}{input file name}{null mandatory}
File with initial condition in old format. (OBSOLETE)
%
\keyitem{storativity}{steady data type}{null mandatory}   
Storativity coefficient. Scalar valued field (1x1 tensor). (for unsteady only)
\end{recordtype}

\begin{recordtype}{Darcy flow output}{}
\keyitem{save\_step}{double}{null optional}
Time step between outputs.
\keyitem{output\_times}{array of doubles}{null optional}
Force output in prescribed times. Can be combined with regular otuptu given by \verb'save_step'.
\keyitem{velocity\_p0}{output stream name}{default\_vtk\_ascii}
\keyitem{pressure\_p0}{output stream name}{default\_vtk\_ascii}
\keyitem{pressure\_p1}{output stream name}{default\_vtk\_ascii} 
\end{recordtype}
 


\subsection{Solver type}
\begin{recordtype}{Solver type}{ (abstract)}
 \keyitem{TYPE}{enum}{petsc}
 \enumhead{solver types}
 \enumitem{petsc}{0} Use any PETSc solver for MPIAIJ matrices.
 \enumitem{bddc}{1} Use BDDC solver (need not to work with every equation).
 %
 \keyitem{accuracy}{double}{solvers defaults}
 Absolute residual tolerance. 
 \keyitem{max\_it}{integer}{1000}
 Maximum number of outer iterations.
 \keyitem{parameters}{string}{null optional}
 String with options for PETSc solvers.
 \keyitem{export\_to\_matlab}{bool}{no}
 Save every solved system in matlab format. Useful for debugging and numerical experiments.
 %
\end{recordtype}


\subsection{Transport type}
\begin{recordtype}{Transport type}{}
 \keyitem{sorption}{bool}{no}
 \keyitem{dual\_porosity}{bool}{no}
 \keyitem{transport\_reactions}{bool}{no}
 What kind of reactions is this? Age of water?
 %
 \keyitem{reactions}{reaction type}{null optional}
 Currently only Semchem is supported. [Interface to Phreaq ...]
 \keyitem{decays}{array of decays}{null optional}
 \keyitem{substances}{array of strings}{null mandatory}
 Names for transported substances. Number of substances is given implicitly by size of the array.
 %
 \keyitem{initial}{steady data type}{null mandatory}
 Vector valued initial condition for mobile phase of all species. 
 \keyitem{initial\_others}{steady data type}{null optional}
 Tensor valued initial condition for immobile, mobile-sorbed, immobile-sorbed phases and all species. (3 x n\_substances).
 [alternatively have separate key for each phase]
 \keyitem{initial\_file}{input file name}{null mandatory}
 File with initial condition in old format. (OBSOLETE)
 %
 \keyitem{boundary\_condition}{array of steady boundary data}{null mandatory} 
 New scheme for setting boundary conditions.
 \keyitem{boundary\_file}{input file name}{null mandatory}
 File with boundary condition in old format. (OBSOLETE)
 For time dependent boundary conditions, the filename is postfixed with 
 number of time level.
 \keyitem{bc\_times}{array of doubles}{null optional}
  Times for changing boundary conditions. If you set this variable, you have to prepare a separate file with boundary conditions for every 
  time in the list. Filenames for individual time level are formed from BC filename by appending underscore and three digits of time level number, e.g. 
  {\tt transport\_bcd\_000, transport\_bcd\_001, etc.} (OBSOLETE)
 \keyitem{output}{transport output}{type defaults}  

\end{recordtype}

\begin{recordtype}{Transport output}{}
\keyitem{save\_step}{double}{null optional}
Time step between outputs.
\keyitem{output\_times}{array of doubles}{null optional}
Force output in prescribed times. Can be combined with regular otuptu given by \verb'save_step'.
\keyitem{mobile\_p0}{output stream name}{default\_vtk\_ascii}
\keyitem{immobile\_p0}{output stream name}{default\_vtk\_ascii}
\keyitem{mobile\_sorbed\_p0}{output stream name}{default\_vtk\_ascii} 
\keyitem{immobile\_sorbed\_p0}{output stream name}{default\_vtk\_ascii} 
\keyitem{mobile\_p1}{output stream name}{default\_vtk\_ascii}
\keyitem{immobile\_p1}{output stream name}{default\_vtk\_ascii}
\keyitem{mobile\_sorbed\_p1}{output stream name}{default\_vtk\_ascii} 
\keyitem{immobile\_sorbed\_p1}{output stream name}{default\_vtk\_ascii} 
\end{recordtype}

\begin{recordtype}{Reaction}{}
% \key{Output\_precission} & \type{int} & 1 &
%Number of decimal places written to output file created by Semchem\_module.
%\\
%\hline
%\key{Number\_of\_further\_species} & \type{int} & 0 &
%Concentrations of these species are not computed, because they are ment to be unexghaustible.
%\\
%\hline
%\key{Temperature} & \type{double} & 0.0 &
%Temperature, one of state variables of the system.
%\\
%\hline
%\key{Temperature\_Gf} & \type{double} & 0.0 &
%Temperature at which Free Gibbs Energy is specified.
%\\
%\hline
%\key{Param\_Afi} & \type{double} & 0.0 &
%Parameter of the Debuy-H\"{u}ckel equation for activity coeficients computation.
%\\
%\hline
%\key{Param\_b} & \type{double} & 0.0 &
%Parameter of the Debuy-H\"{u}ckel equation for activity coeficients computation.
%\\
%\hline
%\key{Epsilon} & \type{double} & 0.0 &
%Epsilon specifies relative norm of residuum estimate to stop numerical algorithms used by Semchem\_module.
%\\
%\hline
%\key{Time\_steps} & \type{int} & 1 &
%Number of transport step subdivisions for Semchem\_module.
%\\
%\hline
%\key{Slow\_kinetics\_substeps} & \type{int} & 0 &
%Number of substeps performed by Runge-Kutta method used for slow kinetics simulation.
%\\
%\hline\\
%\key{Error\_norm\_type} & \type{string} & "Absolute" &
%Through wich kind of norm the error is measured.
%\\
%\hline\\
%\key{Scalling} & \type{boolean} & "No" &
%Type of the problem preconditioning for better convergence of numerical method.
%\\
%\hline\\
%\end{initable}
%\newpage
%\begin{initable}{Aqueous\_species}
%\key{El\_charge} & \type{int} & 0 &
%Electric charge of an Aqueous\_specie particleunder consideration.
%\\
%\hline
%\key{dGf} & \type{double} & 0.0 &
%Free Gibbs Energy valid for TemperatureGf.
%\\
%\hline
%\key{dHf} & \type{double} & 0.0 &
%Enthalpy
%\\
%\hline
%\key{Molar\_mass} & \type{double} & 0.0 &
%Molar mass of Aqueous\_species.
%\\
%\hline
%\end{initable}
%
%\begin{initable}{Further\_species}
%\key{Specie\_name} & \type{string} & "" &
%Name belonging to Further\_specie under consideration.
%\\
%\hline
%\key{dGf} & \type{double} & 0.0 &
%Free Gibbs Energy valid for TemperatureGf.
%\\
%\hline
%\key{dHf} & \type{double} & 0.0 &
%Enthalpy
%\\
%\hline
%\key{Molar\_mass} & \type{double} & 0.0 &
%Molar mass of Further\_species.
%\\
%\hline
%\key{Activity} & \type{double} & 0.0 &
%Activity of Further\_species.
%\\
%\hline
%\end{initable}
%
%\begin{initable}{Reaction\_i}
%\key{Reaction\_type} & \type{string} & "unknown" &
%Type of considered reaction (Equilibrium, Kinetics, Slow\_kinetics).
%\\
%\hline
%\key{Stoichiometry} & \type{int} & 0 &
%Stoichiometric coeficients of species taking part in $i$-th reaction.
%\\
%\hline
%\key{Kinetic\_constant} & \type{double} & 0.0 &
%Kinetic constant for determination of reaction rate.
%\\
%\hline
%\key{Order\_of\_reaction} & \type{int} & 0 &
%Order of kinetic reaction for participating species.
%\\
%\hline
%\key{Equilibrium\_constant} & \type{double} & 0.0 &
%Equilibrium constant defining i-th reaction.
%
\end{recordtype}

\begin{recordtype}{Decay chain}{}
\keyitem{substance\_ids}{array of integers}{empty}
Sequence of $N$ ids of transported substances describing the order of isotopes in the decay chain.
\keyitem{half\_lives}{array of doubles}{empty}
This array of $N-1$ half-lives of individual decays.
If there are no bifurcation key specified, the decay chain is linear $1\to 2 \to 3$.
If there is the bifurcation key, the decay chain is branched $1\to 2, 1\to 3$.
\keyitem{bifurcation}{array of double}{null optional}
Contains $N-1$ probabilities for individual branches of the bifurcation decay.
They should sum to one.
\end{recordtype}


