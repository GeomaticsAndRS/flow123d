% Copyright (C) 2007 Technical University of Liberec.  All rights reserved.
%
% Please make a following refer to Flow123d on your project site if you use the program for any purpose,
% especially for academic research:
% Flow123d, Research Centre: Advanced Remedial Technologies, Technical University of Liberec, Czech Republic
%
% This program is free software; you can redistribute it and/or modify it under the terms
% of the GNU General Public License version 3 as published by the Free Software Foundation.
%
% This program is distributed in the hope that it will be useful, but WITHOUT ANY WARRANTY;
% without even the implied warranty of MERCHANTABILITY or FITNESS FOR A PARTICULAR PURPOSE.
% See the GNU General Public License for more details.
%
% You should have received a copy of the GNU General Public License along with this program; if not,
% write to the Free Software Foundation, Inc., 59 Temple Place - Suite 330, Boston, MA 021110-1307, USA.

\subsection{Neighbouring file format, version 1.0}
The file is divided in two sections, header and data.
The extension {\tt .NGH} is highly recomended for files of this type.
\begin{fileformat}
\$NeighbourFormat\\
  1.0 \vari{file-type} \vari{data-size}\\
\$EndNeighbourFormat\\
\$Neighbours\\
  \vari{number-of-neighbours}\\
  \vari{neighbour-number} \vari{type} \vari{$<$type-specific-data$>$}\\
  \dots\\
\$EndNeighbours\\
\end{fileformat}
where
\begin{description}
 \ditem{file-type}{int} --- is equal 0 for the ASCII file format.
 \ditem{data-size}{int} --- the size of the floating point numbers used in
  the file. Usually \vari{data-size} = sizeof(double).
 \ditem{number-of-neighbours}{int} --- Number of neighbouring defined in the
  file.
 \ditem{neighbour-number}{int} --- is the number (index) of the n-th
  neighbouring. These numbers do not have to be given in a consecutive (or even an
  ordered) way. Each number has to be given only onece, multiple definition
  are treated as inconsistency of the file and cause stopping the
  calculation.
 \ditem{type}{int} --- is type of the neighbouring. 
 \ditem{$<$type-specific-data$>$}{} --- format of this list depends on the
  \vari{type}.
\end{description}
\subsection*{Types of neighbouring and their specific data}
    \begin{description}
      \ditem{type =}{10} --- ``Edge with common nodes'', i.e. sides of
        elements with common nodes. (Possible many elements)
      \ditem{type =}{11} --- ``Edge with specified sides'', i.e. sides of
        the edge are explicitly defined. (Possible many elements)
      \ditem{type =}{20} --- ``Compatible'', i.e. volume of an element with a
        side of another element. (Only two elements)
      \ditem{type =}{30} --- ``Non-compatible'' i.e. volume od an element with
        volume of another element. (Only two elements)
   \end{description}
   \begin{tabular}{|c|l|l|}
      \hline
      \vari{type} & \vari{type-specific-data} & Description \\
      \hline
      \hline
      10 & \vari{n\_elm} \vari{eid1} \vari{eid2} \dots & number of elements
      and their ids \\
      \hline
      11 & \vari{n\_sid} \vari{eid1} \vari{sid1} \vari{eid2} \vari{sid2} \dots
      & number of sides, their elements and local ids \\
      \hline
      20 & \vari{eid1} \vari{eid2} \vari{sid2} \vari{coef} & Elm 1 has to have
      lower dimension\\
      \hline 
      30 & \vari{eid1} \vari{eid2} \vari{coef} & Elm 1 has to have
      lower dimension\\
      \hline
   \end{tabular}

   \vari{coef} is of the {\tt double} type, other variables are {\tt int}s.
\subsection*{Comments concerning {\tt 1-2-3-FLOW}:}
\begin{itemize}
  \item Every inconsistency or error in the {\tt .NGH} file causes stopping
    the calculation. These are especially:
    \begin{itemize}
      \item Multiple usage of the same \vari{neighbour-number}.
      \item Difference between \vari{number-of-neighbours} and actual number
        of data lines.
      \item Reference to nonexisting element.
      \item Nonsence number of side.
    \end{itemize}
  \item The variables \vari{sid?} must be nonegative and lower than the number
    of sides of the particular element. 
\end{itemize}
