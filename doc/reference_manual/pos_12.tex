% Copyright (C) 2007 Technical University of Liberec.  All rights reserved.
%
% Please make a following refer to Flow123d on your project site if you use the program for any purpose,
% especially for academic research:
% Flow123d, Research Centre: Advanced Remedial Technologies, Technical University of Liberec, Czech Republic
%
% This program is free software; you can redistribute it and/or modify it under the terms
% of the GNU General Public License version 3 as published by the Free Software Foundation.
%
% This program is distributed in the hope that it will be useful, but WITHOUT ANY WARRANTY;
% without even the implied warranty of MERCHANTABILITY or FITNESS FOR A PARTICULAR PURPOSE.
% See the GNU General Public License for more details.
%
% You should have received a copy of the GNU General Public License along with this program; if not,
% write to the Free Software Foundation, Inc., 59 Temple Place - Suite 330, Boston, MA 021110-1307, USA.
\parindent = 0pt

\section*{ASCII post-processing file format version 1.2}
File format of this file comes from the GMSH system. 
Following text is copied from the GMSH documentation.\\[0.5em]

{\tt =============== BEGIN OF INSERTED TEXT ===============}\\[0.3em] 

The ASCII post-processing file is divided in several sections: one format
section, enclosed between {\tt \$PostFormat}-{\tt \$EndPostFormat} tags, and
one or more post-processing views, enclosed between
{\tt \$View}-{\tt \$EndView} tags:

\begin{fileformat}
\$PostFormat\\
1.2 \vari{file-type} \vari{data-size}\\
\$EndPostFormat\\
\$View\\
\vari{view-name} \vari{nb-time-steps}\\
\vari{nb-scalar-points} \vari{nb-vector-points} \vari{nb-tensor-points}\\
\vari{nb-scalar-lines} \vari{nb-vector-lines} \vari{nb-tensor-lines}\\
\vari{nb-scalar-triangles} \vari{nb-vector-triangles} \vari{nb-tensor-triangles}\\
\vari{nb-scalar-quadrangles} \vari{nb-vector-quadrangles} \vari{nb-tensor-quadrangles} \\
\vari{nb-scalar-tetrahedra} \vari{nb-vector-tetrahedra} \vari{nb-tensor-tetrahedra} \\
\vari{nb-scalar-hexahedra} \vari{nb-vector-hexahedra} \vari{nb-tensor-hexahedra}\\
\vari{nb-scalar-prisms} \vari{nb-vector-prisms} \vari{nb-tensor-prisms}\\
\vari{nb-scalar-pyramids} \vari{nb-vector-pyramids} \vari{nb-tensor-pyramids}\\
\vari{nb-text2d} \vari{nb-text2d-chars} \vari{nb-text3d} \vari{nb-text3d-chars}\\
$<$\vari{time-step-values}$>$\\
$<$\vari{scalar-point-values}$>$\\
$<$\vari{vector-point-values}$>$\\
$<$\vari{tensor-point-values}$>$\\
$<$\vari{scalar-line-values}$>$\\
$<$\vari{vector-line-values}$>$\\
$<$\vari{tensor-line-values}$>$\\
$<$\vari{scalar-triangle-values}$>$\\
$<$\vari{vector-triangle-values}$>$\\
$<$\vari{tensor-triangle-values}$>$\\
$<$\vari{scalar-quadrangle-values}$>$\\
$<$\vari{vector-quadrangle-values}$>$\\
$<$\vari{tensor-quadrangle-values}$>$\\
$<$\vari{scalar-tetrahedron-values}$>$\\
$<$\vari{vector-tetrahedron-values}$>$\\
$<$\vari{tensor-tetrahedron-values}$>$\\
$<$\vari{scalar-hexahedron-values}$>$\\
$<$\vari{vector-hexahedron-values}$>$\\
$<$\vari{tensor-hexahedron-values}$>$\\
$<$\vari{scalar-prism-values}$>$\\
$<$\vari{vector-prism-values}$>$\\
$<$\vari{tensor-prism-values}$>$\\
$<$\vari{scalar-pyramid-values}$>$\\
$<$\vari{vector-pyramid-values}$>$\\
$<$\vari{tensor-pyramid-values}$>$\\
$<$\vari{text2d}$>$ $<$\vari{text2d-chars}$>$\\
$<$\vari{text3d}$>$ $<$\vari{text3d-chars}$>$\\
\$EndView
\end{fileformat}

where:
\begin{description}
\item[\vari{file-type}]
is an integer equal to 0 in the ASCII file format.

\item[\vari{data-size}]
is an integer equal to the size of the floating point numbers used in the
file (usually, \vari{data-size} = sizeof(double)).

\item[\vari{view-name}]
is a string containing the name of the view (max. 256 characters).

\item[\vari{nb-time-steps}]
is an integer giving the number of time steps in the view.

\item[\vari{nb-scalar-points}, \vari{nb-vector-points}, \vari{\dots}]
are integers giving the number of scalar points, vector points,\dots
in the view.

\item[\vari{nb-text2d}, \vari{nb-text3d}]
are integers giving the number of 2D and 3D text strings in the
view. 

\item[\vari{nb-text2d-chars}, \vari{nb-text3d-chars}]
are integers giving the total number of characters in the 2D and 3D strings.

\item[\vari{time-step-values}]
is a list of \vari{nb-time-steps} double precision numbers giving the value
of the time (or any other variable) for which an evolution was saved.

\item[\vari{scalar-point-value}, \vari{vector-point-value}, \vari{\dots}]
are lists of double precision numbers giving the node coordinates and the
values associated with the nodes of the \vari{nb-scalar-points} scalar
points, \vari{nb-vector-points} vector points,\dots, for each of the
\vari{time-step-values}.

For example, \vari{vector-triangle-value} is defined as:
\begin{fileformat}
\vari{coord1-node1} \vari{coord1-node2} \vari{coord1-node3}\\
\vari{coord2-node1} \vari{coord2-node2} \vari{coord2-node3}\\
\vari{coord3-node1} \vari{coord3-node2} \vari{coord3-node3}\\
\vari{comp1-node1-time1} \vari{comp2-node1-time1} \vari{comp3-node1-time1}\\
\vari{comp1-node2-time1} \vari{comp2-node2-time1} \vari{comp3-node2-time1}\\
\vari{comp1-node3-time1} \vari{comp2-node3-time1} \vari{comp3-node3-time1}\\
\vari{comp1-node1-time2} \vari{comp2-node1-time2} \vari{comp3-node1-time2}\\
\vari{comp1-node2-time2} \vari{comp2-node2-time2} \vari{comp3-node2-time2}\\
\vari{comp1-node3-time2} \vari{comp2-node3-time2} \vari{comp3-node3-time2}\\
\dots
\end{fileformat}

\item[\vari{text2d}]
is a list of 4 double precision numbers:
\begin{fileformat}
\vari{coord1} \vari{coord2} \vari{style} \vari{index}
\end{fileformat}
where \vari{coord1} and \vari{coord2} give the coordinates of the leftmost
element of the 2D string in screen coordinates, \vari{index} gives the
starting index of the string in \vari{text2d-chars} and \vari{style} is
currently unused.

\item[\vari{text2d-chars}]
is a list of \vari{nb-text2d-chars} characters. Substrings are separated with
the `$^\wedge$' character (which is a forbidden character in regular strings).

\item[\vari{text3d}]
is a list of 5 double precision numbers
\begin{fileformat}
\vari{coord1} \vari{coord2} \vari{coord3} \vari{style} \vari{index}
\end{fileformat}
where \vari{coord1}, \vari{coord2} and \vari{coord3} give the coordinates of
the leftmost element of the 3D string in model (real world) coordinates,
\vari{index} gives the starting index of the string in \vari{text3d-chars} and
\vari{style} is currently unused.

\item[\vari{text3d-chars}]
is a list of \vari{nb-text3d-chars} chars. Substrings are separated with the
`$^\wedge$' character.
\end{description}
 
{\tt =============== END OF INSERTED TEXT ===============}\\[0.5em]

More information about GMSH can be found at its homepage:\\
{\tt http://www.geuz.org/gmsh/}\\

\subsection*{Comments concerning {\tt FFLOW20}:}
\begin{itemize}
  \item {\tt FFLOW20} generates {\tt .POS} file with four views: Elements'
     pressure, edges' pressure, interelement fluxes and complex view. First
     three views shows "raw data", results obtained by the solver without any
     interpolations, smoothing etc. The fourth view contains data processed in
     this way.
     \begin{description}
       \item[Elements' pressure:] Contains only \vari{scalar-triangle-values}.
         Triangles are the same as the elements of the original mesh. We
         prescribe constant value of the pressure on the element, as it was
         calculated by the solver as the unknown $p$. Therefore, the three
         values on every triangle are the same.
       \item[Edge pressure:]  Contains only \vari{scalar-line-values}. The
         lines are the same as the edges of the elements of the original
         mesh. We prescribe constant value of the pressure on the edge, as it
         was calculated by the solver as the unknown $\lambda$. Therefore, the
         two values on every edge are the same.
       \item[Interelement flux:] Contains \vari{vector-point-values} and
         \vari{scalar-triangle-values}. The \vari{scalar-triangle-values}
         carry no information, all values are set to 0, these are in the file
         only to define a shape of the elements. The points for the
         \vari{vector-point-values} are midpoints of the sides of the
         elements. The vectors are calculated as $u{\bf n}$, where $u$ is
         value of the flux calculated by the solver and ${\bf n}$ is
         normalized vector of outer normal of the element's side.
       \item[Complex view:] Contains \vari{scalar-triangle-values} and
         \vari{vector-point-values}. The \vari{scalar-triangle-values} shows the
         shape of the pressure field. The triangles are the the same as the
         elements of the original mesh. Values of pressure in nodes are
         interpolated from $p$s and $\lambda$s. The \vari{vector-point-values}
         shows the velocity of the flow in the centres of the elements.
     \end{description}
\end{itemize}

