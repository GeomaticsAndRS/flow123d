% conventions:
% crossrefs - typechap:name (type: [s]ection,[e]quation,[d]efinition,[t]heorem(lemma,..))
%                           (chap: [p]hysical background, [m]ath tools, [f]luid, [b]odies, [s]teady)
\documentclass[a4paper]{article}
% ***************************************** PACKAGES
\usepackage{amsmath}
\usepackage{amsfonts}
\usepackage{amssymb}
\usepackage{amsthm}
\usepackage{fancyhdr}
\usepackage{natbib}

% ***************************************** SYMBOLS
\def\abs#1{\lvert#1\rvert}
\def\argdot{{\hspace{0.18em}\cdot\hspace{0.18em}}}
\def\D{{\tn D}}
\def\div{\operatorname{div}}
\def\grad{\nabla}
\def\vc#1{\mathbf{\boldsymbol{#1}}}     % vector
\def\th{\vartheta}
\def\tn#1{{\mathbb{#1}}}    % tensor
\def\Tr{\operatorname{Tr}}
\def\where{\,|\,}
%***************************************************************************



\begin{document}


\section{Advection-diffusion-reaction equation in Flow123d}


\subsection{Physical model}
On the domain $\Omega^d$ of dimension $d\in\{1,2,3\}$, we consider a system of mass balance equations in the following form:
\begin{equation}
    \label{e:ADE}
   \partial_t ( \th c^i) + \div ( \vc q c^i ) - \div (\th \D^i \grad c^i ) = F(c^1,\dots, c^s)  \quad \text{ on } \Omega^d.
\end{equation}
The principal unknown is the concentration $c^i$ $[kg/m^3]$ of a substance $i\in\{1,\dots, s\}$, which means weight of the substance in unit volume of the water.
Other quantities are:
\begin{itemize}
\item $\th$ $[-]$ is the porosity, i.e. fraction of space occupied by water and the total volume.
\item $\vc q$ $[m s^{-1}]$ is the Darcy flux or the \emph{macroscopic} water velocity.
It is related to the \emph{microscopic} water velocity $\vc v$ by the relation $\vc q = \th\vc v$.
\item The hydrodynamic dispersivity tensor $\D^i$ $[m^2 s^{-1}]$ has the form
\[
  \D^i =D_m^i \tau \tn I + \abs{\vc v}\big(\alpha_T^i \tn I + (\alpha_L^i - \alpha_T^i) \big) \frac{\vc v \times \vc v}{\abs{\vc v}^2},
\]
which models (isotropic) molecular diffusion, and dispersion in longitudal and transversal direction to the flow.
Here $D_m^i$ $[m^2 s^{-1}]$ is the molecular diffusion coefficient of the $i$-th substance (usual magnitude in clear water is $10^{-9}$), $\tau=\th^{1/3}$ is the tortuosity (by \cite{millington_quirk}), $\alpha_L^i$ and $\alpha_T^i$ is the longitudal and the transversal dispersivity $[m]$, respectively.

\item The reaction term $F(\dots)$ is currently neglected.
\end{itemize}


In lower dimensions $d=1,2$, equation \eqref{e:ADE} represents transport processes in planar or channel fractures whose cross-cut $\delta^d$ ($[m]$ for 2D and $[m^2]$ for 1D) is negligible with respect to the dimensions of the physical domain.


\paragraph{Boundary conditions.}
The physical boundary $\partial\Omega^d$ is decomposed into two parts:
\begin{align*}
\Gamma_D(t) &= \{\vc x\in \partial\Omega^d\where \vc q(t,\vc x)\cdot\vc n(\vc x)<0\},\\
\Gamma_N(t) &= \{\vc x\in \partial\Omega^d\where \vc q(t,\vc x)\cdot\vc n(\vc x)\ge 0\},
\end{align*}
where $\vc n$ stands for the unit outward normal vector to $\partial\Omega^d$.
On the inflow part $\Gamma_D$, concentrations have to be prescribed (Dirichlet boundary condition):
$$ c^i(t,\vc x) = c^i_D(t,\vc x) \mbox{ on }\Gamma_D(t), $$
while on $\Gamma_N$ we impose homogeneous Neumann boundary condition:
$$ -\D^i(t,\vc x)\nabla c^i(t,\vc x)\cdot\vc n(\vc x) = 0 \mbox{ on }\Gamma_N(t). $$




\paragraph{Communication between dimensions.}
Transport of substances is considered also on interfaces of physical domains with adjacent dimensions.
Denoting $c_{d+1}$, $c_d$ the concentration of a given substance in $\Omega^{d+1}$ and $\Omega^d$, respectively, the comunication on the interface between $\Omega^{d+1}$ and $\Omega^d$ is described by:
\begin{equation}
  \label{e:inter_dim_flux}
  q^c = \sigma^c (c_{d+1} - c_d) + \begin{cases}q^w c_{d+1} & \mbox{ if }q^w\ge 0,\\q^w c_d & \mbox{ if }q^w<0,\end{cases}
\end{equation}
where $q^c$ is the concentration flux from $d+1$ to $d$ dimensions, $\sigma^c$ is a transition parameter, $q^w$ is water flux from $d+1$ to $d$ dimensions.
Equation \eqref{e:inter_dim_flux} is incorporated as a boundary condition for the problem on $\Omega^{d+1}$:
$$ -\D\nabla c_{d+1}\cdot\vc n + q^w c_{d+1} = q^c $$
and a source term in $\Omega^d$:
$$ f^c_d = \frac{\delta_{d+1}}{\delta_d}(\sigma^c+\abs{q^w})(c_{d+1}-c_d). $$


\end{document}




TODO:
\begin{itemize}
  \item Write equations for sorption and integrate them into \eqref{e:ADE}. 
$R_i(\argdot)$ is a ``retardation function'' its a function which includes various types of equilibrium sorption.
It is in general dependent as on the substance $i$ as on the material in particular location in space. 
  
  \item Check physical dimensions.
  \item Physical dimensions of quantities involved in equation \eqref{e:inter_dim_flux}
  \item More precise \eqref{e:inter_dim_flux},  for individual dimensions and including $\delta$ for 2D - 1D interaction.
  \item Like in the flow model, we can consider continuous concentrations on the higher dimension or discontinuous ( this can not be
        emulated by different choice of $\sigma^c$ !!)
  \item boundary conditions
\end{itemize}



\subsection{Numerical solution}
\begin{enumerate}
 \item Explicit solution - technicaly this is only slight modification of the already implemented transport model. One has to add appropriate 
       diffusive flux approximation. Something was done by Sembera nad Jiranek, however there are more possible approximations (further search). Weakness:
       \begin{itemize}
        \item too restrictive condition on timestep size for big diffusion on small elements
	      \[
	         dt \le \min \frac{dx^2}{2D},\qquad dt\le \min \frac{dx}{v}
	      \]
	\dots so the former condition is more restrictive if $\min dx / D \le 2\min 1/v$ \dots Peclet number can not be used since it 
        measure only local balnace between diffusion and convection

        \item persisting problems with CFL condition
        \item problems with approximation of the diffusive flux with anisothropy raising from the dispersion
       \end{itemize}
  \item Implicit solution
  \begin{enumerate}
       \item Finite volumes/ Discontinuous Galerkin - search for suitable approximation of the diffusive flux (as above)
       \item Lumped MH  for diffusion, upwind for convection. Problem with zero or nearly zero $\tn D$. There has to be also implementation without
             diffusion (no whole MH stuff). Problem that with lumping this leads to edge centered finite volumes.
  \end{enumerate}   
  
\end{enumerate}



\bibliographystyle{abbrvnat}
\bibliography{ref}


\end{document}
