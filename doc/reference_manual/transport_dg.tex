% conventions:
% crossrefs - typechap:name (type: [s]ection,[e]quation,[d]efinition,[t]heorem(lemma,..))
%                           (chap: [p]hysical background, [m]ath tools, [f]luid, [b]odies, [s]teady)
\documentclass[a4paper]{article}
% ***************************************** PACKAGES
\usepackage{amsmath}
\usepackage{amsfonts}
\usepackage{amssymb}
\usepackage{amsthm}
\usepackage{fancyhdr}
\usepackage{graphicx}
\usepackage[numbers]{natbib}

% ***************************************** SYMBOLS
\def\abs#1{\lvert#1\rvert}
\def\argdot{{\hspace{0.18em}\cdot\hspace{0.18em}}}
\def\avg#1{\left\{#1\right\}}
\def\D{{\tn D}}
\def\div{\operatorname{div}}
\def\Eh{\mathcal E_h}       % edges of \Th
\def\Ehb{\mathcal E_{h,B}}  % edges of \Th on boundary
\def\Ehcom{\mathcal E_{h,C}}         % edges of \Th on interface with lower dimension
\def\Ehdir{\mathcal E_{h,D}}         % Dirichlet edges of \Th
\def\Ehint{\mathcal E_{h,I}}       % interior edges of \Th
\def\grad{\nabla}
\def\jmp#1{[#1]}
\def\n{\vc n}
\def\vc#1{\mathbf{\boldsymbol{#1}}}     % vector
\def\R{\mathbb R}
\def\sc#1#2{\left(#1,#2\right)}
\def\Th{\mathcal T_h}       % triangulation
\def\th{\vartheta}
\def\tn#1{{\mathbb{#1}}}    % tensor
\def\Tr{\operatorname{Tr}}
\def\wavg#1{\avg{#1}_\omega}
\def\where{\,|\,}
%***************************************************************************



\begin{document}


\section{Numerical solution}

For the numerical approximation of the advection-dispersion equation \eqref{e:ADE} we distinguish whether the dispersion $\D$ is present or not.
Since the true solution has qualitatively different properties, we also choose different numerical methods for each case.

\subsection{Pure advection}

\subsection{Advection with dispersion}

For the general case we use the discontinuous Galerkin space approximation described in \citet{ern_stephansen_zunino} and implicit Euler time discretization.
Let $\tau$, $h$ be the time step and the space discretization parameter, respectively.
We assume that $\Th^d$ is a regular partition of the domain $\Omega^d$ into simplices, $d=1,2,3$.
We define the set $\Eh^d$ of all edges of elements in $\Th$ (triangles for $d=3$, line segments for $d=2$ and points for $d=1$).
Further, $\Ehint^d$, $\Ehb^d$ will stand for interior and boundary edges, respectively, $\Ehdir^d(t)$ for edges that coincide with $\Gamma_D^d(t)$ and $\Ehcom^d$ for edges on interface with $\Omega^{d-1}$.

Let us fix one substance and the space dimension $d$.
At each time instant $t_k=k\tau$ we search for the concentration field $c_d^{h,k}\in V_d^h$, where
$$ V_d^h = \{v:\overline{\Omega^d}\to\R\where v_{|T}\in P_1(T)~\forall T\in\Th^d\} $$
is the space of functions piecewise affine on the elements of $\Th^d$, possibly discontinuous across the element interfaces.
The initial concentration $c^{h,0}_d$ is set to the projection of the initial data $c_0$ onto $V_d^h$.
For $k=1,2,\ldots$, the discrete problem reads:
\begin{equation*}
\sc{\delta^d\th\frac{c_d^{h,k}-c^{h,k-1}_d}\tau}{v}_{\Omega^d} + a^{h,k}_d(c^{h,k}_d,v) = b^{h,k}_d(v) \quad \forall v\in V^h_d.
\end{equation*}
Here $\sc{f}{g}_{\Omega^d}=\int_{\Omega^d} f g$, $c^{h,k-1}_d$ is the solution from the previous time step and the forms $a^{h,k}_d$, $b^{h,k}_d$ are defined as follows:
\begin{multline*}
a^{h,k}_d(u,v) = \sc{\delta^d\th\D\nabla u}{\nabla v}_{\Omega^d}
- \sc{\vc q u}{\nabla v}_{\Omega^d}\\
- \sum_{E\in\Ehint^d}\left(\sc{\wavg{\delta^d\th\D\nabla u}\cdot\n}{\jmp{v}}_E + \theta\sc{\wavg{\delta^d\th\D\nabla v}\cdot\n}{\jmp{u}}_E\right)\\
+ \sum_{E\in\Ehint^d}\sc{\vc q\cdot\n\avg{u}}{\jmp{v}}_E
+ \sum_{E\in\Ehb^d}\sc{\vc q\cdot\n u}{v}_E \\
+ \sum_{E\in\Ehint^d}\gamma_E\sc{\jmp{u}}{\jmp{v}}_E
+ \sum_{E\in\Ehdir^d(t_k)}\gamma_E\sc{u}{v}_E,
\end{multline*}
% 
\begin{equation*}
b^{h,k}_d(v) = \sum_{E\in\Ehdir^d(t_k)}\gamma_E\sc{c_D}{v}_E.
\end{equation*}
For an interior edge $E$ we denote by $T^-(E)$ and $T^+(E)$ the elements sharing $E$.
By $\n$ we mean the unit normal vector to $E$ pointing from $T^-(E)$ towards $T^+(E)$, the inter-element jump is defined as $\jmp{f}=f_{|T^-(E)}-f_{|T^+(E)}$, $\avg{f}=\frac12(f_{|T^-(E)} + f_{|T^+(E)})$ and $\wavg{f}=\omega f_{|T^-(E)} + (1-\omega) f_{|T^+(E)}$ denotes the usual and weighted average of the quantity $f$.
The weight $\omega$ is chosen in a specific way (see \cite{ern_stephansen_zunino} for details) taking into account possible inhomogeneity of $\D$.
The stabilization parameter $\gamma_E>0$ is user dependent; its value affects the inter-element jumps of the solution.
The constant $\theta$ can take the value $-1$, $0$ or $1$, which corresponds to the nonsymmetric, incomplete and symmetric variant of the interior penalty DG method.

% \paragraph{Communication between regions of the same dimension.}
In case that lower dimensional domains ($\Omega^1$, $\Omega^2$) have complex topology, e.g. if there are more triangles sharing one line segment, then we consider ideal mixing, i.e. the concentration entering the edge through every inlet element ($\vc q$ points out of this element) is divided among all outlet elements proportionally to their fluxes.

If there are interfaces between adjacent dimensions, then the following terms are added to the bilinear forms:
\begin{multline*}
a^{h,k}_{d+1,C}((u_{d+1},u_d),v_{d+1})\\
= \sum_{E\in\Ehcom^{d+1}} \sc{\delta_{d+1}\sigma^c(\th_{d+1}u_{d+1}-\th_d u_d)+(\vc q\cdot\n)^+u_{d+1} - (\vc q\cdot\n)^-u_d}{v_{d+1}}_E,
\end{multline*}
\begin{multline*}
a^{h,k}_{d,C}((u_{d+1},u_d),v_d)\\
= -\sum_{E\in\Ehcom^{d+1}} \sc{\delta_{d+1}\sigma^c(\th_{d+1}u_{d+1}-\th_d u_d)+(\vc q\cdot\n)^+u_{d+1} - (\vc q\cdot\n)^-u_d}{v_d}_E.
\end{multline*}
Here $f^+=\max\{0,f\}$ and $f^-=-\min\{0,f\}$ is the positive, negative part, respectively and $\Ehcom^4=\Ehcom^1=\emptyset$.

A priori error estimates for the equation in a single domain are done in \cite{ern_stephansen_zunino}, for a posteriori estimates see \cite{ern2010guaranteed}.

% TODO:
% \begin{itemize}
%   \item Write equations for sorption and integrate them into \eqref{e:ADE}. 
% $R_i(\argdot)$ is a ``retardation function'' its a function which includes various types of equilibrium sorption.
% It is in general dependent as on the substance $i$ as on the material in particular location in space. 
% 
% \end{itemize}
% 
% 
% 
% 
% \begin{enumerate}
%  \item Explicit solution - technicaly this is only slight modification of the already implemented transport model. One has to add appropriate 
%        diffusive flux approximation. Something was done by Sembera nad Jiranek, however there are more possible approximations (further search). Weakness:
%        \begin{itemize}
%         \item too restrictive condition on timestep size for big diffusion on small elements
% 	      \[
% 	         dt \le \min \frac{dx^2}{2D},\qquad dt\le \min \frac{dx}{v}
% 	      \]
% 	\dots so the former condition is more restrictive if $\min dx / D \le 2\min 1/v$ \dots Peclet number can not be used since it 
%         measure only local balnace between diffusion and convection
% 
%         \item persisting problems with CFL condition
%         \item problems with approximation of the diffusive flux with anisothropy raising from the dispersion
%        \end{itemize}
%   \item Implicit solution
%   \begin{enumerate}
%        \item Finite volumes/ Discontinuous Galerkin - search for suitable approximation of the diffusive flux (as above)
%        \item Lumped MH  for diffusion, upwind for convection. Problem with zero or nearly zero $\tn D$. There has to be also implementation without
%              diffusion (no whole MH stuff). Problem that with lumping this leads to edge centered finite volumes.
%   \end{enumerate}   
%   
% \end{enumerate}



\bibliographystyle{abbrvnat}
\bibliography{flow123d_doc}


\end{document}
